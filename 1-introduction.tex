%!TEX root = icse_seet16.tex
\section{Introduction}
%new best practices for software engineering education and training;

%focuses:
%- the contributing student
%- GH to facilitate that
%- student perceptions of the tool
%- best uses of GH

Learning about computer science and software engineering is challenging for many reasons. Students must learn the theory, but they also need to apply what they have learned to practical settings. They learn best by doing, by making mistakes, and by iterating on solutions. And because modern complex systems are not written by developers working in isolation, students need to spend a significant amount of time learning how to collaborate and coordinate with others.

Perhaps in an effort to address some of these challenges, many software engineering and computer science instructors have recently adopted GitHub for hosting their course content or handling assignment submissions. In some cases, they are using it in place of a more traditional learning management environment (such as Moodle).
%
% Peggy:  I think this paragraph is needed....  but I moved it... as not all the reviewers will know what Git or GitHub is...   Be careful it was not copied to later in the paper as Alexey suggested.  I've tweaked it a bit and reordered sentences.
GitHub is a popular social code sharing platform that leverages the Git distributed version control system (DVCS). It encourages an open workflow where collaborators can participate in a number of ways, such as contributing to discussions regarding bugs and features, or making changes to a project and allowing other collaborators to review and accept the work. Millions of people use GitHub for collaboration, and while it was originally designed for software development, there has been recent uptake in a variety of domains\footnote{\url{http://readwrite.com/2013/11/08/seven-ways-to-use-github-that-arent-coding}}.

In a previous study \cite{zagalsky2015emergence}, we investigated how and why \emph{educators} in disciplines such as computer science and statistics use GitHub. We found that they use GitHub as a learning platform because of its open workflow and transparency features, and because it offers educators the ability to reuse and remix course materials (through distributed version control), while offering their students the chance to also contribute to course materials (e.g., through \emph{pull requests}).

The educators in our study mentioned several challenges when using GitHub, notably a steep learning curve and the lack of a knowledge base of best practices around how to use GitHub in teaching. We also found that they felt it was important that their students learn how to use GitHub due to its wide use in industry, and that the open and transparent features GitHub provides would support experiential and collaborative learning in a software engineering context. However, in our first study, we neither validated these expectations nor did we gain insights into the student experience.

In this paper, we turn our attention to understanding how \emph{students} may benefit or face challenges when using GitHub in a pedagogical software engineering context. We wished to investigate whether GitHub helps students learn more effectively and if it opens up learning opportunities that the educators from our first study anticipated (e.g., peer review and easier integration of external learning resources). Such insights can be used by future instructors who may struggle to decide if they should use GitHub in their courses, or if they decide to use it, how they should use GitHub in an educational context. Furthermore, these insights can be beneficial to GitHub's design team (or to the designers of similar tools, such as GitLab or BitBucket) so that they can improve their tools' suitability for an educational context.

% Peggy: Cutting this next bit out, the paragraphs that were here before were quite repetitive, trying to remove that.
%In using GitHub, educators can take advantage of its open workflow and transparency features to facilitate an open, peer review process, bring in external sources of learning, and simplify the process of remixing course materials, among others. Insights from this work can determine how educators can best take advantage of these benefits, as well as helping to shape considerations towards the development of future educationally-focused tools which, similar to GitHub, gives students opportunities to contribute to the learning experience in multiple ways.
%too bias?!?


% Peggy: some of the following could fit in motivation, not sure why it is commented out here?  Am I working on the right version...
%In computer science and software engineering education, the focus has shifted from not just technical skills, but also the development of soft skills such as communication and teamwork \cite{jazayeri2004education}. One such way to develop these skills is to allow students to contribute to each other's learning experiences and to course materials \cite{falkner2012supporting}.

%This concept, called `Contributing Student Pedagogy' \cite{hamer2008contributing}, relies highly on the technology used to facilitate this learning experience.

%It is also important, however, to explore the student perspective and to discern how the use of GitHub as an educational tool might affect students.

%particularly focusing on the collaborative and contributive activities GitHub could enable students to partake in

%Conclusions from this work helped determine how GitHub-like systems can best be used in an educational context, and helped shape considerations towards the development of future educationally-focused tools which, similar to GitHub, gives students opportunities to contribute to the learning experience in multiple ways.
In this paper, we present a longitudinal case study whereby GitHub is used as a learning platform for two distinct project-based software engineering courses. We set out to investigate how GitHub impacted student learning in these courses, as well as to determine the benefits and challenges students experienced while using GitHub as part of their studies. We also gained some insights from the instructor that taught both of the courses in our investigation.

The research questions that shaped our study are:
\begin{enumerate}
\item How do students \textbf{benefit} from using GitHub in their courses?
\item What \textbf{challenges} do students face when GitHub is used by software engineering course instructors?
\item What \textbf{recommendations} can we give to software engineering instructors who wish to use GitHub for teaching?
\end{enumerate}

% Peggy: this next paragraph is quite repetitive too, as Cassie suggests it could go in the abstract.
% NF: Commenting out for now, re-adding to abstract when I get there tomorrow afternoon
% The findings from our case study indicate that software engineering students do benefit from GitHub's transparency features and open workflow. However, students were concerned that since GitHub is not inherently an educational tool, it lacks key features that prevent it from being an ideal tool for this context. Moreover, we highlight the importance of tool literacy and of educating both students and instructors about GitHub's features and workflow, to create an environment that maximizes its benefits in software engineering education. We are just at the start of this adoption curve, so it is important to understand the opportunities and risks using GitHub may introduce to the teaching of software engineering and computer science.
