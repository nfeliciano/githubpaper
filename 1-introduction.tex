%!TEX root = icse_seet16.tex
\section{Introduction}
%new best practices for software engineering education and training;

%focuses:
%- the contributing student
%- GH to facilitate that
%- student perceptions of the tool
%- best uses of GH

GitHub is a social code sharing platform popular amongst software developers. Utilizing the Git distributed version control system (DVCS), GitHub is a tool millions of people use for collaboration. Though primarily used by software developers, it has seen utilization in other areas, such as technical writing\footnote{\url{http://readwrite.com/2013/11/08/seven-ways-to-use-github-that-arent-coding}}. GitHub offers advantages with its open, collaborative workflow, where collaborators can be involved in a project in a number of ways, such as contributing to discussions regarding bugs and features, or making changes to a project itself and allowing other collaborators to review and accept their changes.

Instructors have also began to use GitHub as a tool to assist their classes, utilizing its open workflow and its transparency features to benefit their teaching. In a previous study \cite{zagalsky2015emergence}, we interviewed instructors who were early adopters of using GitHub as a learning platform and extracted their motivations and the benefits and challenges they encountered. These instructors described benefits such as the ability to reuse and remix course materials and the ease in which GitHub facilitated student participation and contributions to the course material. However, while we extracted some benefits and challenges the instructors' students experienced, gathering student perspectives on GitHub as a learning tool may reveal new insights.

%In computer science and software engineering education, the focus has shifted from not just technical skills, but also the development of soft skills such as communication and teamwork \cite{jazayeri2004education}. One such way to develop these skills is to allow students to contribute to each other's learning experiences and to course materials \cite{falkner2012supporting}.

%This concept, called `Contributing Student Pedagogy' \cite{hamer2008contributing}, relies highly on the technology used to facilitate this learning experience.

%It is also important, however, to explore the student perspective and to discern how the use of GitHub as an educational tool might affect students.

%The use of version control systems has become increasingly commonplace in software development. Distributed version control systems (DVCS) in particular are typically used in collaboration and to interact with the global software development community. As such, using DVCS has become an essential skill for software engineers. However, the integration of DVCS into computing education as a way of teaching students these skills has so far gained little attention. In this study, we discover the implications of using a distributed version control system as a course platform.

%GitHub is a social code sharing service and version control system. It is a popular tool for many groups and projects that require collaboration, and has even seen utilization in areas outside software development, such as technical writing\footnote{\url{http://readwrite.com/2013/11/08/seven-ways-to-use-github-that-arent-coding}}. The advantages of GitHub and similar tools include their awareness and transparency features, where collaborators can easily stay informed of others' work \cite{dabbish2012social}. As well, collaborators in a GitHub repository can be involved in a project in a number of ways, such as contributing to discussions regarding bugs and features, or making changes to a project itself and allowing other collaborators to review and accept their changes. This open, collaborative workflow is called `The GitHub Way'\footnote{\url{http://www.wired.com/2013/09/github-for-anything/}} as these features are not necessarily exclusive to GitHub and can be found in other Distributed Version Control Systems (DVCSes) such as BitBucket.

%particularly focusing on the collaborative and contributive activities GitHub could enable students to partake in
This work presents a case study in which GitHub was used as an e-learning tool for project-based, computer science and software engineering courses. The work is largely exploratory in order to discern how GitHub impacted the learning of students in these courses, as well as to learn the benefits and challenges students met or expect to meet with GitHub's use in this context. Conclusions from this work can help determine the viability of GitHub-like systems as educational tools, or can help shape the development of future educationally-focused tools which, similar to GitHub, gives students opportunities to contribute to the learning experience in multiple ways. To do so, we interviewed the course participants (the students and the teaching team) regarding their experiences with using GitHub in their course to gather their perspectives on whether or not the GitHub workflow can support computer science and software engineering courses.

%While the instructors and students reaped many benefits from using GitHub in these courses, there were drawbacks and challenges with using a tool not built for education.

This study aims to explore the use of a tool such as GitHub and its effectiveness in the educational context. These research questions are exploratory in nature:
\begin{enumerate}
\item \textbf{What are student perceptions on the benefits of using GitHub for their courses?}
\item \textbf{What are the challenges students face related to the use of GitHub in their courses?}
\item \textbf{What are student recommendations for instructors wishing to use GitHub in a course?}
\end{enumerate}

Our main contributions include gathering insights into the benefits and weaknesses of using GitHub for computer science and software engineering courses from the point of view of both students and the teaching team. As well, we extract suggestions for ways to use GitHub as a learning platform to maximize the benefits of its use in an educational context.

In this case study, we conducted interviews with the students and the teaching team of two courses, one computer science course and one software engineering course. Our findings indicate that students enjoyed the benefits of GitHub's transparency features and open workflow. However, students expressed concerns that GitHub is not inherently an educational tool and is therefore missing key features that prevent it from being an ideal tool for this context.

% We conducted interviews with the students and the teaching team involved in the courses selected for these cases and followed up with a survey to validate our findings.


%This allows for an approach to learning characterized by a \emph{demand-pull} model rather than a \emph{supply-push} model, and focuses on participation and providing students access to rich learning communities \cite{seely2008open}
