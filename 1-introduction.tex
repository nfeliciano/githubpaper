%!TEX root = icse_seet16.tex
\section{Introduction}
%new best practices for software engineering education and training;

%focuses:
%- the contributing student
%- GH to facilitate that
%- student perceptions of the tool
%- best uses of GH

% Peggy:  I think this paragraph is needed....  otherwise it starts very abruptly and sadly not all the reviewers will know what Git or GitHub is...   you will need to not copy it later as Alexey suggested, if you remove, the paper needs to start with something else.  
GitHub is a popular social code sharing platform that utilizes the Git distributed version control system (DVCS). Millions of people use GitHub for collaboration, and while it was designed for software development, there has been recent uptake in a variety of domains\footnote{\url{http://readwrite.com/2013/11/08/seven-ways-to-use-github-that-arent-coding}}. GitHub encourages an open workflow where collaborators can participate in a number of ways, such as contributing to discussions regarding bugs and features, or making changes to a project and allowing other collaborators to review and accept the work.

Instructors have begun to use GitHub as a classroom tool, utilizing its open workflow and transparency features to improve their teaching. In a previous study \cite{zagalsky2015emergence}, we found ways in which educators benefited from leveraging GitHub as a learning platform, such as the ability to reuse and remix course materials, and the ease in which GitHub facilitates student participation and contributions to course materials. However, instructors discussed issues regarding the learning curve, where there were difficulties learning how to use the tool efficiently, magnified by the lack of a shared knowledge base of suggested practices between educators.

We also discovered benefits and challenges that the educators' students experienced, however, gathering further student perspectives on GitHub as a learning tool may reveal new insights. By investigating a specific case of GitHub use in education and gathering insights on how GitHub benefits students and what issues they might encounter, we are able to produce suggestions for future instructors who wish to use GitHub for their courses. In using GitHub, educators can take advantage of its open workflow and transparency features to facilitate an open, peer review process, bring in external sources of learning, and simplify the process of remixing course materials, among others. Insights from this work can determine how educators can best take advantage of these benefits, as well as helping to shape considerations towards the development of future educationally-focused tools which, similar to GitHub, gives students opportunities to contribute to the learning experience in multiple ways.
%too bias?!?


%NFTODO: New paragraph? for more motivation

%In computer science and software engineering education, the focus has shifted from not just technical skills, but also the development of soft skills such as communication and teamwork \cite{jazayeri2004education}. One such way to develop these skills is to allow students to contribute to each other's learning experiences and to course materials \cite{falkner2012supporting}.

%This concept, called `Contributing Student Pedagogy' \cite{hamer2008contributing}, relies highly on the technology used to facilitate this learning experience.

%It is also important, however, to explore the student perspective and to discern how the use of GitHub as an educational tool might affect students.

%The use of version control systems has become increasingly commonplace in software development. Distributed version control systems (DVCS) in particular are typically used in collaboration and to interact with the global software development community. As such, using DVCS has become an essential skill for software engineers. However, the integration of DVCS into computing education as a way of teaching students these skills has so far gained little attention. In this study, we discover the implications of using a distributed version control system as a course platform.

%GitHub is a social code sharing service and version control system. It is a popular tool for many groups and projects that require collaboration, and has even seen utilization in areas outside software development, such as technical writing\footnote{\url{http://readwrite.com/2013/11/08/seven-ways-to-use-github-that-arent-coding}}. The advantages of GitHub and similar tools include their awareness and transparency features, where collaborators can easily stay informed of others' work \cite{dabbish2012social}. As well, collaborators in a GitHub repository can be involved in a project in a number of ways, such as contributing to discussions regarding bugs and features, or making changes to a project itself and allowing other collaborators to review and accept their changes. This open, collaborative workflow is called `The GitHub Way'\footnote{\url{http://www.wired.com/2013/09/github-for-anything/}} as these features are not necessarily exclusive to GitHub and can be found in other Distributed Version Control Systems (DVCSes) such as BitBucket.

%particularly focusing on the collaborative and contributive activities GitHub could enable students to partake in

%Conclusions from this work helped determine how GitHub-like systems can best be used in an educational context, and helped shape considerations towards the development of future educationally-focused tools which, similar to GitHub, gives students opportunities to contribute to the learning experience in multiple ways.
This work presents a case study of using GitHub as a learning platform for project-based software engineering courses. The work is exploratory in order to discern how GitHub impacted student learning, as well as to learn the benefits and challenges students met or expected to meet by using GitHub in this context. We conducted interviews with students in the course regarding their experiences using GitHub in their courses to gather their perspectives on whether or not the GitHub workflow can support education in computer science and software engineering.%\todo[inline]{your motivation needs more oompf...you need to sell it! why I should spend my valuable time reading this paper? the middle of this paragraph talks about how github-like systems can be used, but what is a github-like system and why do I care that someone would want to use one?}
%students and teachers?

%While the instructors and students reaped many benefits from using GitHub in these courses, there were drawbacks and challenges with using a tool not built for education.

This study aims to explore the use of a tool such as GitHub and its effectiveness in the educational context. The research questions that helped shaped this study are:%\todo[inline]{too repetitive}
\begin{enumerate}
\item \textbf{What are student perceptions on the benefits of using GitHub for their courses?}
\item \textbf{What are the challenges students face related to the use of GitHub in their courses?}
\item \textbf{What are student recommendations for instructors wishing to use GitHub in a course?}
\end{enumerate}

This work discusses the main strengths and weaknesses of using GitHub for software engineering courses from both the student and instructor point of view. Our findings from this case study indicate that students benefited from GitHub's transparency features and open workflow. However, students expressed concerns that GitHub is not inherently an educational tool and is therefore missing key features that prevent it from being an ideal tool for this context. Moreover, we highlight the importance of tool literacy and of educating both students and instructors about GitHub's features and workflow to create an environment that maximizes its benefits in education.
%FROM CASSIE: \todo[inline]{this paragraph should be in an abstract and not the intro}, I don't know if I agree

% We conducted interviews with the students and the teaching team involved in the courses selected for these cases and followed up with a survey to validate our findings.


%This allows for an approach to learning characterized by a \emph{demand-pull} model rather than a \emph{supply-push} model, and focuses on participation and providing students access to rich learning communities \cite{seely2008open}
