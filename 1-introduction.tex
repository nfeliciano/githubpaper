%!TEX root = icse_seet16.tex
\section{Introduction}
%new best practices for software engineering education and training;

%focuses:
%- the contributing student
%- GH to facilitate that
%- student perceptions of the tool
%- best uses of GH

% Peggy:  I think this paragraph is needed....  otherwise it starts very abruptly and sadly not all the reviewers will know what Git or GitHub is...   you will need to not copy it later as Alexey suggested, if you remove, the paper needs to start with something else.  I've tweaked it a bit and reordered sentences. 
GitHub is a popular social code sharing platform that leverages the Git distributed version control system (DVCS). It encourages an open workflow where collaborators can participate in a number of ways, such as contributing to discussions regarding bugs and features, or make changes to a project, allowing other collaborators to review and accept the work.  Millions of people use GitHub for collaboration, and while it was originally designed for software development, there has been recent uptake in a variety of domains\footnote{\url{http://readwrite.com/2013/11/08/seven-ways-to-use-github-that-arent-coding}}. 

Many software engineering and computer science instructors have recently adopted GitHub for hosting their course content, and in some cases even use it exclusively in place of a learning management environment (such as Moodle).  In a previous study \cite{zagalsky2015emergence}, we investigated how \emph{educators} use GitHub. We found that educators use GitHub as a learning platform because of its open workflow and transparency features, and because it offers educators the ability to reuse and remix course materials (through distributed version control), while offering their students the chance to also contribute to course materials (e.g., through pull requests). We also found that there were some challenges experienced by educators, notably a steep learning curve and the lack of a knowledge base of best practices around how to use GitHub in teaching.  We also found that educators deemed it was important for their students to learn how to use GitHub, as it is widely used in industry, and that the open and transparent features GitHub provides would support collaborative learning in a software engineering context. However, in our first study, we neither validated these expectations nor did we gain insights directly from students on their experiences.  
% TODO: I know this first study wasn't just software engineering... should perhaps note that, but many where -- add a footnote?

In this paper, we thus turn our attention to understanding how \emph{students} may benefit or face challenges using GitHub in a pedagogical software engineering context.  We wished to investigate whether GitHub helps students to learn more effectively and if it opens up learning opportunities that the educators we studied in our first study anticipated (e.g., peer review, easier integration of external learning resources). Such insights can be used by future instructors who may struggle to decide if they should use GitHub in their courses, or if they decide to use it, how they should use GitHub in an educational context.   Furthermore, these insights can be beneficial to the designers of GitHub (or to the designers of similar tools, such as GitLab or BitBuckets) so that they can improve their tools' suitability for an educational context. 

% Peggy: Cutting this next bit out, the paragraphs that were here before were quite repetitive, trying to remove that.
%In using GitHub, educators can take advantage of its open workflow and transparency features to facilitate an open, peer review process, bring in external sources of learning, and simplify the process of remixing course materials, among others. Insights from this work can determine how educators can best take advantage of these benefits, as well as helping to shape considerations towards the development of future educationally-focused tools which, similar to GitHub, gives students opportunities to contribute to the learning experience in multiple ways.
%too bias?!?


%NFTODO: New paragraph? for more motivation

% Peggy: some of the following could fit in motivation, not sure why it is commented out here?  Am I working on the right version...
%In computer science and software engineering education, the focus has shifted from not just technical skills, but also the development of soft skills such as communication and teamwork \cite{jazayeri2004education}. One such way to develop these skills is to allow students to contribute to each other's learning experiences and to course materials \cite{falkner2012supporting}.

%This concept, called `Contributing Student Pedagogy' \cite{hamer2008contributing}, relies highly on the technology used to facilitate this learning experience.

%It is also important, however, to explore the student perspective and to discern how the use of GitHub as an educational tool might affect students.

%The use of version control systems has become increasingly commonplace in software development. Distributed version control systems (DVCS) in particular are typically used in collaboration and to interact with the global software development community. As such, using DVCS has become an essential skill for software engineers. However, the integration of DVCS into computing education as a way of teaching students these skills has so far gained little attention. In this study, we discover the implications of using a distributed version control system as a course platform.

%GitHub is a social code sharing service and version control system. It is a popular tool for many groups and projects that require collaboration, and has even seen utilization in areas outside software development, such as technical writing\footnote{\url{http://readwrite.com/2013/11/08/seven-ways-to-use-github-that-arent-coding}}. The advantages of GitHub and similar tools include their awareness and transparency features, where collaborators can easily stay informed of others' work \cite{dabbish2012social}. As well, collaborators in a GitHub repository can be involved in a project in a number of ways, such as contributing to discussions regarding bugs and features, or making changes to a project itself and allowing other collaborators to review and accept their changes. This open, collaborative workflow is called `The GitHub Way'\footnote{\url{http://www.wired.com/2013/09/github-for-anything/}} as these features are not necessarily exclusive to GitHub and can be found in other Distributed Version Control Systems (DVCSes) such as BitBucket.

%particularly focusing on the collaborative and contributive activities GitHub could enable students to partake in

%Conclusions from this work helped determine how GitHub-like systems can best be used in an educational context, and helped shape considerations towards the development of future educationally-focused tools which, similar to GitHub, gives students opportunities to contribute to the learning experience in multiple ways.
In this paper, we present a case study whereby GitHub is used as a learning platform for two distinct project-based software engineering courses. We set out to investigate how GitHub impacted student learning in these courses, as well as to determine the benefits and challenges students experienced while using GitHub in their courses, but we also gained some insights from the instructor that taught both of these courses.
% Peggy: how you did it can probably wait...
%We conducted interviews with students in the course regarding their experiences using GitHub in their courses to gather their perspectives on whether or not the GitHub workflow can support education in computer science and software engineering.
%
%\todo[inline]{your motivation needs more oompf...you need to sell it! why I should spend my valuable time reading this paper? the middle of this paragraph talks about how github-like systems can be used, but what is a github-like system and why do I care that someone would want to use one?}
%students and teachers?
%
%While the instructors and students reaped many benefits from using GitHub in these courses, there were drawbacks and challenges with using a tool not built for education.
%
% Peggy: This next sentence is repetitive... 
% This study aims to explore the use of a tool such as GitHub and its effectiveness in the educational context. 
The research questions that shaped our study are:%\todo[inline]{too repetitive}
\begin{enumerate}
\item What are student perceptions on the \textbf{benefits} of using GitHub for their software engineering courses?
\item What \textbf{challenges} do students face when GitHub is used by software engineering course instructors?
\item What \textbf{recommendations} can we give to software engineering instructors who wish to use GitHub for teaching?
\end{enumerate}

% Peggy: this next paragraph is quite repetitive too, as Cassie suggests it could go in the abstract.
The findings from our case study indicate that students do benefit from GitHub's transparency features and open workflow. However, students were concerned  that since GitHub is not inherently an educational tool, it lacks key features that prevent it from being an ideal tool for this context. Moreover, we highlight the importance of tool literacy and of educating both students and instructors about GitHub's features and workflow, to create an environment that maximizes its benefits in software engineering education.  We are just at the start of this adoption curve, so it is important to understand the opportunities and risks using GitHub may introduce to the teaching of software engineering and computer science.  
%FROM CASSIE: \todo[inline]{this paragraph should be in an abstract and not the intro}, I don't know if I agree

% We conducted interviews with the students and the teaching team involved in the courses selected for these cases and followed up with a survey to validate our findings.


%This allows for an approach to learning characterized by a \emph{demand-pull} model rather than a \emph{supply-push} model, and focuses on participation and providing students access to rich learning communities \cite{seely2008open}
