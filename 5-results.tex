%!TEX root = icse_seet16.tex
\section{Findings}
We present the findings according to each of the research questions posed for this study. For each question, we discuss the main themes that emerged from my analysis, providing relevant participant quotes from the interviews. Each participant quote will be identified depending on which course they were taking, as seen from table \ref{table:interviews:students}.  Before that, we provide some insights how GitHub was used in the courses in our case study.

\subsection{How GitHub was Used in the Courses}
Despite being relatively unfamiliar with GitHub and its features, the course instructor opted to utilize GitHub in the same way for both courses, using its features in three pivotal ways: material dissemination through the course repository, lab work through the `Issues' feature, and project hosting through various repositories. The advanced use cases other instructors described in \cite{zagalsky2015emergence}, such as utilizing pull requests and assignment submissions, were not used for these courses. The main course instructor was aware of some of these features but was not comfortable using GitHub beyond their knowledge.

The main use of GitHub was for material dissemination: the instructor hosted a public repository which all students could access to find the work they had to do for any given week. The instructor would update this repository weekly, adding lab assignments, links to readings, and the student homework for the week. All of the content was organized into a calendar table made with Markdown, and it was visible on the home page of the course repository as a `readme' file. Students could `fork' the repository to request changes to be made if they wished, though this possibility was not advertised to the students explicitly.

The other main use case was the use of the repository's `issues' page, where all labs (2-3 hour long sessions once a week in addition to the course lectures) were hosted. These labs would often involve researching a topic and reporting results, or giving other groups feedback on their projects. A dedicated issue would be created for each lab, similar to a forum post, and students would then make comments on these issues based on their lab work. Students were free to work in groups, and when commenting on an issue, would `@mention' their group members to indicate who the respondents were.

GitHub was also used for students to host their individual or group projects. Although students were not mandated to use GitHub for their course projects, most projects were hosted on GitHub in individual repositories. These repositories were public so others in the course could view the work and give feedback.

In addition to GitHub, the course instructor opted to use a version of the Moodle LMS\footnote{\url{https://moodle.org/}}. CourseSpaces generally allows instructors to make their course content available for students to access and interact with, to enable communication between the instructors and students through forums, to post quizzes, create wikis for a class to edit, and to track student progress and performance. For these courses, CourseSpaces was used for work that the course instructor felt should not be publicly available including student grades and student responses to the course readings.

\subsection{RQ1: What are computer science and software engineering student perceptions on the benefits of using GitHub for their courses?}
In this section, we discuss the benefits that emerged from the students' perspectives from the three main uses of GitHub in their courses: for schedule and material dissemination, for discussions, and for hosting their project work.  %Many of these benefits stem from GitHub being a tool commonly used in industry, as well as from the advantages that Git offers for managing individual and group work. \\

\textbf{Benefit: Gaining Experience with an Industrially Relevant Tool} \\
In the current software development landscape, GitHub is a very popular tool for working collaboratively. As such, it is essential that developers are familiar with either GitHub or other distributed version control systems, particularly when working on collaborative, multi-person projects. Many of the interviewees mentioned that using GitHub in class provided a good introduction to the tool for them, even with just the basic use of GitHub to manage course activities such as material dissemination and discussion. However, hosting their projects on GitHub provided practice for real-life scenarios: \textit{``I think when you go and work in software development too, you should get used to [having] lots of eyes being all over your work; that's just the way it's gonna be, so it's practice before real life.''} [SE8]

% Students came into the course with varying degrees of experience with GitHub, as shown on table \ref{table:interviews:students}. Five interviewees had minimal or no experience using it, while others were very knowledgeable about the tool, either through their own uses, through group projects for other classes, or through co-op jobs. Many, at least those who attended the University of Victoria for the majority of their undergraduate studies, had some experience with Subversion, a different version control tool that is taught in a second-year course.

% Many of the interviewees mentioned that using GitHub in class provided a good introduction to the tool for them, even with just the basic use of GitHub to manage course activities such as material dissemination and discussion: \textit{``I think it's pretty good. I mean one thing is that because I'm using it in class, it's made me learn the tool \ldots and that's where the big takeaway is: that I've been able to transfer those skills, I've done some other projects just on my own time using GitHub.''} [SE2]

% For the most part, students who supported this theme believed that the use of GitHub for their courses and projects helped them experience a style of collaboration that they will encounter often in their careers. In comparison to the use of more traditional LMSes, one student noted why using GitHub might be advantageous for them: \textit{``Well, I like how it's the bonus of more practice of something you're gonna use in industry, whereas none of us are gonna use CourseSpaces or Connex when we're out on a co-op or out on a job.''} [SE3]

% Importantly, however, putting their projects on GitHub provides practice for real-life scenarios. SE8 describes why it was beneficial to have their work publicly available for both classmates and outsiders to see: \textit{``I think when you go and work in software development too, you should get used to [having] lots of eyes being all over your work; that's just the way it's gonna be, so it's practice before real life.''} [SE8]

% Beyond the benefit of using GitHub in programming projects, which is what it was designed for, the basic use of GitHub to manage course activities such as material dissemination and discussion was also beneficial to students as an introduction to the tool, with some caveats. \textit{``It's a good introduction to GitHub as a platform; it might not be a good introduction to Git as a tool. Because there's a lot of wizardry that you can do with Git that you'd never learn just doing what we did here \ldots but definitely a good start to get people using Git.''} [SE11]

Some students were introduced to specific GitHub features that they were not necessarily aware of, features they believed were important to learn. \textit{``This is the first time I've actually used the issues portion of GitHub \ldots So it showed me that portion of the capabilities of GitHub.''} [SE13]

% Out of all the benefits described by students, the benefit of getting an introduction to GitHub and its features was talked about the most, as [SE2, SE3, SE4, SE5, SE6, SE7, SE8, SE11, SE13, CS4] specifically mentioned this benefit. The importance of this benefit is further emphasized by these students asserting their intention to continue using GitHub or to use GitHub even more after the course ends. This benefit was shared by students irrespective of their prior experience with GitHub. \\

% \textbf{Benefit: GitHub as a Portfolio} \\
% Many students believed that using GitHub to host their course projects will be beneficial to them in the future. These students described that hosting their code from other courses or from personal projects on their GitHub accounts benefited them in various ways. For example, SE5 organized their code on GitHub for easy access when helping friends: \textit{``I know that when you're trying to help somebody out, you can always just say `Check out my GitHub', I know I've done that with a few of my buddies \ldots and I don't have to search through my files, it's just on GitHub, and you look on there. It's a good organization tool.''}
%
% Many interviewees shared that GitHub could serve as a type of portfolio where their publicly-hosted projects and code could be used to present to potential employers when job hunting. Many employers nowadays refer to GitHub for hiring purposes\footnote{\url{http://www.cnet.com/news/forget-linkedin-companies-turn-to-github-to-find-tech-talent/}}. In fact, some of the students already experienced job interviewers asking them to show their GitHub accounts: \textit{``I think all three companies that I applied to this semester wanted me to link to my GitHub. So I was really lucky that I had [a class] project on there. And I think when this [course's] project is done too, it'll also be really nice to have up there, after we clean it up.''} [SE6]

% CS2 also shared that interviewers inspected their code during an interview, highlighting the importance of having functional code in one's GitHub account: \textit{``These days I see that employers also want to see your GitHub page. While I was giving an interview for my coop, he did actually go into my GitHub profile and try to compile some of my code, so they do want you to have some online presence on GitHub.''}

%\textit{``Well I believe it's good for future employers. I remember I put directly on my resume saying you can check out the work I've done on GH. I included the link right on there and every person I handed my resume to were just like \'hey, fantastic!\' \ldots it's a good way to get your skillset out there.''} [SE5]

%The ability for students to use GitHub as a portfolio where they can show off their projects to potential employers was, for many, an important benefit of using the tool. There were students who were introduced to GitHub as part of their course, but knew the importance of having work on GitHub. This benefit motivated some students to continue putting their work on GitHub.
% The benefit of using GitHub as a portfolio was shared by [SE5, SE6, SE7, SE8, SE11, SE13, CS3, CS4]. \\


\textbf{Benefit: Supporting Student Contributions to Course Content} \\
According to various instructors \cite{zagalsky2015emergence}, one of the benefits that GitHub offered over traditional Learning Management Systems is the ability for students to make changes, fixes, or suggestions to the course materials via Pull Requests (PRs) that instructors can easily reject or accept.

Throughout the two courses in this case study, three PRs were submitted to make fixes to the materials or to add links to new materials. These pull requests were submitted in the first month of the courses and by only one student who was well-versed in GitHub (and who was registered in both courses). SE1 explains their reasoning: \textit{``I like being able to fix the mistakes that [the course instructor] might make, like with a bad link or something, by making a PR \ldots I really like being able to do that because it makes me feel a little more involved.''}

This style of contribution didn't continue, however, perhaps because these initial pull requests to fix material or add links to other content were not merged quickly enough. When they needed to be immediate, the instructor did not merge them for a day or two. Moreover, this method of contributing to the course is limited because of the type of files GitHub and its \emph{diff} feature supports, where they are optimally plain text files rather than files common to educational materials such as PDFs and Powerpoint presentations.

% Another student described how this hindered more participation of this kind: \textit{``Because we did not have the access. If we had the access, then I think people would have collaborated''} [CS2]

% Although SE6 did not contribute in this manner, they saw the potential advantages of using pull requests as follows: \textit{``I think everybody's had experience with mistakes in the course material. \ldots The alternative is just emailing the prof and asking them to change something \ldots this is always there, and they can always check it to see if there's something. This way someone can actually make the change, all they'd have to do is accept it.''} %SE6 discussed the convenience this feature offers to instructors, as changes are listed on a separate page in the GitHub repository and can be accepted with one click. Of the three PRs submitted to the courses, two other students participated by either trying to accept the PR (and failing), or adding a `+1' to the PR's comments, supporting its acceptance.

% However, this method of contributing to the course is limited because of the types of files GitHub supports.  \\ %Nevertheless, the following students agreed that being able to contribute to course materials via GitHub is a benefit: [SE1, SE3, SE5, SE6, SE10, SE13, CS2, CS4]. \\

\textbf{Benefit: Support for Students to Contribute to Each Other's Work} \\
In these courses, projects were open and visible to other students, which allowed more opportunities for student contribution. This was demonstrated by a student's group working with others:
\textit{``For instance, one [issue] was our script wasn't taking in command line arguments if there were spaces in them properly. And then someone [gave us a suggestion we used]. And then to be able to see what other people are having problems with and give suggestions. Even at one point, they were trying to find refactorings, and we said `hey you can use our tool, it'll help.' ''} [SE3]

Making students host their course projects on GitHub and relying heavily on GitHub in the courses resulted in students looking at each other's work and making contributions in the way of advice or suggestions. While some lab assignments required students to look at other repositories, many reported that they would often peruse other projects outside of the requirements. As well, some students actually utilized code from other groups, so they helped fix issues in the code when necessary. \textit{``I believe that one other group decided for project 2 to use [our project 1] and they made a couple of pull requests I think.''} [SE10]

% \begin{figure}[h!]
%  \caption{A student providing feedback for other students' projects in an `issue'.}
%  \centering
%    \includegraphics[width=0.9\textwidth]{contributing}
%  \label{fig:contributing}
% \end{figure}

% Two specific lab assignments asked students to look at the repositories of other groups and comment on them, an exercise that students found useful, or at the very least, interesting. \textit{``Yeah, I liked getting the comments, I liked knowing that people were kind of checking it out, and I assume they would let me know if I was doing anything horribly wrong, and I didn't get any of those comments, I'm assuming that everything was going alright.''} [SE5] %An example of a student providing feedback can be seen in figure \ref{fig:contributing}, where a student attempted to give three other students helpful advice on their projects.

SE11 extended the feedback the students provided each other to the idea of peer reviewing, where students would judge the work of others and make comments on their work, explaining the benefits of such a system: \textit{``I thought [peer reviews] was the best way to learn actually \ldots It forced you to put yourself in a position where you have to defend what you did, which I think is good for quality because you have to actually care.''} The ability to open up student work to anyone can simplify the peer review process, particularly with the `social coding' emphasis of GitHub where others can make inline comments or corrections themselves.

% Helping other projects, either through discussion or through collaborating on the code, offered students new ways to participate that are unique to GitHub and similar types of systems. Students effectively collaborated with each other with the aim of producing better work, as was noted by [SE2, SE3, SE5, SE7, SE10, SE11, SE12, SE13]. \\

\textbf{Benefit: Keeping Each Other Accountable} \\
One benefit that stemmed from GitHub's transparency features was the ability to see a history of commits to a project. This was cited by some students who used GitHub to manage their group projects---they could easily see if and when their partners submitted work. Their repositories kept an account of when each change was made, which provided collaborators an easy way to track the work being done on the project. This helped the students to keep up with each other's work: \textit{``You can see exactly what the other person has contributed, and you can look it up again a month later \ldots then it's a good way to keep accountable. And it's good for yourself too, because you know they can see your work, so you wanna make sure that it's top notch and easily readable.''} [SE5]

% Moreover, the students knew exactly how much work each member of their group contributed to the project and this helped the student keep themselves and each other accountable: \textit{``We decided to switch to pull requests instead of just committing straight to master, because \ldots for a couple of reasons, first of all, if there's something majorly wrong with it, everyone can see it, right? And the second thing is, everyone sees it, so if people have to work on [the same code], in the future, which we all did, then they know exactly what just went in, so that next time they come to the code and pull it, they're not like `where did this all come from?'\,''} [SE9]

% This is a useful feature to have when working in group work as it allows for awareness between group members. By using GitHub for their group projects, students were able to take advantage of the collaborative features that GitHub offers to improve their processes or their product. This benefit was described by [SE5, SE9, SE11]. \\
%%%http://www.cs.usask.ca/faculty/gutwin/866/2010-T2/readings/p72-gutwin.pdf

\textbf{Benefit: Version Controlled Assignments} \\
% Using version control for their assignments and projects benefited the students in multiple ways. The ability to revert to previous states of the code was useful: \textit{``You're working on a project, and you make a change that breaks everything. Well you can just go back to a different commit, one that works. Boom, fixed, try again.''} [SE11]
%
Although the instructor for these courses did not use their repositories for marking, some students believed the system could allow instructors to give constructive feedback while they built their projects and assignments. One student believed that the ability to see the student's process could be important: \textit{``You'd see all the mistakes they made getting there, too, which is just as important to learning as the finished product.''} [DS3]
\\
% SE8 described a the potential of using GitHub for grading, where instructors could use the student's repositories as submissions as opposed to the traditional way of submitting through an LMS: sending the code only when finished. SE8 said that this way of submission would be \textit{``so much more useful \ldots You could see everybody's contributions, you could comment on them too \ldots Unless you're doing a live code demo with a TA or any instructor, you're not getting any real feedback [with the traditional submission system] \ldots You have no idea where you lost the marks or where you went wrong.''}

\textbf{Benefit: Bring in Outside Sources of Learning} \\
The final benefit that emerged from the interviews relates to the way in which work hosted on GitHub is often publicly available for others to contribute to. For example, SE1 was highly active in the community of a certain programming language, and for their first project, they were building something related to the language. They advertised their work to the community, and members from the community then tried to help with their project in multiple ways: \textit{``So here I have people involved in the discussion. These are just people in the community I've been talking to about how to do different things, and they've been giving me suggestions. And that's really cool because I actually have some community involvement in my course project.''} They also noted how this was helpful: \textit{``But for me, I find it really validating when someone else is like `that is really cool, have you considered doing this?' ''} [SE1]

% SE1 was the only student interviewed who used the public nature of the course projects to solicit outside contributions. However, the exposure to GitHub gave students opportunities to discover work outside of the course and to use other repositories to aid their projects. When prompted, most interviewees mentioned that they sought out public repositories either to pull their code and use it, or to find inspiration for their own projects. One student recalled an experience where their group looked at an open-source library: \textit{``We just looked at how Gitstats, [an open-source library] did it, and then wrote our own thing into our project \ldots I think that more than anything is the biggest reason why Git should be used for education, because it takes, I think, until you start being forced to do it \ldots to actually go and look at other people's code, and I think looking at other people's code is the most important thing.''} [SE6]

% Speculatively, however, this likely would have happened regardless of whether or not GitHub was pushed by the instructor as students tended to seek out other code and libraries for their projects: \textit{``And in industry, the first thing you do is check Stack Overflow, look for someone else who has done the same thing and jack their code.''} [SE7] [SE1, SE2, SE3, SE4, SE6, SE7, SE10, SE12, SE13, CS2, CS5] mentioned looking at outside work and public repositories for their projects.

\subsection{RQ2: What are the challenges students face related to the use of GitHub in their courses?}
This section outlines the challenges the students described relating to GitHub use in courses. Some of these challenges were related to tool literacy, where more knowledge of the tool and more experience using it in an educational context could have mitigated the challenges. Yet they are worth mentioning as potential challenges that students might encounter. \\
%
% \begin{table}[h]
%     \vspace{1pt}
%         \caption{Summary of challenges with using GitHub for education from the student perspective.}\label{table:interviews:students:challenges}
%     \vspace{1pt}
%     \begin{center}
%         \begin{tabular}{ | m{3cm} | m{12cm} | }
%             \hline
%             \emph{Challenges} & \emph{Description} \\
%             \hline
%             Privacy is All-or-Nothing & When student work is publicly available, students saw potential issues such as incomplete or rushed work being judged by others. However, making their projects private might limit some of the benefits from the previous section. \\
%             \hline
%             Lack of Training on Git and GitHub & Students felt it would have benefitted them to have a tutorial session or even further integration of the tool in previous courses so that they may learn how to use GitHub and what benefits it provides. \\
%             \hline
%             Notification Overload & If students opted to receive notifications, they would receive too many notifications and emails, which they perceived as noise. \\
%             \hline
%         \end{tabular}
%     \end{center}
% \end{table}

\textbf{Challenge: Privacy is All-or-Nothing} \\
While publicly sharing student projects on GitHub publicly provided several benefits, others acknowledged that it may not be appropriate for a class environment. SE4 describes this dilemma: \textit{``So [using GitHub for your work has] got benefits and drawbacks: benefits being that other people can access your data, drawbacks being that other people can access your data.''}

Most interviewees didn't mind that the class repository and their project work was public. However, many students could see the potential problems that might surface from their work being publicly available. Students mentioned two issues, 1) that their school work may not be of interest to the public and 2) that although they would ideally put 100\% effort into all their submissions, this is not always realistic due to the time constraints students face. For example, SE6 acknowledged that sometimes, students rush through their work, and therefore, they might not want that work to be publicly available: \textit{``You know it would actually be nice if they were separate or private somehow so I wouldn't have to go through everything and sanitize all the stuff I've submitted, because you know, for as much as you'd want to think you're putting 100\% into it, you're not really, you know, writing some great work of art or careful analysis, so private would be nicer. For things like that.''}

%For example, CS3 noted that although it can be advantageous to host code publicly so that employers are able to see their projects, the employer may not always agree: \textit{``I think it comes back to what do you want to show your employers? When your employer looks at your work, will they understand that work I submitted in Git was when I didn't yet understand what I was doing, I was still learning? \ldots If I could make the assumption that an employer would understand that, I would have no problem with it being public. That said, I can't make that assumption. I have to assume that everything they look at they're judging in the harshest light possible. So I try to show only things that are of quality that I'm proud of. And that's unfortunately not a lot of the classwork until I'm done with it \ldots The final product I'm happy to show, but all those steps getting there, they're often filled with pitfalls and horrible programming and badly factored code.''} As such, courses mandating the use of Git and GitHub to publicly host student work could be problematic for students if employers are looking at work-in-progress in a negative light.

%not everything is 100% effort
% SE6 acknowledged that sometimes, students rush through their work, and therefore, they might not want that work to be publicly available: \textit{``You know it would actually be nice if they were separate or private somehow so I wouldn't have to go through everything and sanitize all the stuff I've submitted, because you know, for as much as you'd want to think you're putting 100\% into it, you're not really, you know, writing some great work of art or careful analysis, so private would be nicer. For things like that.''}

%not everything is of interest to public
% As well, SE1 felt that some of the work in the course repository wouldn't even be of interest to the public or to potential employers, and as such, they saw no need for the repository to be public: \textit{``I'd rather have [our comments] be private. But only because there's not a whole lot of participation, so I don't feel they're of interest to someone publicly.''}

% \emph{Discrepancy:} However, others saw no issue and even preferred all their work be in the public space. \textit{``Personally I don't have a problem with it being public. I would like to have a good online activity of myself on GitHub, so that's not really an issue. I'm not really concerned if someone is going to read my blog or not.''} [CS2]

One student acknowledged that there are workarounds to some of these privacy issues---where students do not have to attach their names to the work they contribute to the GitHub repositories used for the courses. \textit{``I think part of that would be \ldots you can decide that on your own, depending on if you use your main git account or just make a separate git account for your class.''} [SE3] Indeed, one student created a new GitHub account solely for their contributions to the class. Unfortunately, this student did not want to be interviewed, and group members were uncertain as to what the motivation behind creating a new user was; presumably, they were motivated by some of the issues discussed above.

% In summary, although students enjoyed the benefits that came with making their work publicly available, many students also acknowledged that these benefits are accompanied by a number of caveats. Importantly, some students described the lack of a middle ground as a limitation, where in the context of this course, students had to host their work publicly. This challenge may be mitigated by the instructor giving the students the option of creating a new GitHub user account for work done in their courses, as suggested by SE3 above. This challenge was shared by [SE1, SE4, SE5, SE6, SE7, SE10, SE13, CS3, CS6], while [SE3, SE5, CS2, CS5] disagreed on these issues. \\

\textbf{Challenge: Lack of Training on Git and GitHub} \\
% For example, SE9 believed that students who were less experienced with the tool could not take advantage of its benefits, such as the ability to make pull requests on the course materials. SE9 believed that if the instructor did not set a precedent for that behavior, it may not be used: \textit{``I think you just have to advertise it so that the students know [to] use this as a communication tool. And then layout or give some examples on how it could be used.''}

Another issue that many students described related to education and training is that there were varying degrees of experience with and knowledge of GitHub and its features, which presented difficulties with the use of GitHub in these courses. The main course instructor for these two cases was inexperienced with using GitHub, which made it difficult to educate the students on its features and caused some frustration for some of the interviewees. In fact, most students who were asked mentioned that the course could have benefited from more education on Git, GitHub, and what they can do with it. Students said that they could have hosted a lecture or a lab dedicated to learning the tool, perhaps at the beginning of the course or as an extra session. \textit{``I think it would've been good to do some demo \ldots cause I think [the instructor] talked too much about theory in class and there's no actual coding or no actual demoing.''} [DS1]

% SE2 acknowledged the potential difficulties in hosting such a session: \textit{``On the other hand, when someone teaches it to you, it often doesn't make sense until you actually do it yourself. Cause you'd actually have to go through the struggles of actually doing a commit and pressing all the buttons, so I don't really know how much could be done in that regard.''}

% Students also asserted that the University of Victoria needs to further emphasize teaching version control systems such as GitHub at the undergraduate level. As it stands, there is one required course that teaches version control systems and how to use them, utilizing Subversion and touching on Git. Some students, however, felt that one course was not enough, particularly when Subversion is not very popular anymore: \textit{``I think in [SENG265], we did SVN, which is a good introduction to the idea. But I don't think it's widely used anymore.''} [SE3]

% Three of the interviewees [SE1, SE6, and CS3] believed that students should get an account quickly after their first introductory Computer Science courses. \textit{``If I was teaching someone how to code, as soon as they start working on code that was bigger than 100 lines, I would teach them how to use version control.''} [SE1]

This is an issue of tool literacy and an instructor who was experienced in using GitHub might have been able to better educate their students on GitHub and the features they intended to use. Beyond the instructor, many students mentioned that this issue could have been alleviated by a greater focus on version control, DVCSes and what students can do with these tools earlier in a curriculum. As it stands, students were not able to properly utilize some of the benefits of using GitHub due to inexperience and unfamiliarity. %This was discussed by [SE1, SE2, SE3, SE4, SE5, SE6, SE9, SE11, SE12, SE13, CS1, CS3, CS5]. \\

\textbf{Challenge: Notification Overload} \\
Although few students brought up this issue, how GitHub handles notifications from the repository nevertheless emerged as a challenge. The only way to get notifications from a repository is to `watch' the repository. `Watching' provides two different options: 1) to get a notification and an email only when the user is mentioned in issues or commits, and in discussions the user has commented on; or 2) to get a notification and an email when anything at all happens in the main branch (master), when someone comments on issues, commits, or pull requests, and when someone makes or accepts a pull request. %The `Watch' feature comes with some drawbacks, not the least of which was how a student's lack of familiarity with the feature prevented them from using it: \textit{``I didn't like that [repository] at all, because I didn't get notified when [the instructor] adds stuff to there, so I don't really know what's going on without remembering to check it on GitHub. ''} [SE9] This student did hear about the `watch' solution, but thought that it \textit{``would be a good solution, but it might be overkill. For like a spelling change.''} [SE9]

Unless students were `watching' the repository, they would not receive email notifications for any activities unless they were directly mentioned. However, if the students did `watch' the repository, they would receive an influx of notifications for every user comment on the discussions, which can become overwhelming. SE10 shared that they were engaged less in the activities of others because of the noise from notifications: \textit{``It sent me a million emails, both of [the tools] actually. I should have just turned that off, but I was worried about missing something. Because every time someone would post, you would get another email \ldots I actually did not read anyone else's feedback because it was just so many emails, to be totally honest.''} [SE10]

% As such, the `Watch' feature was problematic for courses like the ones studied, where every single comment would trigger a notification and an email, causing an overload of notifications. [SE7, SE9, SE10, SE11] shared this issue.

\subsection{RQ3: What are student recommendations for instructors wishing to use GitHub in a course?}

Given that the use of GitHub in these courses was relatively basic, many students, particularly those who were experienced with using GitHub for collaboration purposes, had ideas on how GitHub could be further utilized to be more beneficial for both themselves and for their instructors. Many students discussed recommendations such as which classes GitHub could best serve and the need to utilize additional GitHub features. This section outlines those responses, highlighting the suggestions students gave about the workflow for using GitHub in a course. \\

% \begin{table}[h]
%     \vspace{1pt}
%         \caption{Summary of student recommendations for instructors who wish to use GitHub for their courses.}\label{table:interviews:students:recommendations}
%     \vspace{1pt}
%     \begin{center}
%         \begin{tabular}{ | m{3cm} | m{12cm} | }
%             \hline
%             \emph{Recommendations} & \emph{Description} \\
%             \hline
%             Use GitHub in More Open-Ended Courses & Students recommended that GitHub should be used in courses were students are given more freedom for their projects and assignments rather than single-solution work. \\
%             \hline
%             Mandate the Use of GitHub's Unique Features & Students suggested that instructors use features such as pull requests and commit history extensively to take advantage of GitHub's benefits. \\
%             \hline
%             Define and Advertise a Workflow & Students believed the instructor should define a workflow for using GitHub and advertise the workflow to successfully use the tool in a course. \\
%             \hline
%         \end{tabular}
%     \end{center}
% \end{table}

\textbf{Recommendation: Use GitHub in More Open-Ended Courses} \\
As discussed earlier, students had concerns regarding the public nature of the work they host on GitHub. While most students interviewed did not mind their work and their comments being in the public space, there were concerns regarding how this way of working could apply to different types of courses, particularly courses in which students are afforded less freedom in the nature of their work. As such, some students suggested that for courses where the discussions are self-contained, the repository does not need to be public. This would avoid some challenges, such as students submitting incomplete or messy work, but would also conflict with some of the benefits extracted from these interviews, where students can build their online presence with public work and instructors can make their courses open for outsiders to contribute to.
%[SE1, SE5, SE6]

%As well, some did not even see sense in keeping a course repository open for the public because of issues highlighted earlier, such as that it just won't be interesting for outsiders and that their work might often be rushed and therefore unappealing to have on display.

%A suggestion for avoiding these issues came from one interviewee: \textit{``I think that as long as we have the option to make [our discussion comments] private, maybe after the course ends. So keep it intact while the course is ongoing and then we have the option [to change the privacy], everything will be okay.''} [SE12] Currently, GitHub does not support doing such tasks, unless the course instructor decides to privatize the course repository as a whole after a course concludes or an individual student deletes their comments and posts. %While this is a potential solution, these tools like GitHub would need work to meet such a solution. %this might be CSCS-worthy

However, students had opinions regarding what type of course would best suit GitHub. Interviewees suggested that a course similar to the two cases studied, where the work is very open-ended and could therefore exist in a public space, is where using a tool like GitHub would benefit students the most. When asked about their experiences with viewing others' projects in this course versus in other courses, SE7 said: \textit{``I would say this class is specifically different because we had so much flexibility over what we were doing. It's not like in our Operating Systems class, [where] we make a shell that does this, this, and this. Where this was way more open ended, everyone's doing something different, so even if you could see what everyone else is doing, no one could've helped us.''} [SE7] Other students acknowledged that the use of GitHub may not work in less-open ended courses because of plagiarism concerns.

% Students acknowledged that the open-ended nature of these courses was what enabled the successful use of GitHub, but that it would not work in less open-ended courses because of plagiarism concerns. Regarding the potential use of GitHub in their future courses, SE5 discussed: \textit{``Like I said, I like seeing other people's work and whatnot. Maybe not if everyone has the same assignment, because everyone's just gonna cheat off each other.''}

% These students related back to the privacy issue, where having completely public work might be a detriment to the work being done when there are concerns of plagiarism, such as when the assignments posted have a single solution. Some students, such as SE6, attempted to conceptualize a way to use private repositories, but ultimately felt it might be too cumbersome. As such, students believed that instructors would have to consider the nature of the work before deciding on the workflow they will use GitHub with, or indeed, whether they want to use GitHub at all. This consideration was suggested by [SE2, SE5, SE6, SE7, SE13].

It should be noted that there are ways to use GitHub privately within a course, where even student assignments are private. This type of workflow involves the instructor creating an organization and having each student create private repositories for their work to keep them private from each other. However, this workflow would then minimize many of the benefits listed in RQ1.

\textbf{Recommendation: Mandate the Use of GitHub's Collaborative Features} \\
The students who were more experienced with GitHub mentioned that GitHub's more collaborative features should have been further utilized to take advantage of the uniqueness of GitHub over traditional LMSes. One issue that some students discussed was that they saw little reason to use GitHub for courses if it was used only for material dissemination. For example, as mentioned in RQ1 above, only three pull requests were made throughout the semester.

DS3 was very outspoken on why using GitHub for this course was somewhat unnecessary: \textit{``I don't see any benefit that GitHub has offered that we wouldn't have had in CourseSpaces. All it appears to me is it's a place where it's a file repo \ldots and we already have that.''} They also noted that while there's potential, the unidirectional nature of the work being done meant that the potential benefits were not realized. \textit{``If there was a way to collaborate on the material, that would be useful \ldots But in this class, every one of our labs so far has been demo to the lab TA, so nothing's going back to GitHub \ldots Maybe if we were submitting things to it, maybe that would be helpful. I can see how it could be useful, it's just that in our usage it's not really adding anything to the experience.''} [DS3] %It should be noted that this student's project group did not use GitHub to collaborate, but they used Docker Hub\footnote{\url{https://hub.docker.com/account/signup/}} instead.

% SE7 echoed these sentiments: \textit{``I think that you can accomplish the same thing with a simple HTML website, honestly \ldots It's not using a lot of the features of Git, like looking at changes, commits, pull requests. The issues were kinda cool for the lab, and, again, you can accomplish that with any sort of forum, I would think \ldots We're not actually delivering code to the professor, so maybe it doesn't make a ton of sense [to be using GitHub].''}

% As such, many students believed that GitHub was not being used to its full potential in their courses. The underlying suggestion was to consider which features of GitHub the instructor would like to use, such as pull requests or grading via commits, and use those features thoroughly and consistently. As it stands, some of the benefits they described to using such a system were only possibilities. An example, which will be highlighted later in Section 5.8, was reported from a student in the CS course, where even the issues were not used for discussion during labs: \textit{``So basically we had to show it to our TA that we have done [the lab], and [the TA] used to mark it in a piece of paper. So putting [our responses in the issues] was not really necessary?''} [CS2]

Consequently, many students suggested that it would have been important to define a workflow for using this tool in a course in order to gain the benefits described earlier in the chapter. This workflow could include aforementioned activities such as utilizing pull requests or using just one tool instead of two. In the case of pull requests, for example, students advocated that the instructor should be advertising their use, thereby defining to the students that contributing to the material would be part of the course workflow: \textit{``I think [the idea is] good, but I think it would've needed to have been advertised more that [the instructor] was looking for input on things, and if [the instructor] said that, maybe more people would have [contributed] to maybe propose extensions for assignments or something.''} [SE7]

An important lesson to learn is that GitHub only equips instructors and students with the possibility to take advantage of the benefits on offer. It is then up to the instructor to realize those benefits by using the features involved. As such, the underlying suggestion was to consider which features of GitHub the instructor would like to use, such as pull requests or grading via commits, and use those features thoroughly and consistently. \\ %[SE3, SE5, SE6, SE7, SE11, CS2, CS3, CS4, CS6] touched on this suggestion. \\

% \textbf{Recommendation: Define and Advertise a Workflow} \\
% Students acknowledged that GitHub was not being used to its full potential and that there was confusion surrounding the use of two tools (GitHub and CourseSpaces). CourseSpaces was used to fill some of the gaps in education support offered by GitHub, such as private forums and a gradebook. However, students tended to be displeased with this decision. \textit{``One thing I really don't like is that we have both systems set up, and so sometimes the announcements are in GitHub, and some of the times, they're in CourseSpaces, and that can get kind of confusing, like did [the instructor] post an assignment here or there?''} [SE2]

% This was an almost unanimous issue between the students interviewed, with only a few stating that they did not mind either way. Most mentioned that they would have preferred the use of just one tool, even if everything was public in GitHub. As a result, many students suggested that it would have been important to define a workflow for using this tool in a course in order to gain the benefits described earlier in the chapter. This workflow could include aforementioned activities such as utilizing pull requests or using just one tool instead of two. In the case of pull requests, for example, students advocated that the instructor should be advertising their use, thereby defining to the students that contributing to the material would be part of the course workflow: \textit{``I think [the idea is] good, but I think it would've needed to have been advertised more that [the instructor] was looking for input on things, and if [the instructor] said that, maybe more people would have [contributed] to maybe propose extensions for assignments or something.''} [SE7]

% One student mentioned that although GitHub does not do everything needed in a course, defining a workflow will cover up many of those weaknesses: \textit{``Even if there are no enhancements on GitHub, but if you define a proper workflow for using it, then it can be quite successful, because even the present Learning Management Systems are not perfect right?''} [CS2]

% While most students did not have suggestions as to what workflow to use, they acknowledged the importance of defining it and teaching it to the students early on in the course. SE6 wanted to \textit{``enforce more actual Git and GitHub features in the way that we interact with the course material, and enforce GitHub use for actual projects. In a way that everybody had sort of a base level of understanding. So maybe at the beginning of the course \ldots there should definitely be a time when you learn Git.''}

% In summary, many of the students interviewed were frustrated by the lack of a clearly defined workflow, and believed that the course would have been improved greatly if a workflow had been created and advertised in the beginning. This recommendation emerged from interviews with [SE1, SE2, SE3, SE4, SE5, SE7, SE9, SE11, SE12, SE13, CS2, CS3, CS4, CS5].

\subsection{Validation Survey}
%In both courses, students indicated that their level of familiarity with GitHub had improved from when the course began. 30 students agreed that they would continue using GitHub for group work and for individual work after the course concluded. Given that 14 of these students were completely or somewhat unfamiliar with GitHub before the course began, students seemed to believe that using GitHub can be beneficial for them in some way.

A validation survey was sent to all students at the last laboratory session of each course in order to confirm or reject our findings. 33 students responded to a series of likert-scale style questions from each finding. 30 students agreed that they would continue using GitHub for group work and for individual work after the course concluded. Given that 14 of these students were completely or somewhat unfamiliar with GitHub before the course began, students seemed to believe that using GitHub can be beneficial for them outside of courses.

11 of the 15 SE respondents agreed with feeling more involved in the class from viewing and commenting on other projects compared to 8 of the 18 DS students who agreed. As well, while no students in SE felt that there was not enough collaboration in the courses to justify the use of GitHub, 7 of the DS students felt that way, which highlights a potential difference between how GitHub was used by the lab instructors or between the students of the two courses. %most students in SE (9) felt that there was enough collaboration or student contribution to justify using GitHub in the course, whereas only half of the participants in CS felt that GitHub use was justified.

Given the interview responses where privacy was a concern to many students, it was surprising that most students in both courses disagreed or were neutral with the suggestion that their school work should not be publicly available. 10 DS students disliked using the \emph{issues} as a discussion system on GitHub over forums with threaded discussions, while only 5 SE students disliked using GitHub `issues' for discussion. 10 students from each course disagreed that the classes needed a tutorial in the beginning of the semester, and both strongly agreed that Git, GitHub, and other DVCSes should play a bigger role in their curriculum.
