%!TEX root = icse_seet16.tex
\section{Conclusion}

%Limitations

%Future Work

%Contributions
In this work, we conducted a case study to investigate student perceptions of using GitHub in an educational context. From our findings, we uncovered ways in which GitHub use was beneficial for the students in the project-based courses we investigated as well as what the students struggled with. Importantly, however, we extracted some ways in which GitHub has the potential to support their learning even further than how the tool was used in these two courses. Beyond the experience to a popular industry tool, GitHub simplifies the process in which students can participate in a number of activities such as reviewing each other's work, contributing helpful comments or work to each other, and making changes or suggestions to course materials.

\subsection{Recommendations for Educators}
This section provides recommendations for educators who want to use GitHub to support their courses. These recommendations are based on the findings from the two phases of this work, as well as from the review of literature surrounding tools in computer science and software engineering education.

Before proceeding, we note that GitHub has their own set of recommendations for setting up an organization for a class\footnote{\url{https://education.github.com/guide}}. Their classroom guide is useful for those looking for a step-by-step process, where they recommend applying for an organization for a course and assigning a private repository for each assignment for each student. Likewise, it can also be helpful to use the available resources: use GitHub support, look for other instructor experiences for guidance, or discuss experiences in a blog or in spaces dedicated to the topic\footnote{\url{https://github.com/education/teachers/issues}}. Contributing to these resources can serve towards building a common knowledge base for instructors to share to and learn from. Moreover, GitHub recently released a tool that automates many of the tasks educators need to set up on GitHub \footnote{\url{https://classroom.github.com}}. \\

% \textbf{Recommendation: Utilize GitHub's Features} \\
Computer science and software engineering students benefit from early exposure to Git and GitHub. By utilizing these (or similar) tools in their courses, educators provide students a way to familiarize themselves and practice with these tools, which can benefit their careers. Beyond exposure, hosting assignments, projects, and code on student accounts could be valuable when seeking employment, as companies continue to investigate the online presence prospective employees have (e.g., their GitHub accounts) for hiring purposes.

While simply using GitHub as a system for material dissemination can be helpful, using more of GitHub's features, such as pull requests and issues, provides even more benefits for the students. For example, allowing students to contribute to the course and to each other's work can help develop skills such as teamwork and communication \cite{hamer2006some}. For example, educators can use GitHub's transparency features to provide feedback to students in unique ways, such as tracing the history of student projects and assignments hosted on GitHub, detailing where students made mistakes and intervening when a student seems to be struggling. Moreover, in group projects, instructors can note how much work each student has contributed, and can use this transparency for assigning grades.

%As another example, exposure to GitHub's Issues feature, even for basic discussions, was helpful for one of the students interviewed during the second phase as the student learned how the feature works for use in future projects.

%One important lesson noted from the case study was to communicate the workflow the instructor decides clearly and properly to the teaching team and to the students. When deciding to use a feature like pull requests on course material, for example, the instructor must advertise this workflow properly, perhaps even offering bonus points for added material. To communicate a workflow to students and introduce GitHub and its features to novices, instructors should consider creating a guide or hosting a tutorial session. \\

%notifications

% \textbf{Recommendation: Use Free Private Repositories for Single Solution Assignments} \\
%type of course - better for open-ended
%assignment submission
%something on Project BasedLearning here?
Many students believed that GitHub worked best when a course has open-ended projects and assignments. This stems from plaigarism concerns that exist when students are putting their code up online where others can potentially see their solutions. Of course, students can submit their assignments in private repositories that only the instructors can view and contribute to. However, single-solution assignments being hosted in private repositories limit one of the most important benefits of using a system like GitHub---the ability to view, comment on, and contribute to the work of other students. As such, although GitHub can be used in any type of course, its benefits are maximized in courses with open-ended projects and courses where student contributions and participations.

%; if the instructor creates a private repository for each student to submit their assignments and adds only the student as a collaborator, plaigarism would only be as much of a concern as it would be without using GitHub. Otherwise, an instructor could ask students to create a private repository for their assignments that only the instructors can view and contribute to.

% This style of repository management (where a private repository is dedicated to each student) could work for assignment submission as well. The instructor could ask the students to create a branch, or ask the students to fork off the main repositories and make the forks private, and then mandate that the student must make a pull request before a deadline. Thanks to GitHub's transparency features, an instructor can continuously observe the work in each student's repository and can provide further assistance to students based on the work history.

% However, the set up for this more private style of repository management requires some time and assistance from GitHub. An educator can create an organization for the course, which is granted an amount of private repositories depending on how much the instructor pays. While GitHub has stated that they would give teachers a free organization for their courses\footnote{\url{https://github.com/blog/1775-github-goes-to-school}}, an organization must be set up well before the course begins in order to get the private repositories in time.

 % As such, although GitHub is usable and helpful in any type of course, courses with open-ended projects and courses with a culture of participation are where instructors and students will see the primary benefits of using GitHub as a learning tool. If an instructor chooses to pursue the open-ended style of work similar to the courses in this study, it is recommended that they list projects and assignments on the home page using the readme markdown file so students can easily access the other projects.

% That said, GitHub continues to offer its benefits when used to submit single solution assignments. It involves some preparation to get free private repositories for students, but at the same time, it allows instructors to provide better feedback through versioning, and it maintains the benefits for students of learning Git and GitHub and hosting their work for future portfolio use (if allowed to publicize their work after the course concludes). \\

% \textbf{Recommendation: Encourage Contributions from the Students} \\
%contributions from others (slides in html, comment on other projects issues)
Another recommendation is to encourage contribution from the students in the ways that GitHub affords them. First, students can contribute to the course materials by making corrections, changes, and adding resources. Second, students can contribute to other students' work and projects (provided the work is open-ended), bringing in an element of peer review that students may benefit from \cite{sondergaard2012collaborative}. And third, students can contribute to projects outside the course by making changes and pull requests in open-source repositories. Encouraging this `Contributing Student Pedagogy' can help students develop skills such as critical analysis and collaboration \cite{falkner2012supporting}.

% Moreover, all student contributions are available for the course instructor to see. As an example, an instructor can grade students based on their contributions, such as when they create an issue or a pull request on another project. However, one issue with student contributions that must be noted is that contributing to the course materials could present difficulties depending on the file types used, as binary files such as PDF documents and PowerPoint slides are not compatible with the GitHub web interface. Although GitHub has recently provided support for viewing PDF files on the platform\footnote{\url{https://github.com/blog/1974-pdf-viewing}}, these files remain unsupported by GitHub's `diff' feature, which means that changes to the file are difficult to discern and changes to the file by multiple people will always result in a `merge conflict'. For this reason, I recommend hosting class material and slides in either markdown or HTML, file types that GitHub supports and can be easily altered using its Web platform.

\subsection{Future Work}
An important consideration from this work relates to the future of tools for computer science and software engineering education---what's next? First, we consider the importance of participation, group work, and group learning for students in technical fields in order to develop non-technical `soft' skills such as communication and teamwork \cite{jazayeri2004education}. This work demonstrates how using GitHub can unlock activities where students can contribute to each other's learning, and as a result, we believe it can be beneficial to add support for GitHub's open, collaborative workflow to current and future tools focused on learning.

%GitHub for Education
% The fact that GitHub easily supports participatory activities has multiple implications. Literature has shown that LMSes have been adding `Web 2.0' features such as blogs and wikis to their feature set \cite{downes2005feature}---students are being offered more opportunities to participate by discussion or by contributing content in blogs or wikis. Where the GitHub Way excels in education, however, is in the opportunities for students to contribute to and change the materials, and to contribute to each other's learning by getting involved in and providing feedback to projects other than their own. This is potentially the next step for Learning Management Systems, where students are more easily able to make these contributions to the work of others. The concern, however, is that implementing features similar to GitHub in an LMS might seem forced and haphazardly planned, and tool builders would be better served building a tool that supports and encourages an open, collaborative workflow from the outset.

As such, one possible path is the `GitHub for Education' Greg Wilson discussed\footnote{\url{http://software-carpentry.org/blog/2011/12/fork-merge-and-share.html}}, where a tool like GitHub can be altered or built to be more focused towards education. The main weakness of GitHub when used in this context is in the lack of flexibility in its privacy and in the lack of administrative functions such as gradebooks and announcements. Meanwhile, there are open-source alternatives to GitHub such as GitLab\footnote{\url{https://about.gitlab.com}}, that could be further developed into a tool that fulfils more educational needs. As an example, it could be valuable to implement a form of announcements, a notification feature that students have more control over, and a way to make some discussions or issues within a repository private while others remain public. This is potentially an avenue for future work, where such a tool can be evaluated.

% As well, because of the exploratory nature of the work, we sought to obtain teacher and student perspectives regarding just the viability of GitHub as a tool for education. However, other studies have investigated using tools such as wikis \cite{minocha2007collaborative} and how they possibly affect or correlate with student performance. This is one natural extension of this work: running a field experiment to see whether or not using the tool simply engages the students more or if it can ultimately affect grades.
