\section{Conclusion}

%Limitations

%Future Work

%Contributions
Another important consideration from this work relates to the future of tools for computer science and software engineering education---what's next? First, we consider the importance of participation, group work, and group learning for students in technical fields in order to develop non-technical `soft' skills such as communication and teamwork \cite{jazayeri2004education}. This work demonstrates how using GitHub can unlock activities where students can contribute to each other's learning, and as a result, I believe it can be beneficial to add support for GitHub's open, collaborative workflow to current and future tools focused on learning.

%GitHub for Education
% The fact that GitHub easily supports participatory activities has multiple implications. Literature has shown that LMSes have been adding `Web 2.0' features such as blogs and wikis to their feature set \cite{downes2005feature}---students are being offered more opportunities to participate by discussion or by contributing content in blogs or wikis. Where the GitHub Way excels in education, however, is in the opportunities for students to contribute to and change the materials, and to contribute to each other's learning by getting involved in and providing feedback to projects other than their own. This is potentially the next step for Learning Management Systems, where students are more easily able to make these contributions to the work of others. The concern, however, is that implementing features similar to GitHub in an LMS might seem forced and haphazardly planned, and tool builders would be better served building a tool that supports and encourages an open, collaborative workflow from the outset.

As such, one possible path is the `GitHub for Education' Greg Wilson discussed\footnote{\url{http://software-carpentry.org/blog/2011/12/fork-merge-and-share.html}}, where a tool like GitHub can be altered or built to be more focused towards education. The main weakness of GitHub when used in this context is in the lack of flexibility in its privacy and in the lack of administrative functions such as gradebooks and announcements. Meanwhile, there are open-source alternatives to GitHub such as GitLab\footnote{\url{https://about.gitlab.com}}, that could be further developed into a tool that fulfils more educational needs. As an example, it could be valuable to implement a form of announcements, a notification feature that students have more control over, and a way to make some discussions or issues within a repository private while others remain public. This is potentially an avenue for future work, where such a tool can be evaluated.

% As well, because of the exploratory nature of the work, we sought to obtain teacher and student perspectives regarding just the viability of GitHub as a tool for education. However, other studies have investigated using tools such as wikis \cite{minocha2007collaborative} and how they possibly affect or correlate with student performance. This is one natural extension of this work: running a field experiment to see whether or not using the tool simply engages the students more or if it can ultimately affect grades.

In summary, this work has shown the viability of using GitHub for education, and has demonstrated why the open, collaborative workflow associated with GitHub should be considered when deciding which tools to use to support a course. Based on the findings of this work, we included a set of recommendations for educators interested in using GitHub as a learning tool, and list the implications on tools that could provide the same benefits as GitHub while mitigating the limitations.
