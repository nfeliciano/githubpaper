\section{Methodology}
Our study used exploratory research methods to gain insights on a relatively new phenomenon \cite{easterbrook2008selecting}. We strove to explore the student perspective to gain insights on how the use of GitHub can benefit their learning, as well as how GitHub compares to the traditional learning tools students are exposed to and what GitHub provides beyond those tools. The research questions addressed during this phase include:

\textbf{RQ1: What are computer science and software engineering student perceptions on the benefits of using GitHub for their courses?} We've seen evidence that GitHub can benefit educators in a number of ways. The natural progression was to explore student perceptions on how this tool and this way of working might also benefit them.

\textbf{RQ2: Will students face challenges related to the use of GitHub in their courses? If so, what are these challenges?} When adopting a new tool for a course, particularly a tool not tailored towards education, there may be some friction involved due to a lack of educationally-focused features. We aimed to identify these challenges so as to make recommendations towards designing a system more suitable for educational purposes.

\textbf{RQ3: What are student recommendations for instructors wishing to use GitHub in a course?} Just as there are multiple ways to use GitHub for development purposes, an educator has multiple options regarding how they can utilize GitHub as a tool for their courses. We aimed to learn what students consider to be important for instructors when creating a GitHub workflow that students would deem appropriate for their courses.

% \bigskip
% \textbf{RQ4: From the student perspective, how does GitHub compare to traditional Learning Management Systems?} Specifically, we aimed to discover student perceptions on currently used Learning Management Systems (LMS) like CourseSpaces (Moodle) and Connex, specifically pertaining to GitHub's potential as such a portal for student interactions in their courses.

\subsection{Research Design}
We used a qualitative approach to study the student perspective of GitHub use in education. As Creswell \cite{creswell2013research} suggests, a qualitative and exploratory approach best suits research when a concept or phenomenon requires more understanding because there is little pre-existing research. Yin \cite{yin2013case} introduces case studies as \textit{``an empirical inquiry that investigates a contemporary phenomenon within its real-life context, especially when the boundaries between phenomenon and context are not clearly evident.''} Case study design, according to Yin, should be used when (a) the study seeks to answer `how' or `why' things happen; (b) the study is focused on the natural behavior of participants; (c) the context is important for the study; or (d) there are no clear descriptions of what is happening between the phenomenon and context. Because these conditions apply to the nature of the research questions asked in this study, I chose the case study design for this work. Specifically, the study was exploratory, serving as an early investigation on the phenomenon of GitHub in the classroom and to potentially build new theories or derive new hypotheses \cite{easterbrook2008selecting}.

Specific to software engineering, Runeson \cite{runeson2012case} defines case studies as \textit{``an empirical enquiry that draws on multiple sources of evidence to investigate one instance (or a small number of instances) of a contemporary software engineering phenomenon within its real-life context, especially when the boundary between phenomenon and context cannot be clearly specified.''} In this work, we aimed to draw from multiple sources of evidence---multiple students and instructors---to investigate the potential of using GitHub for post-secondary computer science and software engineering courses. It's important to learn student perspectives in this context and to explore the suitability of GitHub for supporting education.

\subsubsection{Recruitment}
%The participants in this phase included the main stakeholders in technical courses: the professor (the main instructor), lab instructors, and students. As the learning tools used in courses directly involve and impact these stakeholders, I wished to obtain their perspectives and opinions to answer our research questions. The aim was to explore their perspectives while the course was ongoing so that they could recall recent experiences and provide their opinions on GitHub as a learning tool.

For this study, we were opportunistic in finding cases and sought instructors who could and were willing to try using GitHub. We were able to recruit a professor who wanted to try using the tool in two different courses. Having multiple cases would allow us to explore some possible differences between the two scenarios \cite{yin2013case}. The two courses were a computer science (CS) course aimed at both undergraduate and graduate students, and a software engineering course (SE) that was only taken by undergraduate students. Both courses were similar in size (30-40 students) and in learning activities (weekly labs and two course projects). Student participants were recruited via a sign-up sheet during the first lectures of the course which gave us permission to contact them. Participation was voluntary and students who signed up were not required to participate in the study at all in any given phase, meaning those who were interviewed did not necessarily respond to the validation survey.

% When the term began, we attended one of the first lectures to describe my goals and to recruit students to participate in the study---interested students signed up by providing their names and email addresses, giving us permission to contact them to participate in various phases of the study. However, participation was voluntary and students who signed up were not required to participate in the study at all in any given phase. This method of recruitment meant that different students participated in the different phases. For example, those who were interviewed did not necessarily respond to the validation survey.

%preliminary survey stuff removed
\subsubsection{Research Methods}
Students who signed up during the recruitment process were emailed to be invited to interview. Most interviews with the students were one-on-one; however, due to scheduling reasons, some students requested to be interviewed as a group of 2 or 3. Interviews with the students lasted 20-30 minutes and were all conducted face-to-face in a meeting room. Audio from every interview was recorded with participant consent and notes were taken for reference. The interviews were semi-structured based on 12 guiding questions and we probed further with additional questions as deemed appropriate. %This supported the exploratory nature of the work and allowed for the discovery of interesting insights.

We also interviewed the course instructors: the main course instructor for both courses was interviewed in the middle of the semester and again after the semester concluded, and the lab instructors (one for each course) were interviewed towards the end of the course in order to find out how they utilized GitHub in their labs and to uncover their opinions on the tool's effectiveness towards the learning activities they engaged in with the students. Interviews with the instructors had a similar format to those with the students: semi-structured, 20 minutes long, with approximately 7 guiding questions.

Finally, we conducted a survey to validate the findings and confirm or contradict the themes that emerged from analysis. The survey was distributed during the final lab session of each course, where students were asked to anonymously fill in an online survey about their experiences. Respondents did not neccesarily participate in the interview phase. 18 students responded from the CS course (4 of which were interviewed), while the SE survey received 15 responses (9 of which were interviewed).

\subsubsection{Interview Participants}
We conducted interviews with 13 students from SE, 1 of which was in both courses, alongside 6 others from CS. These interviews began about 7 weeks after the semester began and concluded at the end of the semester. The main distinction between the two courses was that SE was an undergraduate Software Engineering (SENG) course whereas CS was a Computer Science course with a mix of undergraduate and graduate students. Otherwise, the courses were laid out in a similar manner (as outlined below in Section 5.5). Table \ref{table:interviews:students} summarizes the students who participated in interviews.

\begin{table}[h]
    \vspace{1pt}
        \caption{Participants and their prior experience with GitHub.}\label{table:interviews:students}
    \vspace{1pt}
    \begin{center}
        \begin{tabular}{c | c | c}
            \hline
            ID & Prior GitHub Experience & Degree Type \\
            \hline
            CS1 & Inexperienced & Graduate \\ \hline
            CS2 & Used Academically, Professionally & Graduate \\ \hline
            CS3 & Used Academically, Professionally & Graduate \\ \hline
            CS4 & Inexperienced & Graduate \\ \hline
            CS5 & Used Academically & Graduate \\ \hline
            CS6 & Used Academically & Graduate \\ \hline
            SE1 & Used Academically, Professionally & Undergraduate \\ \hline
            SE2 & Inexperienced & Undergraduate \\ \hline
            SE3 & Used Professionally & Undergraduate \\ \hline
            SE4 & Inexperienced & Undergraduate \\ \hline
            SE5 & Used Personally & Undergraduate \\ \hline
            SE6 & Used Academically & Undergraduate \\ \hline
            SE7 & Used Professionally & Undergraduate \\ \hline
            SE8 & Inexperienced & Undergraduate \\ \hline
            SE9 & Used Professionally & Undergraduate \\ \hline
            SE10 & Used Casually & Undergraduate \\ \hline
            SE11 & Used Professionally & Undergraduate \\ \hline
            SE12 & Used Academically & Undergraduate \\ \hline
            SE13 & Used Academically & Undergraduate \\ \hline
        \end{tabular}
    \end{center}
\end{table}

\subsubsection{Data Analysis}
Following the interviews, we carefully transcribed every interview, then read and re-read the content for familiarity, noting important sections or responses. Next, we coded the data by labeling various segments based on the research questions of the study. Afterwards, we identified themes and concepts that surfaced multiple times. After separating the themes into well-defined categories, we compiled a final list of themes.
