\section{GitHub Use}
This section describes the courses used in the case study and details GitHub and how it was used as a platform to support learning and teaching.

\subsection{Courses}
The two courses were a computer science (CS) course aimed at both undergraduate and graduate students, and a software engineering course (SE) that was only taken by undergraduate students. Both classes were similar in size (30-40 students) and in learning activities (weekly labs and two course projects). The course projects involved programming individually or collaboratively working with others (in groups of 2-4 students) to produce a project relating to the course topics - in one case, projects relating to the use of existing software systems to create new ones and in another, projects relating to building systems that involve multiple computational devices.

Projects were open-ended with regard to what the students created, what topic they address, and what technologies and languages they utilize. Student work was required to be publicly available so that other people, both inside and outside of the course, could view their projects. The overwhelming majority of students opted to use GitHub to host their projects. The instructor provided no formal introduction of GitHub to the students, so students unfamiliar to the tool had to learn from others or teach themselves. %The course instructor for both courses informed them that course materials were to be hosted on GitHub, and during the first laboratory session, students had to create GitHub accounts if they did not already have one.

\subsection{How GitHub was Used}
Despite being relatively unfamiliar with GitHub and its features, the course instructor opted to utilize GitHub in the same way for both courses, using its features in three pivotal ways: material dissemination through the course repository, lab work through the `Issues' feature, and project hosting through various repositories. The advanced use cases other instructors described in \cite{zagalsky2015emergence}, such as utilizing pull requests and assignment submissions, were not used for these courses. The main course instructor was aware of some of these features but was not comfortable using GitHub beyond their knowledge.

% \begin{figure}[h!]
%  \caption{The front page of the SE Course Repository---the course schedule}
%  \centering
%    \includegraphics[width=0.8\textwidth]{schedule}
%  \label{fig:schedule}
% \end{figure}

%todo add figure here?
The main use of GitHub was for material dissemination: the instructor hosted a public repository which all students could access to find the work they had to do for any given week. The instructor would update this repository weekly, adding lab assignments, links to readings, and the student homework for the week. All of the content was organized into a calendar table made with Markdown, and it was visible on the home page of the course repository as a `readme' file. Students could `fork' the repository to request changes to be made if they wished, though this possibility was not advertised to the students explicitly.

The other main use case was the use of the repository's `issues' page, where all labs (2-3 hour long sessions once a week in addition to the course lectures) were hosted. These labs would often involve researching a topic and reporting results, or giving other groups feedback on their projects. A dedicated issue would be created for each lab, similar to a forum post, and students would then make comments on these issues based on their lab work. Students were free to work in groups, and when commenting on an issue, would `@mention' their group members to indicate who the respondents were.

GitHub was also used for students to host their individual or group projects. Although students were not mandated to use GitHub for their course projects, most projects were hosted on GitHub in individual repositories. These repositories were public so others in the course could view the work and give feedback.

In addition to GitHub, the course instructor opted to use a version of the Moodle LMS\footnote{\url{https://moodle.org/}}. CourseSpaces generally allows instructors to make their course content available for students to access and interact with, to enable communication between the instructors and students through forums, to post quizzes, create wikis for a class to edit, and to track student progress and performance. For these courses, CourseSpaces was used for work that the course instructor felt should not be publicly available including student grades and student responses to the course readings.
