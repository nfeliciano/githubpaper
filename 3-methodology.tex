%!TEX root = icse_seet16.tex
\section{Methodology}
%CASE STUDY
Having previously explored the use of GitHub by computer science and software engineering educators \cite{zagalsky2015emergence}, we turned to investigate student perspectives on the suitability of GitHub for supporting their education. We focused on how students feel GitHub can benefit their learning and the challenges they might meet in using such a tool as part of their education. To gain insights, we used exploratory research methods \cite{easterbrook2008selecting} and conducted a software engineering case study \cite{yin2013case,runeson2012case} where we drew from multiple sources of evidence---interviews and a survey---to investigate the potential of using GitHub for post-secondary computer science and software engineering courses. The research questions addressed in this work include:

% We used a qualitative approach to study the student perspective of GitHub use in education. As Creswell \cite{creswell2013research} suggests, a qualitative and exploratory approach best suits research when a concept or phenomenon requires more understanding because there is little pre-existing research. Yin \cite{yin2013case} introduces case studies as \textit{``an empirical inquiry that investigates a contemporary phenomenon within its real-life context, especially when the boundaries between phenomenon and context are not clearly evident.''} Case study design, according to Yin, should be used when the study is focused on the natural behavior of participants and when the context is important for the study. Because these conditions apply to the nature of the research questions asked in this study, we chose the case study design for this work. Specifically, the study was exploratory, serving as an early investigation on the student perspective of using GitHub in the classroom and to potentially build new theories or derive new hypotheses \cite{easterbrook2008selecting}.
%
% Specific to software engineering, Runeson \cite{runeson2012case} defines case studies as \textit{``an empirical enquiry that draws on multiple sources of evidence to investigate one instance (or a small number of instances) of a contemporary software engineering phenomenon within its real-life context, especially when the boundary between phenomenon and context cannot be clearly specified.''} In this work, we aimed to draw from multiple sources of evidence---students and instructors, interviews and a survey---to investigate the potential of using GitHub for post-secondary computer science and software engineering courses. It's important to learn student perspectives in this context and to explore the suitability of GitHub for supporting education.

\textbf{RQ1: What are student perceptions on the benefits of using GitHub for their courses?} We've seen evidence that GitHub can benefit educators in a number of ways \cite{zagalsky2015emergence}. As such, we wanted to explore student perceptions about how this tool and its workflow might also benefit them.

\textbf{RQ2: What are the challenges students face related to the use of GitHub in their courses?} When adopting a new tool for a course, particularly a tool not tailored towards education, there may be some friction involved. We aimed to identify these challenges so as to make recommendations towards designing a system more suitable for educational purposes.

\textbf{RQ3: What are student recommendations for instructors wishing to use GitHub in a course?} Just as there are a variety of ways to use GitHub for development purposes, an educator has multiple options for utilizing GitHub in their courses. In order to extract future recommendations for instructors, we aimed to learn what students believed as important points for educators to consider when creating an appropriate GitHub course workflow.
\todo[inline]{NF: we might need to discuss this RQ a bit, from Alexey's concerns}

\subsection{The Case Study}
For this study, we were opportunistic in finding cases and sought instructors who could and were willing to try using GitHub. We were fortunate to recruit a university professor who wanted to try using the tool in two different software engineering courses offered in the same semester: a Distributed Systems (DS) course aimed at both undergraduate and graduate students, and a Software Evolution (SE) course for undergraduate students. The DS course was about topics surrounding distributed systems and covered concepts such as design considerations, fault tolerance, and cloud computing. The SE course covered the development of large-scale systems and how software evolves due to the many individuals that have a part in developing it over its lifetime.

Both classes were similar in size (30-40 students) and in learning activities (weekly labs and two course projects). The course projects involved individual programming or collaboratively working with others (in groups of 2-4 students) to produce a project relating to the course topic: in the DS course, building systems that involve multiple computational devices, and in the SE course, evolving existing systems or utilizing them to create new ones.

Projects were open-ended and students could choose what they created, what topics they addressed, and what technologies and languages they utilized. Students were required to make their work publicly available so that other people, both inside and outside the course, could view their projects. The overwhelming majority of students opted to use GitHub to host their projects. The instructor did not formally introduce GitHub to the students, so those unfamiliar with the tool had to learn from others or teach themselves. %The course instructor for both courses informed them that course materials were to be hosted on GitHub, and during the first laboratory session, students had to create GitHub accounts if they did not already have one.

\subsection{Research Methods}
We recruited student participants using a sign-up sheet during the first week of the course, which provided us with the necessary ethical consent to contact them. Participation was voluntary and students who signed up were not required to participate in the study in any given phase---those who were interviewed did not necessarily respond to the validation survey.

The first phase of the study consisted of interviews with the students. Most interviews were one-on-one, however, due to scheduling reasons, some students requested to be interviewed in groups of two or three. Interviews lasted 20-30 minutes and were conducted face to face in a meeting room. Audio from every interview was recorded with participant consent and notes were taken for reference. The interviews were semi-structured based on 12 guiding questions\footnote{to-add} and we probed further with additional questions as deemed appropriate. This supported the exploratory nature of the work and allowed for the discovery of interesting insights.

\todo[inline]{I don't have this in findings yet, because I'm not entirely sure the instructor stuff adds enough to the paper}
We also interviewed the instructors---the professor and the lab instructors (one for each course)---towards the end of the semester in order to find out how they utilized GitHub in their labs and to uncover their opinions on the tool's effectiveness towards the learning activities they engaged in with their students. These interviews followed a similar format to those with the students: semi-structured, 20 minutes long, with approximately 7 guiding questions.

We then conducted a survey to validate our findings and confirm or contradict the themes that emerged from our analysis of the  interview data. The survey was distributed during the final lab session of each course and students were asked to anonymously complete an online survey about their experiences. Survey respondents did not necessarily participate in the interview phase. We received 18 student responses from the DS course (4 of which were interviewed), 15 responses  from the SE course (9 of which were interviewed).

\subsection{Data Collection}
We conducted interviews with 13 students from the SE course, 1 of which was in both courses, along with 6 from the DS course. To give students sufficient experience with GitHub, these interviews began about 7 weeks after the start of the semester and concluded at the end of the semester. The main distinction between the two courses was that SE was an undergraduate course whereas DS had a mix of undergraduate and graduate students. Otherwise, the courses were laid out in a similar manner (as outlined in Section 5.5). Table \ref{table:interviews:students} summarizes the students that we interviewed.

\begin{table}[h]
    \vspace{1pt}
        \caption{Participants and their prior experience with GitHub.}\label{table:interviews:students}
    \vspace{1pt}
    \begin{center}
        \begin{tabular}{c | c | c}
            \hline
            ID & Prior GitHub Experience & Degree Type \\
            \hline
            DS1 & Inexperienced & Graduate \\ \hline
            DS2 & Used Academically, Professionally & Graduate \\ \hline
            DS3 & Used Academically, Professionally & Graduate \\ \hline
            DS4 & Inexperienced & Graduate \\ \hline
            DS5 & Used Academically & Graduate \\ \hline
            DS6 & Used Academically & Graduate \\ \hline
            SE1 & Used Academically, Professionally & Undergraduate \\ \hline
            SE2 & Inexperienced & Undergraduate \\ \hline
            SE3 & Used Professionally & Undergraduate \\ \hline
            SE4 & Inexperienced & Undergraduate \\ \hline
            SE5 & Used Personally & Undergraduate \\ \hline
            SE6 & Used Academically & Undergraduate \\ \hline
            SE7 & Used Professionally & Undergraduate \\ \hline
            SE8 & Inexperienced & Undergraduate \\ \hline
            SE9 & Used Professionally & Undergraduate \\ \hline
            SE10 & Used Casually & Undergraduate \\ \hline
            SE11 & Used Professionally & Undergraduate \\ \hline
            SE12 & Used Academically & Undergraduate \\ \hline
            SE13 & Used Academically & Undergraduate \\ \hline
        \end{tabular}
    \end{center}
\end{table}

\subsection{Data Analysis}
We transcribed every interview and repeatedly read the content for familiarity, noting important sections or responses. Next, the data was coded by a researcher by labeling various segments based on the research questions of the study. Afterwards, we identified themes and concepts that surfaced multiple times. After separating the themes into well-defined categories, we compiled a final list of themes.
\todo[inline]{NF: Add citations here, longer}
