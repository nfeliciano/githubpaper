%!TEX root = icse_seet16.tex
\section{Methodology}
%CASE STUDY
Since we previously studied the use of GitHub by educators from various disciplines \cite{zagalsky2015emergence}, we turned to investigate student perspectives on the suitability of GitHub for supporting their education. We focused on how students feel their learning benefits from GitHub and the challenges they meet in using such a tool as part of their education. To gain insights on our research questions, we conducted a case study \cite{yin2013case,runeson2012case} where we drew from multiple sources of evidence---interviews and a survey---to investigate the potential of using GitHub for post-secondary computer science and software engineering courses. The research questions addressed in this work include:

% We used a qualitative approach to study the student perspective of GitHub use in education. As Creswell \cite{creswell2013research} suggests, a qualitative and exploratory approach best suits research when a concept or phenomenon requires more understanding because there is little pre-existing research. Yin \cite{yin2013case} introduces case studies as \textit{``an empirical inquiry that investigates a contemporary phenomenon within its real-life context, especially when the boundaries between phenomenon and context are not clearly evident.''} Case study design, according to Yin, should be used when the study is focused on the natural behavior of participants and when the context is important for the study. Because these conditions apply to the nature of the research questions asked in this study, we chose the case study design for this work. Specifically, the study was exploratory, serving as an early investigation on the student perspective of using GitHub in the classroom and to potentially build new theories or derive new hypotheses \cite{easterbrook2008selecting}.
%
% Specific to software engineering, Runeson \cite{runeson2012case} defines case studies as \textit{``an empirical enquiry that draws on multiple sources of evidence to investigate one instance (or a small number of instances) of a contemporary software engineering phenomenon within its real-life context, especially when the boundary between phenomenon and context cannot be clearly specified.''} In this work, we aimed to draw from multiple sources of evidence---students and instructors, interviews and a survey---to investigate the potential of using GitHub for post-secondary computer science and software engineering courses. It's important to learn student perspectives in this context and to explore the suitability of GitHub for supporting education.

\textbf{RQ1: How do students benefit from using GitHub in their courses?} We've seen evidence that GitHub can benefit educators in a number of ways \cite{zagalsky2015emergence} and that they believe GitHub helps their students. We wished to validate educators' impressions, but also reveal students' perceptions of how GitHub and its workflow may benefit them.

\textbf{RQ2: What challenges do students face when GitHub is used by software engineering course instructors?} When adopting a new tool for a course, particularly a tool not tailored towards education, users may experience friction or a variety of other challenges. We aimed to identify these issues in order to alert educators using GitHub in their courses.

\textbf{RQ3: What recommendations can we provide software engineering instructors who wish to use GitHub for teaching?} There are a variety of ways to use GitHub for development purposes---educators also have many options for using GitHub in their courses. Based on insights obtained from students as well as relevant literature, we provide recommendations to future software engineering instructors wanting to use GitHub to support their courses.

\subsection{The Case Study}
For this study, we opportunistically sought instructors who could and were willing to try using GitHub. We were fortunate to recruit a university instructor who wanted to try using the tool in two different software engineering courses offered in the same semester: a Distributed Systems (DS) course aimed at both undergraduate and graduate students, and a Software Evolution (SE) course for senior undergraduate students. The DS course covered topics surrounding distributed systems and included concepts such as design considerations, fault tolerance, and cloud computing. The SE course covered the development of large-scale systems and how software evolves due to the many individuals that have a part in developing it over its lifetime.

Both classes were similar in size (29 in SE and 34 in DS) and learning activities (weekly labs and two course projects). The course projects involved both individual programming and collaboratively working with others (in groups of 2-4 students) to produce a project relating to the course topic: in the DS course, students built systems that involved multiple computational devices; in the SE course, students evolved existing systems or utilized them to create new ones.

Projects were open-ended and students could choose what they created, what topics they addressed, and what technologies and languages they utilized. Students were required to make their work publicly available so that other people, both inside and outside the course, could view their projects. The overwhelming majority of students opted to use GitHub to host their projects. The instructor did not formally introduce GitHub to the students, so those unfamiliar with the tool had to learn from others or teach themselves.
% Peggy: why is this commented out? NF please check.
% Noel: I didn't think this was necessary to add (an unimportant detail)
 %The course instructor for both courses informed them that course materials were to be hosted on GitHub, and during the first laboratory session, students had to create GitHub accounts if they did not already have one.

\subsection{How GitHub Was Used During the Case Study}
Despite being relatively unfamiliar with GitHub and its features, the course instructor opted to utilize GitHub in the same way for both courses, using its features in three pivotal ways: material dissemination through the course repository, lab work through the `Issues' feature, and project hosting through student repositories. The advanced use cases other instructors described in our previous study \cite{zagalsky2015emergence}, such as utilizing pull requests for assignment submissions, were not used for these courses. The main course instructor was aware of some of these features but was not comfortable using GitHub beyond their knowledge of the tool.

The main use of GitHub was for material dissemination: the instructor hosted a public repository which all students could access to find the work they had to do for any given week. The instructor updated this repository weekly, adding lab assignments, links to readings, and the student homework for the week. All of the content was organized into a calendar-style table made with Markdown and was posted on the home page of the course repository. If students wanted to make changes to the content, they could `fork' the repository and use a pull request to alert the instructors, although this possibility was not emphasized during the courses.
%\todo[inline]{CP: is the fact that it wasn't advertise actually significant? if not, I'd remove that bit. also, did the students even make use of it? I thought no one did, but I can't remember anymore.}
% NF: In this case, yes. Some students used it but others didn't, and a few said that she should have advertised it a little more

The other main use of GitHub was for hosting lab content and related discussions. The courses contained labs---a two- to three-hour session once a week---in addition to the regular lectures, which often involved researching a topic and reporting results, or giving other groups feedback on their projects. Using each course repository's `Issues' page, a dedicated issue was created for each lab, similar to a forum post, and students made comments on the appropriate issue based on their lab work. Students were free to work in groups, and when commenting on an issue, would \emph{`@mention'} their group members to indicate who they were interacting with.

GitHub was also used for students to host their individual or group project work. Although students were not mandated to use GitHub for their projects, most work was hosted on GitHub in individual repositories. These repositories were publicly available so others in the course could view the work and give feedback.

%\footnote{\url{https://moodle.org/}}
In addition to GitHub, the course instructor opted to use a version of the Moodle LMS. Moodle generally allows instructors to make their course content available for students to access and interact with, enable communication between the instructors and students through forums, post quizzes, create wikis for a class to edit, and track student progress and performance. Moodle is a closed environment that is invite-only and, unlike GitHub, has more sophisticated security settings where artifacts can be kept private from individual participants, not just the public. For the courses in our study, Moodle was used for artifacts that the course instructor felt should not be publicly available, including student grades and student responses to the course readings.
%\todo[inline]{CP: should you talk a little more about how Moodle allows one to control privacy/access better?}

\subsection{Research Methods}
We recruited student participants using a sign-up sheet during the first week of each course. Participation was voluntary and students who signed up were not required to participate in all phases of the study---those who were interviewed did not necessarily respond to the subsequent validation survey (discussed below).

The first phase of the study consisted of interviews with the students. Most interviews were one-on-one, however, due to scheduling reasons, some students requested to be interviewed in groups of two or three. Interviews lasted 20-30 minutes and were conducted face to face in a meeting room. Audio from each interview was recorded with participant consent and notes were taken for reference. The interviews were semi-structured based on 12 guiding questions\footnote{\url{https://goo.gl/oszl17}} and we probed further with additional questions (as needed) to gain their insights. We also interviewed the course instructor at the end of the semester to further understand how GitHub supported their teaching. This fit the exploratory nature of our research questions and helped us discover interesting insights.
%TODO: footnote

%\todo[inline]{I don't have this in findings yet, because I'm not entirely sure the instructor stuff adds enough to the paper, %Peggy: since it is not a question I think it is perhaps ok to leave it out but it may be nice to have some of this in discussion to add to the previous study}
%We also interviewed the instructors---the professor and the lab instructors (one for each course)---towards the end of the semester in order to find out how they utilized GitHub in their labs and to uncover their opinions on the tool's effectiveness towards the learning activities they engaged in with their students. These interviews followed a similar format to those with the students: semi-structured, 20 minutes long, with approximately 7 guiding questions. These interviews provided additional context for understanding the responses we received from the students.

In a second phase, we conducted a survey with the students to validate our findings and to confirm or contradict the themes that emerged from our analysis of the interview data in phase one.

\subsection{Data Collection and Analysis}
We conducted interviews with 12 students from the SE course, 6 students from the DS course, and one student who was taking both courses, for a total of 19 interviews.
To give students sufficient experience with GitHub, these interviews occurred no earlier than 7 weeks after the start of the semester and all were concluded within 5 weeks.

 The main distinction between the two courses was that SE was an undergraduate course whereas DS had a mix of undergraduate and graduate students. Otherwise, the courses were laid out in a similar manner (as outlined in Section 3.2). Table \ref{table:interviews:students} summarizes the previous experience the interviewed students had with GitHub.

\begin{table}[h]
    \vspace{-10pt}
        \caption{Participants and their prior experience with GitHub.}\label{table:interviews:students}
    \vspace{-10pt}
    \begin{center}
        \begin{tabular}{c | c | c}
            \hline
            ID & Prior GitHub Experience & Degree Type \\
            \hline
            DS1 & Inexperienced & Graduate \\ \hline
            DS2 & Used Academically, Professionally & Graduate \\ \hline
            DS3 & Used Academically, Professionally & Graduate \\ \hline
            DS4 & Inexperienced & Graduate \\ \hline
            DS5 & Used Academically & Graduate \\ \hline
            DS6 & Used Academically & Graduate \\ \hline
            SE1 & Used Academically, Professionally & Undergraduate \\ \hline
            SE2 & Inexperienced & Undergraduate \\ \hline
            SE3 & Used Professionally & Undergraduate \\ \hline
            SE4 & Inexperienced & Undergraduate \\ \hline
            SE5 & Used Personally & Undergraduate \\ \hline
            SE6 & Used Academically & Undergraduate \\ \hline
            SE7 & Used Professionally & Undergraduate \\ \hline
            SE8 & Inexperienced & Undergraduate \\ \hline
            SE9 & Used Professionally & Undergraduate \\ \hline
            SE10 & Used Casually & Undergraduate \\ \hline
            SE11 & Used Professionally & Undergraduate \\ \hline
            SE12 & Used Academically & Undergraduate \\ \hline
            SE13 & Used Academically & Undergraduate \\ \hline
        \end{tabular}
    \end{center}
    \vspace{-12pt}
\end{table}

We transcribed every interview and coded the responses using a content analysis methodology~\cite{charmaz2006constructing}.
We labeled segments according to the research questions of the study, and from these codes, we iteratively identified themes and concepts that surfaced multiple times. After grouping the themes into well-defined categories, we compiled a final list of themes. To check for biases, the coding of the interviews was reviewed by another researcher. To reduce possible biases, as themes emerged, we also searched for and reported counter examples to the findings (some of these counter examples are discussed in a thesis due to space constraints in this paper~\cite{feliciano2015towards}). Furthermore, the themes that emerged from the interviews with students were validated through the survey and the interviews with the instructor.
%\todo[inline]{NF: Add citations here, longer, add a reference to your thesis, Peggy:  I addressed this a bit add, the charmaz to the references NF and check it is correct, also add a reference to your thesis}
% Noel: Done, keeping comment in case I missed something

The validation survey in the second phase of our research was distributed during the final lab session of each course and students were asked to anonymously fill out an online form with details about their experiences. As mentioned above, survey respondents did not necessarily participate in the interview phase. We received 18 student responses from the DS course (4 of which were interviewed) and 15 responses from the SE course (9 of which were interviewed), for a total of 33 responses.

%opportunistic
%group interviews
\subsection{Limitations}
There are a number of limitations with our research design, which we describe here.
%Internal validity concerns biases that might affect the results of a study \cite{creswell2013research}.

Our decision to recruit an instructor that had not used GitHub before (either for other work or for teaching) may have influenced (negatively) the experience of the students we wished to study. We made this decision as our previous research on this topic had gathered perspectives from instructors that generally knew GitHub quite well.

The recruitment methods we used to engage students may have resulted in biased opinions, as the students that were willing to be interviewed or surveyed may have been students who felt strongly about GitHub (positively or negatively), whereas those without a strong opinion may have chosen not to participate. By interviewing many students, however, we were able to uncover contradictions or discrepancies between the opinions we gathered. We also tried to offset this limitation by including a validation survey in our design, which had a reasonable response rate of 53\%.

We also recognize that the questions we asked in the interviews and in the surveys may have been biased. In an attempt to limit this as much as possible, we piloted the questions and had members outside our team review them for biases.

When we coded the interviews, only a single coder was involved, but we had an expert reviewer question the themes that emerged and review the coding for potential biases and inconsistencies. We further validated the emergent themes by interviewing the instructor, and by hosting a validation survey. Finally, we reported discrepant information---in the interviews and between the interview and survey data---to illustrate that not all students shared the same opinions.

Finally, we recognize that the findings from this single case study cannot be generalized to other settings~\cite{runeson2012case} where GitHub may be used in software engineering courses. We tried to offset this by asking the instructor to use the tool in two different courses. One particular issue that we alluded to above was the instructor's lack of experience with GitHub. This may have influenced our results, but in a negative way, as the instructor was not able to give more guidance to the students using the tool. However, many of the findings we report are reflected in other studies that use similar tools for classes, such as Kelleher's study on Git and GitHub \cite{kelleher2014employing}, and Haaranen and Lehtinen's study on Git and GitLab \cite{haaranen2015teaching}.
