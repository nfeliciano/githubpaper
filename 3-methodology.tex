%!TEX root = icse_seet16.tex
\section{Methodology}
%CASE STUDY
Since we previously studied the use of GitHub by computer science and software engineering educators \cite{zagalsky2015emergence}, we turned to investigate student perspectives on the suitability of GitHub for supporting their education. We focused on how students feel GitHub can benefit their learning and the challenges they may meet in using such a tool as part of their education. To gain insights on our research questions, we conducted a case study \cite{yin2013case,runeson2012case} where we drew from multiple sources of evidence---interviews and a survey---to investigate the potential of using GitHub for post-secondary computer science and software engineering courses. The research questions addressed in this work include:

% We used a qualitative approach to study the student perspective of GitHub use in education. As Creswell \cite{creswell2013research} suggests, a qualitative and exploratory approach best suits research when a concept or phenomenon requires more understanding because there is little pre-existing research. Yin \cite{yin2013case} introduces case studies as \textit{``an empirical inquiry that investigates a contemporary phenomenon within its real-life context, especially when the boundaries between phenomenon and context are not clearly evident.''} Case study design, according to Yin, should be used when the study is focused on the natural behavior of participants and when the context is important for the study. Because these conditions apply to the nature of the research questions asked in this study, we chose the case study design for this work. Specifically, the study was exploratory, serving as an early investigation on the student perspective of using GitHub in the classroom and to potentially build new theories or derive new hypotheses \cite{easterbrook2008selecting}.
%
% Specific to software engineering, Runeson \cite{runeson2012case} defines case studies as \textit{``an empirical enquiry that draws on multiple sources of evidence to investigate one instance (or a small number of instances) of a contemporary software engineering phenomenon within its real-life context, especially when the boundary between phenomenon and context cannot be clearly specified.''} In this work, we aimed to draw from multiple sources of evidence---students and instructors, interviews and a survey---to investigate the potential of using GitHub for post-secondary computer science and software engineering courses. It's important to learn student perspectives in this context and to explore the suitability of GitHub for supporting education.

\textbf{RQ1: How do students benefit from using GitHub in their courses?} We've seen evidence that GitHub can benefit educators in a number of ways \cite{zagalsky2015emergence} and their impression that GitHub is helpful for their students. We wished to validate these educator impressions but also reveal student perceptions about how GitHub and its workflow may benefit them.

\textbf{RQ2: What challenges do students face when GitHub is used by software engineering course instructors?} When adopting a new tool for a course, particularly a tool not tailored towards education, there may be some friction involved or other challenges. We aimed to identify these challenges so that educators using GitHub may be aware of them and that the designers of GitHub may also know these issues.

\textbf{RQ3: What recommendations can we give to software engineering instructors who wish to use GitHub for teaching?} Just as there are a variety of ways to use GitHub for development purposes, an educator also has many options for using GitHub in their courses. To extract recommendations for instructors on how to use GitHub in their courses, we asked students what educators should consider and how to use an appropriate GitHub workflow in their courses.
\todo[inline]{NF: this paragraph needs to change}

\subsection{The Case Study}
For this study, we were opportunistic in finding cases and sought instructors who could and were willing to try using GitHub. We were fortunate to recruit a university professor who wanted to try using the tool in two different software engineering courses offered in the same semester: a Distributed Systems (DS) course aimed at both undergraduate and graduate students, and a Software Evolution (SE) course for senior undergraduate students. The DS course was about topics surrounding distributed systems and covered concepts such as design considerations, fault tolerance, and cloud computing. The SE course covered the development of large-scale systems and how software evolves due to the many individuals that have a part in developing it over its lifetime.

Both classes were similar in size (30-40 students) and in learning activities (weekly labs and two course projects). The course projects involved individual programming or collaboratively working with others (in groups of 2-4 students) to produce a project relating to the course topic: in the DS course, building systems that involve multiple computational devices, and in the SE course, evolving existing systems or utilizing them to create new ones.

Projects were open-ended and students could choose what they created, what topics they addressed, and what technologies and languages they utilized. Students were required to make their work publicly available so that other people, both inside and outside the course, could view their projects. The overwhelming majority of students opted to use GitHub to host their projects. The instructor did not formally introduce GitHub to the students, so those unfamiliar with the tool had to learn from others or teach themselves.
% Peggy: why is this commented out? NF please check.
% Noel: I didn't think this was necessary to add (an unimportant detail)
 %The course instructor for both courses informed them that course materials were to be hosted on GitHub, and during the first laboratory session, students had to create GitHub accounts if they did not already have one.

\subsection{How GitHub was Used During the Case Study}
Despite being relatively unfamiliar with GitHub and its features, the course instructor opted to utilize GitHub in the same way for both courses, using its features in three pivotal ways: material dissemination through the course repository, lab work through the `Issues' feature, and project hosting through various repositories. The advanced use cases other instructors described in \cite{zagalsky2015emergence}, such as utilizing pull requests and assignment submissions, were not used for these courses. The main course instructor was aware of some of these features but was not comfortable using GitHub beyond their knowledge.

The main use of GitHub was for material dissemination: the instructor hosted a public repository which all students could access to find the work they had to do for any given week. The instructor would update this repository weekly, adding lab assignments, links to readings, and the student homework for the week. All of the content was organized into a calendar table made with Markdown, and it was visible on the home page of the course repository as a `readme' file. Students could `fork' the repository to request changes to be made if they wished, though this possibility was not advertised to the students explicitly.

The other main use case was the use of the repository's `issues' page, where all labs (2-3 hour long sessions once a week in addition to the course lectures) were hosted. These labs would often involve researching a topic and reporting results, or giving other groups feedback on their projects. A dedicated issue would be created for each lab, similar to a forum post, and students would then make comments on these issues based on their lab work. Students were free to work in groups, and when commenting on an issue, would `@mention' their group members to indicate who the respondents were.

GitHub was also used for students to host their individual or group projects. Although students were not mandated to use GitHub for their course projects, most projects were hosted on GitHub in individual repositories. These repositories were public so others in the course could view the work and give feedback.

In addition to GitHub, the course instructor opted to use a version of the Moodle LMS\footnote{\url{https://moodle.org/}}. CourseSpaces generally allows instructors to make their course content available for students to access and interact with, to enable communication between the instructors and students through forums, to post quizzes, create wikis for a class to edit, and to track student progress and performance. For these courses, CourseSpaces was used for work that the course instructor felt should not be publicly available including student grades and student responses to the course readings.

\subsection{Research Methods}
We recruited student participants from both courses using a sign-up sheet during the first week of the course, which provided us with the necessary ethical consent to contact them. Participation was voluntary and students who signed up were not required to participate in the study in any given phase---those who were interviewed did not necessarily respond to the validation survey (discussed below).

The first phase of the study consisted of interviews with the students. Most interviews were one-on-one, however, due to scheduling reasons, some students requested to be interviewed in groups of two or three. Interviews lasted 20-30 minutes and were conducted face to face in a meeting room. Audio from every interview was recorded with participant consent and notes were taken for reference. The interviews were semi-structured based on 12 guiding questions\footnote{to-add} and we probed further with additional questions as needed to gain their insights. This supported the exploratory nature of our research questions and supported the discovery of interesting insights.

\todo[inline]{I don't have this in findings yet, because I'm not entirely sure the instructor stuff adds enough to the paper, Peggy: since it is not a question I think it is perhaps ok to leave it out but it may be nice to have some of this in discussion to add to the previous study}
We also interviewed the instructors---the professor and the lab instructors (one for each course)---towards the end of the semester in order to find out how they utilized GitHub in their labs and to uncover their opinions on the tool's effectiveness towards the learning activities they engaged in with their students. These interviews followed a similar format to those with the students: semi-structured, 20 minutes long, with approximately 7 guiding questions.  These interviews gave us additional context to understand responses received from the students.

In a second phase, we conducted a survey with the students to validate our findings and to confirm or contradict the themes that emerged from our analysis of the  interview data. The survey was distributed during the final lab session of each course and students were asked to anonymously complete an online survey about their experiences. As mentioned above, survey respondents did not necessarily participate in the interview phase. We received 18 student responses from the DS course (4 of which were interviewed), 15 responses  from the SE course (9 of which were interviewed), for a total of 33 responses.

\subsection{Data Collection}
We conducted interviews with 13 students from the SE course, 5 students from the DS course, and one student who was taking both courses (for a total of 19 interviews).
To give students sufficient experience with GitHub, these interviews occurred no earlier than 7 weeks after the start of the semester and all were concluded within 5 weeks.

 The main distinction between the two courses was that SE was an undergraduate course whereas DS had a mix of undergraduate and graduate students. Otherwise, the courses were laid out in a similar manner (as outlined in Section 5.5). Table \ref{table:interviews:students} summarizes the previous experience the interviewed students had with GitHub.

\begin{table}[h]
    \vspace{1pt}
        \caption{Participants and their prior experience with GitHub.}\label{table:interviews:students}
    \vspace{1pt}
    \begin{center}
        \begin{tabular}{c | c | c}
            \hline
            ID & Prior GitHub Experience & Degree Type \\
            \hline
            DS1 & Inexperienced & Graduate \\ \hline
            DS2 & Used Academically, Professionally & Graduate \\ \hline
            DS3 & Used Academically, Professionally & Graduate \\ \hline
            DS4 & Inexperienced & Graduate \\ \hline
            DS5 & Used Academically & Graduate \\ \hline
            DS6 & Used Academically & Graduate \\ \hline
            SE1 & Used Academically, Professionally & Undergraduate \\ \hline
            SE2 & Inexperienced & Undergraduate \\ \hline
            SE3 & Used Professionally & Undergraduate \\ \hline
            SE4 & Inexperienced & Undergraduate \\ \hline
            SE5 & Used Personally & Undergraduate \\ \hline
            SE6 & Used Academically & Undergraduate \\ \hline
            SE7 & Used Professionally & Undergraduate \\ \hline
            SE8 & Inexperienced & Undergraduate \\ \hline
            SE9 & Used Professionally & Undergraduate \\ \hline
            SE10 & Used Casually & Undergraduate \\ \hline
            SE11 & Used Professionally & Undergraduate \\ \hline
            SE12 & Used Academically & Undergraduate \\ \hline
            SE13 & Used Academically & Undergraduate \\ \hline
        \end{tabular}
    \end{center}
\end{table}

\subsection{Data Analysis}
We transcribed every interview and coded the responses using a content analysis methodology~\cite{charmaz2006constructing}.
% K. Charmaz, Constructing Grounded Theory: A Practical Guide Through Qualitative Analysis. Pine Forge Press, 2006.
We labeled segments according to the research questions of the study. From these codes, we iteratively identified themes and concepts that surfaced multiple times. After grouping the themes into well-defined categories, we compiled a final list of themes. The coding of the interviews was reviewed by another researcher to check for biases in coding. To reduce possible biases, as themes emerged, we also searched for and reported counter examples to the findings (some of these counter examples are discussed in a longer report due to space constraints in this paper~\cite{feliciano2015towards}. Furthermore, the themes that emerged from the interviews with students were validated through the survey in Phase 2 and through the interviews with the instructor and teaching assistants.
%\todo[inline]{NF: Add citations here, longer, add a reference to your thesis, Peggy:  I addressed this a bit add, the charmaz to the references NF and check it is correct, also add a reference to your thesis}
% Noel: Done, keeping comment in case I missed something
