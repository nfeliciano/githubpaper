%cse and seet, shift towards more real life skills gained through PBL, CSP, etc.

%git, github, vc being used in classes

%issues in learning tools?

%why do I talk about CSP? because of its potential for that need. Haaranen -> contributing to course material

%!TEX root = icse_seet16.tex
\section{Background}
Regardless of the field, the use of software tools to support learning, teaching, material dissemination, and course management is an important aspect of education. Traditionally, university educators employ the use of learning management systems (LMSes) to manage the courses they teach. LMSes, such as Blackboard, Moodle, and Sakai, give instructors a variety of features for managing courses, such as file management, grade tracking, assignment hosting, and chat \cite{kumar2011comparative}. The use of an LMS provides students and educators with a set of tools for typical classroom processes. \cite{malikowski2007model} developed a model that dissects the quality of LMS tools into five categories: (1) transmitting course content, (2) evaluating students, (3) evaluating courses and instructors, (4) creating class discussions, and (5) creating computer-based instruction. Their study shows that the most prominent use of an LMS is to transmit information to students, whereas the categories of creating class discussions and evaluating students receive moderate and low-to-moderate use, respectively.

With the rise of the `Web 2.0' and the social web, Learning Management Systems began to incorporate more social approaches. Edrees, for example, \cite{edrees2013elearning}, compares the `2.0' tools and features of Moodle and Blackboard, two of the more popular LMSes, identifying that they both added features to become more social such as wikis, blogs, RSS, podcasts, bookmarking, and virtual environments. However, despite the increase of social features in LMSes, many researchers and educators have expressed concerns regarding their readiness to incorporate student participation. McLoughlin \cite{mcloughlin2007social} believes that participatory learning lends itself well to education as students are provided with more learning opportunities where they can connect and learn from each other. However, he notes that LMSes tend to be more administration-focused, and that there were signs that Web 2.0 tools could make learning environments more personal, participatory, and collaborative. Similarly, Dalsgaard \cite{dalsgaard2006social} argues that students should be provided with a myriad of tools for independent work, reflection, construction, and collaboration, which LMSes typically provide only a minor part of.

\subsection{The Contributing Student}
In computer science and software engineering education, the focus has shifted from not just technical skills, but also the development of soft skills such as communication and teamwork \cite{jazayeri2004education}. One such way to develop these skills is to allow students to contribute to each other's learning experiences and to course materials \cite{hamer2006some}. This concept, called `Contributing Student Pedagogy' (CSP) \cite{hamer2008contributing}, is formally defined as: \textit{``A pedagogy that encourages students to contribute to the learning of others and to value the contributions of others.''} The pedagogy relies highly on the technology used to facilitate this learning experience, where the learning tools would typically support activities such as peer review, content construction, and solution sharing, amongst others.

There are various characteristics of CSP in practice: (a) the people involved (students and instructors) switch roles from passive to active, (b) there is a focus on student contribution, (c) the quality of contributions is assessed, (d) learning communities develop, and (e) student contributions are facilitated by technology. Falkner and Falkner \cite{falkner2012supporting} observe the benefits of incorporating student contributions to their curriculum such as increased engagement and participation, and the development of critical analysis, collaboration, and problem solving skills---important skills for a computer scientist.

\subsection{GitHub's Place in Education}
As such, GitHub has the capability to support some activities to support student contributions. In a case study of Git and the GitLab Web portal (a platform similar to GitHub) being utilized in a large-scale, Haaranen \& Lehtinen \cite{haaranen2015teaching} describe allowing students to contribute to the course material by making corrections via \emph{pull requests} as advantageous for enabling students to learn essential skills needed in industry. Moreover, through \emph{diffs}, issue tracking, and merge requests, GitHub provides support for code reviews \cite{Kalliamvakou}, a peer-reviewing process that promote positive attitudes towards work alongside training in critical reviewing and communication skills \cite{HundhausenAgrawalAgarwal} for students.

In documenting his process of utilizing GitHub in the classroom, Kelleher \cite{kelleher2014employing} describes the transparency of activity as a way of alerting him to possible acts of plagiarism and the integrated issue tracking as a way to annotating code. Griffin \& Seals \cite{Griffin:2013:GCJ:2458539.2458551} leverage the \emph{branch} and \emph{merge} features of Git to simplify assigment work and submission; however, they describe the downside of GitHub being a `social coding' platform, which may not suit standard programming assignments that must be kept private. Moreover, other version control tools such as Concurrent Version Control (CVS) \cite{Reid:2005:LDI:1047124.1047441} and Subversion \cite{Clifton:2007:SFS:1227504.1227344} have been used to simplify the management of courses and allow students to collaborate on work more easily.

%THE CONTRIBUTING STUDENT
% Collis and Moonen \cite{collis2001flexible} proposed a social approach to education, the `contributing student', where students contribute materials for other students to learn from. In this concept, the tool being utilized in the classroom plays an important role, as they note that the tool or site being contributed to should be largely empty before the learners and instructor fill it through course activities. In this concept, learners need to create or find learning materials and share them with others as a way to engage in their learning \cite{collis2006contributing}. By contributing to the course material with their findings and experiences, students can affect each others' learning. This means a student adopts several roles in a learning community, including being a co-creator of learning materials, being someone who extends the work of others (rather than just reading them), and being someone involved in self and peer evaluation.
%
% In the literature surrounding the idea of ``the contributing student'', researchers emphasized the importance of the tools used in a course. Without the appropriate tools, according to Collis and Moonen \cite{collis2006contributing}, this approach to student engagement may not even be feasible in practice. In the next section, we explore the literature surrounding the tools often used in education, and how they fit the aforementioned social approaches to education. \\

% In a review of tools that support CSP in computer science education \cite{hamer2011tools}, seven different CSP activities that tools can support include:
% \begin{itemize}
% \item Peer review%---students can see and analyze each other's work and provide feedback.
% \item Dialogue and discussion%---the student contributions occur in the communication between the students.
% \item Annotation%---students can comment on existing (not student-created) materials and share their comments with other students.
% \item Content construction%---students can create new learning material for other students to consume and learn from.
% \item Solution sharing%---students can share their solutions with other students.
% \item Activity creation%---students can create learning activities for other students to engage in.
% \item Making links%---students can search for external resources that relate to the content.
% \end{itemize}

% While Hamer \textit{et al.}'s literature search provided a number of tools that meet many of these characteristics, they were surprised that there weren't more examples of tools that support student-contributed learning activities. As well, they reported that many of the tools seemed to only be used within the institution where they were developed, not supporting cross-institutional use. This suggests that tools in the computer science and software engineering disciplines need improvements to further support student participation and collaboration.

% As distributed version control systems play a crucial role in many software projects, including their support for developer contribution and collaboration, researchers have attempted to see how these systems can benefit education. Reid \& Wilson \cite{reid2005learning}, introduced Concurrent Version Systems (CVS) for their classes, making it easier for students to work in groups as well as providing a history of student work. Beyond those obvious advantages, instructors and teaching assistants were also able to assist students better as they could easily retrieve an up-to-date copy of student work. Similar advantages are found when other version control tools such as when Subversion \cite{clifton2007subverting} and Git \cite{griffin2013github} are used in education, using features such as branching and merging to organize assignments and assignment submission.

% \subsection{What is GitHub?}
% % \ref{fig:githubfeats}
% GitHub is a Web-based social code sharing service released in 2008 that utilizes the Git distributed version control system. It is a tool utilized by millions of developers all over the world to facilitate collaboration via the use of its awareness and transparency components, collaborative features such as pull requests, and version control. The tool is organized so that developers can create repositories with their work that can be public, meaning that anybody can see them and pull the code into their own repositories; though the owner can decide who can and cannot make changes. Alternatively, they can be private, whereby the repository is viewable and editable only by those given permission by the owner. This provides many opportunities for remixing and reusing content, as well as supporting a workflow where multiple parties can do separate work at their own pace.

% Git is the underlying version control system that GitHub utilizes. There are two very important aspects to Git: that work is distributed, and that work is handled by version control. Being distributed refers to the possibility of work being decentralized: instead of being forced to work in a repository where there is a central hub that everyone pushes code to, individual developers can create public `clones' of that repository and `push' to their respective clones before the original repository's maintainer or owner pulls in the work.

%describe features

%Version control systems have been used in the classroom as a way of managing students and their work. Reid \& Wilson \cite{Reid:2005:LDI:1047124.1047441} introduced the Concurrent Versions System (CVS) to a second-year computer science course. This provided the instructors with a simple way to manage student assignments, made it easier for students to work in pairs or groups, and gave the instructors a history of student work. Clifton, Kaczmarczyk \& Mrozek \cite{Clifton:2007:SFS:1227504.1227344} used Subversion, another version control system, to collaboratively develop and run introductory computer science courses. The ease of managing courses using Subversion allowed the instructors to free up time from administrative demands, allowing them to spend more time focusing on pedagogical issues. In 2013, Griffin \& Seals used GitHub in the classroom as a version control tool, leveraging the \textit{Branch} and \textit{Merge} features \cite{Griffin:2013:GCJ:2458539.2458551}. When students worked on programming assignments, it was easy to \textit{merge} back into the original project if their version worked, or abandon a branch without destroying the original project.

GitHub is a Web-based social code sharing service released in 2008 that utilizes the Git distributed version control system. It is a tool utilized by millions of developers all over the world to facilitate collaboration via the use of its awareness and transparency components, collaborative features such as pull requests, and version control. The tool is organized so that developers can create repositories with their work that can be public, available for other developers to contribute to. This provides many opportunities for remixing and reusing content, as well as supporting a workflow where multiple parties can do separate work at their own pace.

GitHub provides a number of features that aid collaboration and support user contributions. Users can make changes to other people's work in separate repositories or branches, and can make a \emph{pull request} to request that the original repository owner \emph{merge} their changes into the base. Issue tracking allows contributors to discuss any aspect of a project, including bugs, feature requests, and documentation \cite{BissyandeEtc}. Moreover, GitHub's openness and transparency features, which allows users to easily see all activities inside a repository or from a user they're following, fosters both direct and indirect collaboration \cite{dabbish2012social}.

In a previous study \cite{zagalsky2015emergence}, we found a number of ways in which GitHub impacted learning and teaching. Educators we conducted interviews with spoke of benefits such as the ability to monitor student work continuously and the ease with which they can reuse and remix course materials from other instructors. We also noted benefits which impact their students, such as learning how to use a tool relevant to their field, and the ability to make changes to course materials.
