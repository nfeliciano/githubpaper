%cse and seet, shift towards more real life skills gained through PBL, CSP, etc.

%git, github, vc being used in classes

%issues in learning tools?

%why do I talk about CSP? because of its potential for that need. Haaranen -> contributing to course material

%!TEX root = icse_seet16.tex
\section{Background}
The use of software tools to support learning, teaching, material dissemination, and course management is an important aspect of education, regardless of the domain. In the following, we first provide some background on Learning Management Systems (LMS) and research on how students learn through contributions. This is followed by a description of GitHub and an overview of research that has been done on how GitHub-like tools have been used in education.

\subsection{The Evolution of Learning Management Systems}
Traditionally, university educators employ the use of LMSes to manage the courses they teach. An LMS provides students and educators with a set of tools for typical classroom processes. LMSes such as Blackboard, Moodle, and Sakai provide instructors with a variety of features for organizing and administering courses, including file management, grade tracking, assignment hosting, and discussion rooms \cite{kumar2011comparative}.

In addressing common features in LMSes, Malikowski \emph{et al.} \cite{malikowski2007model} developed a model that distinguishes LMS tool features into five categories: (1) transmitting course content, (2) evaluating students, (3) evaluating courses and instructors, (4) creating class discussions, and (5) creating computer-based instruction. Their research showed that the most prominent use of an LMS is to transmit information to students, whereas features for creating class discussions and evaluating students saw moderate to low-to-moderate usage, respectively.
% Peggy: this next sentence isn't shown from the text so I commented it out and then reworded it, but check it is what you wanted.
% This model shows that the most prominent use of an LMS is to transmit information to students,
% whereas features for creating class discussions and evaluating students receive moderate and low-to-moderate use, respectively.

With the rise of Web 2.0 and `the social Web', LMS have become more social and collaborative. For example, Edrees\cite{edrees2013elearning} compares the `2.0' tools and features of Moodle and Blackboard, two of the more popular LMSes, identifying that they both now included social features such as wikis, blogs, RSS, podcasts, bookmarking, and virtual environments. Despite the popularity of these features, many researchers and educators have expressed concerns regarding their readiness to incorporate and emphasize student participation. McLoughlin \cite{mcloughlin2007social} believes that participatory learning lends itself well to education as students are provided with more learning opportunities where they can connect and learn from each other. However, he notes that while there were signs that Web 2.0 tools could make learning environments more personal, participatory, and collaborative, LMSes tend to be more focused on administration rather than the learner. Dalsgaard \cite{dalsgaard2006social} also points out the weaknesses of LMSes for supporting learner-centered activities such as independent work, reflection, construction, and collaboration, arguing that students should be provided with a myriad of other tools to support such activities.

\subsection{The Contributing Student}
Computer science and software engineering education has started embracing a pedagogy that not only focuses on technical skills, but also on soft skills such as communication and teamwork \cite{jazayeri2004education}. One way to develop these skills is to allow students to contribute to each other's learning experiences \cite{hamer2006some}. This concept, which Hamer calls a Contributing Student Pedagogy (CSP) \cite{hamer2008contributing}, is formally defined as: \textit{``A pedagogy that encourages students to contribute to the learning of others and to value the contributions of others.''} It relies on technology to facilitate the learning experience, where the learning tools typically support activities such as peer review, content construction and solution sharing, amongst others.

There are various characteristics of CSP in practice: (a) the people involved (students and instructors) switch roles from passive to active, (b) there is a focus on student contribution, (c) the quality of contributions is assessed, (d) learning communities develop, and (e) student contributions are facilitated by technology. Falkner and Falkner \cite{falkner2012supporting} observe the benefits of incorporating student contributions into their curriculum, such as increased engagement and participation, and the development of critical analysis, collaboration, and problem solving skills---important skills for a computer scientist or engineer.

\subsection{GitHub-style Systems in Education}
GitHub has the ability to support certain CSP activities. It provides a number of features that aid collaboration and support user contributions. Users can make changes to other people's work in separate repositories or branches, and they can make a \emph{pull request} to invite the original repository owner to review and \emph{merge} their changes into the base version of the software repository. Issue tracking allows contributors to discuss any aspect of a project, including bugs, feature requests, and documentation \cite{bissyande2013got}. Moreover, GitHub's openness and transparency features, which allow users to easily see all activities inside a repository or from a user they're following, fosters both direct and indirect collaboration \cite{dabbish2012social}.

Haaranen \& Lehtinen \cite{haaranen2015teaching} conducted a case study where Git and the GitLab Web portal (an open source platform similar to GitHub) were used in a large computer science course. Students could contribute to the course material by making corrections via \emph{pull requests}. The authors discussed that learning how to do pull requests is an essential industry skill.

The educators we probed in our previous study~\cite{zagalsky2015emergence} described how using GitHub in their courses allowed the instructors to encourage student participation through its transparency features. Moreover, \emph{diffs}, issue tracking, and merge requests in GitHub provide support for code reviews \cite{kalliamvakou2014promises}, a peer-review process that promotes a positive student attitude towards work, as well as training in critical reviewing and communication skills \cite{hundhausen2013talking}.
%NFTODO: discuss more of last study

Kelleher \cite{kelleher2014employing} documented his process of using GitHub in the classroom. He described how the transparency of activities alerted him to possible acts of plagiarism, and how the integrated issue tracking could be used for annotating code. Griffin \& Seals \cite{griffin2013github} leveraged Git's \emph{branch} and \emph{merge} features to simplify assignments and submissions, however, they felt that GitHub's `social coding' platform might not suit standard programming assignments that need to remain private. These studies follow early examples of an educational use of other version control tools, such as Concurrent Version Control (CVS) \cite{reid2005learning} and Subversion \cite{clifton2007subverting}. In most of these cases, the tools were used for assignment submission, to simplify the management of courses, and to allow students to collaborate on work more easily. %\todo[inline]{NF: still have stuff to add here, kind of to justify the study}

%THE CONTRIBUTING STUDENT
% Collis and Moonen \cite{collis2001flexible} proposed a social approach to education, the `contributing student', where students contribute materials for other students to learn from. In this concept, the tool being utilized in the classroom plays an important role, as they note that the tool or site being contributed to should be largely empty before the learners and instructor fill it through course activities. In this concept, learners need to create or find learning materials and share them with others as a way to engage in their learning \cite{collis2006contributing}. By contributing to the course material with their findings and experiences, students can affect each others' learning. This means a student adopts several roles in a learning community, including being a co-creator of learning materials, being someone who extends the work of others (rather than just reading them), and being someone involved in self and peer evaluation.
%
% In the literature surrounding the idea of ``the contributing student'', researchers emphasized the importance of the tools used in a course. Without the appropriate tools, according to Collis and Moonen \cite{collis2006contributing}, this approach to student engagement may not even be feasible in practice. In the next section, we explore the literature surrounding the tools often used in education, and how they fit the aforementioned social approaches to education. \\

% In a review of tools that support CSP in computer science education \cite{hamer2011tools}, seven different CSP activities that tools can support include:
% \begin{itemize}
% \item Peer review%---students can see and analyze each other's work and provide feedback.
% \item Dialogue and discussion%---the student contributions occur in the communication between the students.
% \item Annotation%---students can comment on existing (not student-created) materials and share their comments with other students.
% \item Content construction%---students can create new learning material for other students to consume and learn from.
% \item Solution sharing%---students can share their solutions with other students.
% \item Activity creation%---students can create learning activities for other students to engage in.
% \item Making links%---students can search for external resources that relate to the content.
% \end{itemize}

% While Hamer \textit{et al.}'s literature search provided a number of tools that meet many of these characteristics, they were surprised that there weren't more examples of tools that support student-contributed learning activities. As well, they reported that many of the tools seemed to only be used within the institution where they were developed, not supporting cross-institutional use. This suggests that tools in the computer science and software engineering disciplines need improvements to further support student participation and collaboration.

% As distributed version control systems play a crucial role in many software projects, including their support for developer contribution and collaboration, researchers have attempted to see how these systems can benefit education. Reid \& Wilson \cite{reid2005learning}, introduced Concurrent Version Systems (CVS) for their classes, making it easier for students to work in groups as well as providing a history of student work. Beyond those obvious advantages, instructors and teaching assistants were also able to assist students better as they could easily retrieve an up-to-date copy of student work. Similar advantages are found when other version control tools such as when Subversion \cite{clifton2007subverting} and Git \cite{griffin2013github} are used in education, using features such as branching and merging to organize assignments and assignment submission.

% \subsection{What is GitHub?}
% % \ref{fig:githubfeats}
% GitHub is a Web-based social code sharing service released in 2008 that utilizes the Git distributed version control system. It is a tool utilized by millions of developers all over the world to facilitate collaboration via the use of its awareness and transparency components, collaborative features such as pull requests, and version control. The tool is organized so that developers can create repositories with their work that can be public, meaning that anybody can see them and pull the code into their own repositories; though the owner can decide who can and cannot make changes. Alternatively, they can be private, whereby the repository is viewable and editable only by those given permission by the owner. This provides many opportunities for remixing and reusing content, as well as supporting a workflow where multiple parties can do separate work at their own pace.

% Git is the underlying version control system that GitHub utilizes. There are two very important aspects to Git: that work is distributed, and that work is handled by version control. Being distributed refers to the possibility of work being decentralized: instead of being forced to work in a repository where there is a central hub that everyone pushes code to, individual developers can create public `clones' of that repository and `push' to their respective clones before the original repository's maintainer or owner pulls in the work.

%describe features

%Version control systems have been used in the classroom as a way of managing students and their work. Reid \& Wilson \cite{Reid:2005:LDI:1047124.1047441} introduced the Concurrent Versions System (CVS) to a second-year computer science course. This provided the instructors with a simple way to manage student assignments, made it easier for students to work in pairs or groups, and gave the instructors a history of student work. Clifton, Kaczmarczyk \& Mrozek \cite{Clifton:2007:SFS:1227504.1227344} used Subversion, another version control system, to collaboratively develop and run introductory computer science courses. The ease of managing courses using Subversion allowed the instructors to free up time from administrative demands, allowing them to spend more time focusing on pedagogical issues. In 2013, Griffin \& Seals used GitHub in the classroom as a version control tool, leveraging the \textit{Branch} and \textit{Merge} features \cite{Griffin:2013:GCJ:2458539.2458551}. When students worked on programming assignments, it was easy to \textit{merge} back into the original project if their version worked, or abandon a branch without destroying the original project.

% In a previous study \cite{zagalsky2015emergence}, we found a number of ways in which GitHub impacted learning and teaching. Educators we conducted interviews with spoke of benefits such as the ability to monitor student work continuously and the ease with which they can reuse and remix course materials from other instructors. We also noted benefits which impact their students, such as learning how to use a tool relevant to their field, and the ability to make changes to course materials.
