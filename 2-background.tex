%cse and seet, shift towards more real life skills gained through PBL, CSP, etc.

%git, github, vc being used in classes

%issues in learning tools?

%why do I talk about CSP? because of its potential for that need. Haaranen -> contributing to course material

%!TEX root = icse_seet16.tex
\section{Background}
The use of software tools to support learning, teaching, material dissemination, and course management is an important aspect of education, regardless of the domain. In the following, we first provide some background on Learning Management Systems (LMS) and research on how students learn through contributions. This is followed by a description of GitHub and an overview of research related to how GitHub-like tools have been used in education.

\subsection{The Evolution of Learning Management Systems}
Traditionally, university educators employ the use of LMSes to manage the courses they teach. An LMS provides students and educators with a set of tools for typical classroom processes. LMSes such as Blackboard, Moodle, and Sakai provide instructors with a variety of features for organizing and administering courses, including file management, grade tracking, assignment hosting, and discussion rooms \cite{kumar2011comparative}.

In addressing common features in LMSes, Malikowski \emph{et al.} \cite{malikowski2007model} developed a model that distinguishes LMS tool features into five categories: (1) transmitting course content, (2) evaluating students, (3) evaluating courses and instructors, (4) creating class discussions, and (5) creating computer-based instruction. Their research showed that the most prominent use of an LMS is to transmit information to students, whereas features for creating class discussions and evaluating students saw moderate to low-to-moderate usage, respectively.

With the rise of Web 2.0 and `the social Web', LMS have become more social and collaborative. For example, Edrees \cite{edrees2013elearning} compares the `2.0' tools and features of Moodle and Blackboard, two of the more popular LMSes, identifying that they both now included social features such as wikis, blogs, RSS, podcasts, bookmarking, and virtual environments. Despite the popularity of these features, many researchers and educators have expressed concerns regarding their readiness to incorporate and emphasize student participation. McLoughlin \cite{mcloughlin2007social} believes that participatory learning lends itself well to education as students are provided with more learning opportunities where they can connect and learn from each other. However, he notes that while there were signs that Web 2.0 tools could make learning environments more personal, participatory, and collaborative, LMSes tend to be more focused on administration rather than the learner. Dalsgaard \cite{dalsgaard2006social} also points out the weaknesses of LMSes for supporting learner-centered activities such as independent work, reflection, construction, and collaboration, arguing that students should be provided with a myriad of other tools to support such activities.

\subsection{The Contributing Student}
Computer science and software engineering education has started embracing a pedagogy that not only focuses on technical skills, but also on soft skills such as communication and teamwork \cite{jazayeri2004education}. One way to develop these skills is to allow students to contribute to each other's learning experiences \cite{hamer2006some}. This concept, which Hamer calls a Contributing Student Pedagogy (CSP) \cite{hamer2008contributing}, is formally defined as: \textit{``A pedagogy that encourages students to contribute to the learning of others and to value the contributions of others.''} It relies on technology to facilitate the learning experience, where the learning tools typically support activities such as peer review, content construction and solution sharing, amongst others.

There are various characteristics of CSP in practice: (a) the people involved (students and instructors) switch roles from passive to active, (b) there is a focus on student contribution, (c) the quality of contributions is assessed, (d) learning communities develop, and (e) student contributions are facilitated by technology. Falkner and Falkner \cite{falkner2012supporting} observe the benefits of incorporating student contributions into their curriculum, such as increased engagement and participation, and the development of critical analysis, collaboration, and problem solving skills---important skills for a computer scientist or engineer.

\subsection{GitHub-style Systems in Education}
GitHub is one such technology that shows the ability to support certain CSP activities. In particular, it provides several features that aid collaboration and support user contributions. Users can make changes to other people's work in separate repositories or branches, and they can make a \emph{pull request} to invite the original repository owner to review and \emph{merge} their changes into the base version of the software repository. Issue tracking allows contributors to discuss any aspect of a project, including bugs, feature requests, and documentation \cite{bissyande2013got}. Moreover, GitHub's openness and transparency features, which allow users to easily see all activities inside a repository or from a user they're following, fosters both direct and indirect collaboration \cite{dabbish2012social}.

Haaranen \& Lehtinen \cite{haaranen2015teaching} conducted a case study where Git and GitLab (an open source platform similar to GitHub) were used in a large computer science course. Students could contribute to the course material by making corrections via \emph{pull requests}. The authors discussed that learning how to do pull requests is an essential industry skill.

The educators we probed in our previous study~\cite{zagalsky2015emergence} described how using GitHub in their courses allowed the instructors to encourage student participation through its transparency features. Moreover, \emph{diffs}, issue tracking, and merge requests in GitHub provide support for code reviews \cite{kalliamvakou2014promises}, a peer-review process that promotes a positive student attitude towards work, as well as training in critical reviewing and communication skills \cite{hundhausen2013talking}.
%NFTODO: discuss more of last study

Kelleher \cite{kelleher2014employing} documented his process of using GitHub in the classroom. He described how the transparency of activities alerted him to possible acts of plagiarism, and how the integrated issue tracking could be used for annotating code. Griffin \& Seals \cite{griffin2013github} leveraged Git's \emph{branch} and \emph{merge} features to simplify assignments and submissions, however, they felt that GitHub's `social coding' platform might not suit standard programming assignments that need to remain private. These studies follow early examples of an educational use of other version control tools, such as Concurrent Version Control (CVS) \cite{reid2005learning} and Subversion \cite{clifton2007subverting}. In most of these cases, the tools were used for assignment submission, to simplify the management of courses, and to allow students to collaborate on work more easily.
