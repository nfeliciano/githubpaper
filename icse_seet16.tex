% This is "sig-alternate.tex" V2.0 May 2012
% This file should be compiled with V2.5 of "sig-alternate.cls" May 2012
%
% This example file demonstrates the use of the 'sig-alternate.cls'
% V2.5 LaTeX2e document class file. It is for those submitting
% articles to ACM Conference Proceedings WHO DO NOT WISH TO
% STRICTLY ADHERE TO THE SIGS (PUBS-BOARD-ENDORSED) STYLE.
% The 'sig-alternate.cls' file will produce a similar-looking,
% albeit, 'tighter' paper resulting in, invariably, fewer pages.
%
% ----------------------------------------------------------------------------------------------------------------
% This .tex file (and associated .cls V2.5) produces:
%       1) The Permission Statement
%       2) The Conference (location) Info information
%       3) The Copyright Line with ACM data
%       4) NO page numbers
%
% as against the acm_proc_article-sp.cls file which
% DOES NOT produce 1) thru' 3) above.
%
% Using 'sig-alternate.cls' you have control, however, from within
% the source .tex file, over both the CopyrightYear
% (defaulted to 200X) and the ACM Copyright Data
% (defaulted to X-XXXXX-XX-X/XX/XX).
% e.g.
% \CopyrightYear{2007} will cause 2007 to appear in the copyright line.
% \crdata{0-12345-67-8/90/12} will cause 0-12345-67-8/90/12 to appear in the copyright line.
%
% ---------------------------------------------------------------------------------------------------------------
% This .tex source is an example which *does* use
% the .bib file (from which the .bbl file % is produced).
% REMEMBER HOWEVER: After having produced the .bbl file,
% and prior to final submission, you *NEED* to 'insert'
% your .bbl file into your source .tex file so as to provide
% ONE 'self-contained' source file.
%
% ================= IF YOU HAVE QUESTIONS =======================
% Questions regarding the SIGS styles, SIGS policies and
% procedures, Conferences etc. should be sent to
% Adrienne Griscti (griscti@acm.org)
%
% Technical questions _only_ to
% Gerald Murray (murray@hq.acm.org)
% ===============================================================
%
% For tracking purposes - this is V2.0 - May 2012

\documentclass{sig-alternate}

% Load basic packages
\usepackage{balance}  % to better equalize the last page
\usepackage{graphics} % for EPS, load graphicx instead
\usepackage[pdftex, bookmarks=false, hidelinks]{hyperref}
\usepackage{booktabs}
\usepackage{enumerate}
\usepackage{todonotes}% allows to add inline comments with \todo[inline]{comment text}

% The following command makes sure text doesn't go outside of the page limit - added by Alexey
\sloppy

\pagenumbering{arabic}
\begin{document}
%
% --- Author Metadata here ---
\conferenceinfo{ICSE}{'16, May 14 -- 22, 2016, Austin, TX, USA}
%\CopyrightYear{2007} % Allows default copyright year (20XX) to be over-ridden - IF NEED BE.
%\crdata{0-12345-67-8/90/01}  % Allows default copyright data (0-89791-88-6/97/05) to be over-ridden - IF NEED BE.
% --- End of Author Metadata ---

\title{Student Experiences Using GitHub in Software Engineering Courses: A Case Study}

%
% You need the command \numberofauthors to handle the 'placement
% and alignment' of the authors beneath the title.
%
% For aesthetic reasons, we recommend 'three authors at a time'
% i.e. three 'name/affiliation blocks' be placed beneath the title.
%
% NOTE: You are NOT restricted in how many 'rows' of
% "name/affiliations" may appear. We just ask that you restrict
% the number of 'columns' to three.
%
% Because of the available 'opening page real-estate'
% we ask you to refrain from putting more than six authors
% (two rows with three columns) beneath the article title.
% More than six makes the first-page appear very cluttered indeed.
%
% Use the \alignauthor commands to handle the names
% and affiliations for an 'aesthetic maximum' of six authors.
% Add names, affiliations, addresses for
% the seventh etc. author(s) as the argument for the
% \additionalauthors command.
% These 'additional authors' will be output/set for you
% without further effort on your part as the last section in
% the body of your article BEFORE References or any Appendices.

\numberofauthors{1} %  in this sample file, there are a *total*
% of EIGHT authors. SIX appear on the 'first-page' (for formatting
% reasons) and the remaining two appear in the \additionalauthors section.
%
\author{
       \alignauthor Joseph Feliciano, Margaret-Anne Storey, Alexey Zagalsky\\
       \affaddr{University of Victoria}\\
       \affaddr{Victoria, BC, Canada}\\
       \email{\{noelf, mstorey, alexeyza\}@uvic.ca}\\
}
% Just remember to make sure that the TOTAL number of authors
% is the number that will appear on the first page PLUS the
% number that will appear in the \additionalauthors section.

\maketitle
\begin{abstract}
GitHub has been embraced by the software development community as an important social platform for managing software projects and to support collaborative development. More recently, educators have begun to adopt it for hosting course content and student assignments. From our previous research, we found that educators leverage GitHub's collaboration and transparency features to create, reuse and remix course materials, and to encourage student contributions and monitor student activity on assignments and projects. However, our previous research did not consider the student perspective.

In this paper, we present a case study where GitHub is used as a learning platform for two software engineering courses. We gathered student perspectives on how the use of GitHub in their courses might benefit them and to identify the challenges they may face. The findings from our case study indicate that software engineering students do benefit from GitHub's transparent and open workflow. However, students were concerned that since GitHub is not inherently an educational tool, it lacks key features important for education and poses learning and privacy concerns. Our findings provide recommendations for designers on how tools such as GitHub can be used to improve software engineering education, and also point to recommendations for instructors on how to use it more effectively in their courses.

% Peggy: making the abstract shorter...
%Moreover, we highlight the importance of educating both students and instructors about GitHub's features and workflow, tcreate an environment that maximizes its benefits in software engineering education. We are just at the start of this adoption curve, so it is important to understand the opportunities and risks using GitHub may introduce to the teaching of software engineering and computer science.
\end{abstract}

% A category with the (minimum) three required fields
\category{H.5.3.}{Group and Organization Interfaces}{Computer-supported cooperative work}
%A category including the fourth, optional field follows...
%\category{D.2.8}{Software Engineering}{Metrics}[complexity measures, performance measures]

\terms{Human Factors}

\keywords{GitHub, education, learning, software engineering, collaboration}

\section{Introduction}
%new best practices for software engineering education and training;

%focuses:
%- the contributing student
%- GH to facilitate that
%- student perceptions of the tool
%- best uses of GH

GitHub is a social code sharing platform popular amongst software developers. Utilizing the Git distributed version control system (DVCS), GitHub is a tool millions of people use for collaboration. Though primarily used by software developers, it has seen utilization in other areas, such as technical writing\footnote{\url{http://readwrite.com/2013/11/08/seven-ways-to-use-github-that-arent-coding}}. GitHub offers advantages with its open, collaborative workflow, where collaborators can be involved in a project in a number of ways, such as contributing to discussions regarding bugs and features, or making changes to a project itself and allowing other collaborators to review and accept their changes.

Instructors have also began to use GitHub as a tool to assist their classes, utilizing its open workflow and its transparency features to benefit their teaching. In a previous study \cite{Zagalsky}, we interviewed instructors who were early adopters of using GitHub as a learning platform and extracted their motivations and the benefits and challenges they encountered. These instructors described benefits such as the ability to reuse and remix course materials and the ease in which GitHub facilitated student participation and contributions to the course material. However, while we extracted some benefits and challenges the instructors' students experienced, gathering student perspectives on GitHub as a learning tool may reveal new insights.

%In computer science and software engineering education, the focus has shifted from not just technical skills, but also the development of soft skills such as communication and teamwork \cite{jazayeri2004education}. One such way to develop these skills is to allow students to contribute to each other's learning experiences and to course materials \cite{falkner2012supporting}.

%This concept, called `Contributing Student Pedagogy' \cite{hamer2008contributing}, relies highly on the technology used to facilitate this learning experience.

%It is also important, however, to explore the student perspective and to discern how the use of GitHub as an educational tool might affect students.

%The use of version control systems has become increasingly commonplace in software development. Distributed version control systems (DVCS) in particular are typically used in collaboration and to interact with the global software development community. As such, using DVCS has become an essential skill for software engineers. However, the integration of DVCS into computing education as a way of teaching students these skills has so far gained little attention. In this study, we discover the implications of using a distributed version control system as a course platform.

%GitHub is a social code sharing service and version control system. It is a popular tool for many groups and projects that require collaboration, and has even seen utilization in areas outside software development, such as technical writing\footnote{\url{http://readwrite.com/2013/11/08/seven-ways-to-use-github-that-arent-coding}}. The advantages of GitHub and similar tools include their awareness and transparency features, where collaborators can easily stay informed of others' work \cite{dabbish2012social}. As well, collaborators in a GitHub repository can be involved in a project in a number of ways, such as contributing to discussions regarding bugs and features, or making changes to a project itself and allowing other collaborators to review and accept their changes. This open, collaborative workflow is called `The GitHub Way'\footnote{\url{http://www.wired.com/2013/09/github-for-anything/}} as these features are not necessarily exclusive to GitHub and can be found in other Distributed Version Control Systems (DVCSes) such as BitBucket.

%particularly focusing on the collaborative and contributive activities GitHub could enable students to partake in
This work presents a multiple-case study in which GitHub was used as an e-learning tool for project-based, computer science and software engineering courses. The work is largely exploratory in order to discern how GitHub impacted the learning of students in these courses, as well as to learn the benefits and challenges students met or expect to meet with GitHub's use in this context. Conclusions from this work can help determine the viability of GitHub-like systems as educational tools, or can help shape the development of future educationally-focused tools which, similar to GitHub, gives students opportunities to contribute to the learning experience in multiple ways. To do so, we interviewed the course participants (the students and the teaching team) regarding their experiences with using GitHub in their course to gather their perspectives on whether or not the GitHub workflow can support computer science and software engineering courses.

%While the instructors and students reaped many benefits from using GitHub in these courses, there were drawbacks and challenges with using a tool not built for education.

This study aims to explore the use of a tool such as GitHub and its effectiveness in the educational context. These research questions are exploratory in nature:
\begin{enumerate}
\item \textbf{What are student perceptions on the benefits of using GitHub for their courses?}
\item \textbf{What are the challenges students face related to the use of GitHub in their courses?}
\item \textbf{What are student recommendations for instructors wishing to use GitHub in a course?}
\end{enumerate}

Our main contributions include gathering insights into the benefits and weaknesses of using GitHub for computer science and software engineering courses from the point of view of both students and the teaching team. As well, we extract suggestions for ways to use GitHub as a learning platform to maximize the benefits of its use in an educational context.

In this case study, we conducted interviews with the students and the teaching team of two courses, one computer science course and one software engineering course. Our findings indicate that students enjoyed the benefits of GitHub's transparency features and open workflow. However, students expressed concerns that GitHub is not inherently an educational tool and is therefore missing key features that prevent it from being an ideal tool for this context.

% We conducted interviews with the students and the teaching team involved in the courses selected for these cases and followed up with a survey to validate our findings.


%This allows for an approach to learning characterized by a \emph{demand-pull} model rather than a \emph{supply-push} model, and focuses on participation and providing students access to rich learning communities \cite{seely2008open}

%cse and seet, shift towards more real life skills gained through PBL, CSP, etc.

%git, github, vc being used in classes

%issues in learning tools?

%why do I talk about CSP? because of its potential for that need. Haaranen -> contributing to course material

%!TEX root = icse_seet16.tex
\section{Background}
Regardless of the field, the use of software tools to support learning, teaching, material dissemination, and course management is an important aspect of education. Traditionally, university educators employ the use of learning management systems (LMSes) to manage the courses they teach. LMSes, such as Blackboard, Moodle, and Sakai, give instructors a variety of features for managing courses, such as file management, grade tracking, assignment hosting, and chat \cite{kumar2011comparative}. The use of an LMS provides students and educators with a set of tools for typical classroom processes. \cite{malikowski2007model} developed a model that dissects the quality of LMS tools into five categories: (1) transmitting course content, (2) evaluating students, (3) evaluating courses and instructors, (4) creating class discussions, and (5) creating computer-based instruction. Their study shows that the most prominent use of an LMS is to transmit information to students, whereas the categories of creating class discussions and evaluating students receive moderate and low-to-moderate use, respectively.

With the rise of the `Web 2.0' and the social web, Learning Management Systems began to incorporate more social approaches. Edrees, for example, \cite{edrees2013elearning}, compares the `2.0' tools and features of Moodle and Blackboard, two of the more popular LMSes, identifying that they both added features to become more social such as wikis, blogs, RSS, podcasts, bookmarking, and virtual environments. However, despite the increase of social features in LMSes, many researchers and educators have expressed concerns regarding their readiness to incorporate student participation. McLoughlin \cite{mcloughlin2007social} believes that participatory learning lends itself well to education as students are provided with more learning opportunities where they can connect and learn from each other. However, he notes that LMSes tend to be more administration-focused, and that there were signs that Web 2.0 tools could make learning environments more personal, participatory, and collaborative. Similarly, Dalsgaard \cite{dalsgaard2006social} argues that students should be provided with a myriad of tools for independent work, reflection, construction, and collaboration, which LMSes typically provide only a minor part of.

\subsection{The Contributing Student}
In computer science and software engineering education, the focus has shifted from not just technical skills, but also the development of soft skills such as communication and teamwork \cite{jazayeri2004education}. One such way to develop these skills is to allow students to contribute to each other's learning experiences and to course materials \cite{hamer2006some}. This concept, called `Contributing Student Pedagogy' (CSP) \cite{hamer2008contributing}, is formally defined as: \textit{``A pedagogy that encourages students to contribute to the learning of others and to value the contributions of others.''} The pedagogy relies highly on the technology used to facilitate this learning experience, where the learning tools would typically support activities such as peer review, content construction, and solution sharing, amongst others.

There are various characteristics of CSP in practice: (a) the people involved (students and instructors) switch roles from passive to active, (b) there is a focus on student contribution, (c) the quality of contributions is assessed, (d) learning communities develop, and (e) student contributions are facilitated by technology. Falkner and Falkner \cite{falkner2012supporting} observe the benefits of incorporating student contributions to their curriculum such as increased engagement and participation, and the development of critical analysis, collaboration, and problem solving skills---important skills for a computer scientist.

\subsection{GitHub's Place in Education}
As such, GitHub has the capability to support some activities to support student contributions. In a case study of Git and the GitLab Web portal (a platform similar to GitHub) being utilized in a large-scale, Haaranen \& Lehtinen \cite{haaranen2015teaching} describe allowing students to contribute to the course material by making corrections via \emph{pull requests} as advantageous for enabling students to learn essential skills needed in industry. Moreover, through \emph{diffs}, issue tracking, and merge requests, GitHub provides support for code reviews \cite{Kalliamvakou}, a peer-reviewing process that promote positive attitudes towards work alongside training in critical reviewing and communication skills \cite{HundhausenAgrawalAgarwal} for students.

In documenting his process of utilizing GitHub in the classroom, Kelleher \cite{kelleher2014employing} describes the transparency of activity as a way of alerting him to possible acts of plagiarism and the integrated issue tracking as a way to annotating code. Griffin \& Seals \cite{Griffin:2013:GCJ:2458539.2458551} leverage the \emph{branch} and \emph{merge} features of Git to simplify assigment work and submission; however, they describe the downside of GitHub being a `social coding' platform, which may not suit standard programming assignments that must be kept private. Moreover, other version control tools such as Concurrent Version Control (CVS) \cite{Reid:2005:LDI:1047124.1047441} and Subversion \cite{Clifton:2007:SFS:1227504.1227344} have been used to simplify the management of courses and allow students to collaborate on work more easily.

%THE CONTRIBUTING STUDENT
% Collis and Moonen \cite{collis2001flexible} proposed a social approach to education, the `contributing student', where students contribute materials for other students to learn from. In this concept, the tool being utilized in the classroom plays an important role, as they note that the tool or site being contributed to should be largely empty before the learners and instructor fill it through course activities. In this concept, learners need to create or find learning materials and share them with others as a way to engage in their learning \cite{collis2006contributing}. By contributing to the course material with their findings and experiences, students can affect each others' learning. This means a student adopts several roles in a learning community, including being a co-creator of learning materials, being someone who extends the work of others (rather than just reading them), and being someone involved in self and peer evaluation.
%
% In the literature surrounding the idea of ``the contributing student'', researchers emphasized the importance of the tools used in a course. Without the appropriate tools, according to Collis and Moonen \cite{collis2006contributing}, this approach to student engagement may not even be feasible in practice. In the next section, we explore the literature surrounding the tools often used in education, and how they fit the aforementioned social approaches to education. \\

% In a review of tools that support CSP in computer science education \cite{hamer2011tools}, seven different CSP activities that tools can support include:
% \begin{itemize}
% \item Peer review%---students can see and analyze each other's work and provide feedback.
% \item Dialogue and discussion%---the student contributions occur in the communication between the students.
% \item Annotation%---students can comment on existing (not student-created) materials and share their comments with other students.
% \item Content construction%---students can create new learning material for other students to consume and learn from.
% \item Solution sharing%---students can share their solutions with other students.
% \item Activity creation%---students can create learning activities for other students to engage in.
% \item Making links%---students can search for external resources that relate to the content.
% \end{itemize}

% While Hamer \textit{et al.}'s literature search provided a number of tools that meet many of these characteristics, they were surprised that there weren't more examples of tools that support student-contributed learning activities. As well, they reported that many of the tools seemed to only be used within the institution where they were developed, not supporting cross-institutional use. This suggests that tools in the computer science and software engineering disciplines need improvements to further support student participation and collaboration.

% As distributed version control systems play a crucial role in many software projects, including their support for developer contribution and collaboration, researchers have attempted to see how these systems can benefit education. Reid \& Wilson \cite{reid2005learning}, introduced Concurrent Version Systems (CVS) for their classes, making it easier for students to work in groups as well as providing a history of student work. Beyond those obvious advantages, instructors and teaching assistants were also able to assist students better as they could easily retrieve an up-to-date copy of student work. Similar advantages are found when other version control tools such as when Subversion \cite{clifton2007subverting} and Git \cite{griffin2013github} are used in education, using features such as branching and merging to organize assignments and assignment submission.

% \subsection{What is GitHub?}
% % \ref{fig:githubfeats}
% GitHub is a Web-based social code sharing service released in 2008 that utilizes the Git distributed version control system. It is a tool utilized by millions of developers all over the world to facilitate collaboration via the use of its awareness and transparency components, collaborative features such as pull requests, and version control. The tool is organized so that developers can create repositories with their work that can be public, meaning that anybody can see them and pull the code into their own repositories; though the owner can decide who can and cannot make changes. Alternatively, they can be private, whereby the repository is viewable and editable only by those given permission by the owner. This provides many opportunities for remixing and reusing content, as well as supporting a workflow where multiple parties can do separate work at their own pace.

% Git is the underlying version control system that GitHub utilizes. There are two very important aspects to Git: that work is distributed, and that work is handled by version control. Being distributed refers to the possibility of work being decentralized: instead of being forced to work in a repository where there is a central hub that everyone pushes code to, individual developers can create public `clones' of that repository and `push' to their respective clones before the original repository's maintainer or owner pulls in the work.

%describe features

%Version control systems have been used in the classroom as a way of managing students and their work. Reid \& Wilson \cite{Reid:2005:LDI:1047124.1047441} introduced the Concurrent Versions System (CVS) to a second-year computer science course. This provided the instructors with a simple way to manage student assignments, made it easier for students to work in pairs or groups, and gave the instructors a history of student work. Clifton, Kaczmarczyk \& Mrozek \cite{Clifton:2007:SFS:1227504.1227344} used Subversion, another version control system, to collaboratively develop and run introductory computer science courses. The ease of managing courses using Subversion allowed the instructors to free up time from administrative demands, allowing them to spend more time focusing on pedagogical issues. In 2013, Griffin \& Seals used GitHub in the classroom as a version control tool, leveraging the \textit{Branch} and \textit{Merge} features \cite{Griffin:2013:GCJ:2458539.2458551}. When students worked on programming assignments, it was easy to \textit{merge} back into the original project if their version worked, or abandon a branch without destroying the original project.

GitHub is a Web-based social code sharing service released in 2008 that utilizes the Git distributed version control system. It is a tool utilized by millions of developers all over the world to facilitate collaboration via the use of its awareness and transparency components, collaborative features such as pull requests, and version control. The tool is organized so that developers can create repositories with their work that can be public, available for other developers to contribute to. This provides many opportunities for remixing and reusing content, as well as supporting a workflow where multiple parties can do separate work at their own pace.

GitHub provides a number of features that aid collaboration and support user contributions. Users can make changes to other people's work in separate repositories or branches, and can make a \emph{pull request} to request that the original repository owner \emph{merge} their changes into the base. Issue tracking allows contributors to discuss any aspect of a project, including bugs, feature requests, and documentation \cite{BissyandeEtc}. Moreover, GitHub's openness and transparency features, which allows users to easily see all activities inside a repository or from a user they're following, fosters both direct and indirect collaboration \cite{dabbish2012social}.

In a previous study \cite{zagalsky2015emergence}, we found a number of ways in which GitHub impacted learning and teaching. Educators we conducted interviews with spoke of benefits such as the ability to monitor student work continuously and the ease with which they can reuse and remix course materials from other instructors. We also noted benefits which impact their students, such as learning how to use a tool relevant to their field, and the ability to make changes to course materials.

%!TEX root = icse_seet16.tex
\section{Methodology}
Our study used exploratory research methods \cite{easterbrook2008selecting} to gain insights from students on the ways and potential ways GitHub impacts their learning experience. Having explored the perspectives of educators in a previous study \cite{zagalsky}, we strove to explore the student perspective on how the use of GitHub can benefit their learning and the challenges they might meet in using GitHub for their education. The research questions addressed during this phase include:

\textbf{RQ1: What are student perceptions on the benefits of using GitHub for their courses?} We've seen evidence that GitHub can benefit educators in a number of ways \cite{zagalsky}. As such, we wanted to explore student perceptions on how this tool and this way of working might also benefit them.

\textbf{RQ2: What are the challenges students face related to the use of GitHub in their courses?} When adopting a new tool for a course, particularly a tool not tailored towards education, there may be some friction involved due to a lack of educationally-focused features. We aimed to identify these challenges so as to make recommendations towards designing a system more suitable for educational purposes.

\textbf{RQ3: What are student recommendations for instructors wishing to use GitHub in a course?} Just as there are multiple ways to use GitHub for development purposes, an educator has multiple options regarding how they can utilize GitHub as a tool for their courses. We aimed to learn what students believed to be important for instructors to consider when creating a GitHub workflow that students would deem appropriate for their courses in order to extract recommendations for instructors who wish to use GitHub in future courses.

% \bigskip
% \textbf{RQ4: From the student perspective, how does GitHub compare to traditional Learning Management Systems?} Specifically, we aimed to discover student perceptions on currently used Learning Management Systems (LMS) like CourseSpaces (Moodle) and Connex, specifically pertaining to GitHub's potential as such a portal for student interactions in their courses.

\subsection{Research Design}
We used a qualitative approach to study the student perspective of GitHub use in education. As Creswell \cite{creswell2013research} suggests, a qualitative and exploratory approach best suits research when a concept or phenomenon requires more understanding because there is little pre-existing research. Yin \cite{yin2013case} introduces case studies as \textit{``an empirical inquiry that investigates a contemporary phenomenon within its real-life context, especially when the boundaries between phenomenon and context are not clearly evident.''} Case study design, according to Yin, should be used when the study is focused on the natural behavior of participants and when the context is important for the study. Because these conditions apply to the nature of the research questions asked in this study, we chose the case study design for this work. Specifically, the study was exploratory, serving as an early investigation on the student perspective of using GitHub in the classroom and to potentially build new theories or derive new hypotheses \cite{easterbrook2008selecting}.

Specific to software engineering, Runeson \cite{runeson2012case} defines case studies as \textit{``an empirical enquiry that draws on multiple sources of evidence to investigate one instance (or a small number of instances) of a contemporary software engineering phenomenon within its real-life context, especially when the boundary between phenomenon and context cannot be clearly specified.''} In this work, we aimed to draw from multiple sources of evidence---students and instructors, interviews and a survey---to investigate the potential of using GitHub for post-secondary computer science and software engineering courses. It's important to learn student perspectives in this context and to explore the suitability of GitHub for supporting education.

\subsubsection{Recruitment}
%The participants in this phase included the main stakeholders in technical courses: the professor (the main instructor), lab instructors, and students. As the learning tools used in courses directly involve and impact these stakeholders, I wished to obtain their perspectives and opinions to answer our research questions. The aim was to explore their perspectives while the course was ongoing so that they could recall recent experiences and provide their opinions on GitHub as a learning tool.

For this study, we were opportunistic in finding cases and sought instructors who could and were willing to try using GitHub. We were able to recruit a professor who wanted to try using the tool in two different courses offered in the same semester. The two courses were a computer science (CS) course aimed at both undergraduate and graduate students, and a software engineering course (SE) that was only taken by undergraduate students. Both courses were similar in size (30-40 students) and in learning activities (weekly labs and two course projects). Student participants were recruited via a sign-up sheet during the first lectures of the course which gave us permission to contact them. Participation was voluntary and students who signed up were not required to participate in the study at all in any given phase, meaning those who were interviewed did not necessarily respond to the validation survey.
%Having multiple cases would allow us to explore some possible differences between the two scenarios \cite{yin2013case}.

% When the term began, we attended one of the first lectures to describe my goals and to recruit students to participate in the study---interested students signed up by providing their names and email addresses, giving us permission to contact them to participate in various phases of the study. However, participation was voluntary and students who signed up were not required to participate in the study at all in any given phase. This method of recruitment meant that different students participated in the different phases. For example, those who were interviewed did not necessarily respond to the validation survey.

%preliminary survey stuff removed
\subsubsection{Research Methods}
Students who signed up during the recruitment process were emailed to be invited to interview. Most interviews with the students were one-on-one; however, due to scheduling reasons, some students requested to be interviewed as a group of 2 or 3. Interviews with the students lasted 20-30 minutes and were all conducted face-to-face in a meeting room. Audio from every interview was recorded with participant consent and notes were taken for reference. The interviews were semi-structured based on 12 guiding questions and we probed further with additional questions as deemed appropriate. %This supported the exploratory nature of the work and allowed for the discovery of interesting insights.

We also interviewed the course instructors: the main course instructor for both courses and the lab instructors (one for each course) were interviewed towards the end of each course course in order to find out how they utilized GitHub in their labs and to uncover their opinions on the tool's effectiveness towards the learning activities they engaged in with the students. Interviews with the instructors had a similar format to those with the students: semi-structured, 20 minutes long, with approximately 7 guiding questions.

Finally, we conducted a survey to validate the findings and confirm or contradict the themes that emerged from analysis. The survey was distributed during the final lab session of each course, where students were asked to anonymously fill in an online survey about their experiences. Respondents did not neccesarily participate in the interview phase. 18 students responded from the CS course (4 of which were interviewed), while the SE survey received 15 responses (9 of which were interviewed).

\subsubsection{Interview Participants}
We conducted interviews with 13 students from SE, 1 of which was in both courses, alongside 6 others from CS. These interviews began about 7 weeks after the start of the semester to give students sufficient experience with GitHub, concluding at the end of the semester. The main distinction between the two courses was that SE was an undergraduate Software Engineering (SENG) course whereas CS was a Computer Science course with a mix of undergraduate and graduate students. Otherwise, the courses were laid out in a similar manner (as outlined below in Section 5.5). Table \ref{table:interviews:students} summarizes the students who participated in interviews.

\begin{table}[h]
    \vspace{1pt}
        \caption{Participants and their prior experience with GitHub.}\label{table:interviews:students}
    \vspace{1pt}
    \begin{center}
        \begin{tabular}{c | c | c}
            \hline
            ID & Prior GitHub Experience & Degree Type \\
            \hline
            CS1 & Inexperienced & Graduate \\ \hline
            CS2 & Used Academically, Professionally & Graduate \\ \hline
            CS3 & Used Academically, Professionally & Graduate \\ \hline
            CS4 & Inexperienced & Graduate \\ \hline
            CS5 & Used Academically & Graduate \\ \hline
            CS6 & Used Academically & Graduate \\ \hline
            SE1 & Used Academically, Professionally & Undergraduate \\ \hline
            SE2 & Inexperienced & Undergraduate \\ \hline
            SE3 & Used Professionally & Undergraduate \\ \hline
            SE4 & Inexperienced & Undergraduate \\ \hline
            SE5 & Used Personally & Undergraduate \\ \hline
            SE6 & Used Academically & Undergraduate \\ \hline
            SE7 & Used Professionally & Undergraduate \\ \hline
            SE8 & Inexperienced & Undergraduate \\ \hline
            SE9 & Used Professionally & Undergraduate \\ \hline
            SE10 & Used Casually & Undergraduate \\ \hline
            SE11 & Used Professionally & Undergraduate \\ \hline
            SE12 & Used Academically & Undergraduate \\ \hline
            SE13 & Used Academically & Undergraduate \\ \hline
        \end{tabular}
    \end{center}
\end{table}

\subsubsection{Data Analysis}
Following the interviews, we transcribed every interview, then read and re-read the content for familiarity, noting important sections or responses. Next, the data was coded by a researcher by labeling various segments based on the research questions of the study. Afterwards, we identified themes and concepts that surfaced multiple times. After separating the themes into well-defined categories, we compiled a final list of themes.

% \section{GitHub Use}
This section describes the courses used in the case study and details GitHub and how it was used as a platform to support learning and teaching.

\subsection{Courses}
The two courses were a computer science (CS) course aimed at both undergraduate and graduate students, and a software engineering course (SE) that was only taken by undergraduate students. Both classes were similar in size (30-40 students) and in learning activities (weekly labs and two course projects). The course projects involved programming individually or collaboratively working with others (in groups of 2-4 students) to produce a project relating to the course topics - in one case, projects relating to the use of existing software systems to create new ones and in another, projects relating to building systems that involve multiple computational devices.

Projects were open-ended with regard to what the students created, what topic they address, and what technologies and languages they utilize for building. However, student work was required to be publicly available so that other people, both inside and outside of the course, could view their projects. The overwhelming majority of students opted to use GitHub to host their projects. However, there was no formal introduction of GitHub to the students. The course instructor for both courses informed them that course materials were to be hosted on GitHub, and during the first laboratory session, students had to create GitHub accounts if they did not already have one. Otherwise, students were left to teach themselves and each other, or ask questions to their lab instructors should they have any.

\subsection{GitHub Use in the Course}
The course instructor opted to utilize GitHub in the same way for both courses, using its features in three pivotal ways: material dissemination through the course repository, lab work through the `Issues' feature, and project hosting through various repositories. The advanced use cases we discussed in Chapter 4, such as utilizing pull requests and assignment submissions, were not used for these courses. The main course instructor was aware of some of these features but was not comfortable using GitHub beyond their knowledge.

% \begin{figure}[h!]
%  \caption{The front page of the SE Course Repository---the course schedule}
%  \centering
%    \includegraphics[width=0.8\textwidth]{schedule}
%  \label{fig:schedule}
% \end{figure}

The main use of GitHub was for material dissemination: the instructor hosted a public repository which all students could access to find the work they had to do for any given week. The instructor would update this repository weekly, adding lab assignments, links to readings, and the student homework for the week, as seen in figure \ref{fig:schedule}. All of the content was organized into a calendar table made from Markdown, and it was visible on the home page of the course repository as a `readme' file. Students could `fork' the repository to request changes to be made if they wished, though this possibility was not advertised to the students explicitly.

The other main use case was in the repository's `issues' page, where all labs (2-3 hour long sessions once a week in addition to the course lectures) were hosted. These labs would often involve researching a topic and reporting results, or giving other groups feedback on their projects. A dedicated issue would be created for each lab, similar to a forum post, and students would then make comments on these issues based on their lab work. Students were free to work in groups, and when commenting on an issue, would `@mention' their group members to indicate who the respondents were.

GitHub was also used for project hosting. Although students were not mandated to use GitHub for their course projects, most projects were hosted on GitHub in individual repositories. These repositories were public so others in the course could view the work and give feedback.

In addition to GitHub, the course instructor opted to use CourseSpaces\footnote{\url{https://www.uvic.ca/til/services/services_cs/index.php}}, a version of the Moodle LMS\footnote{\url{https://moodle.org/}}. CourseSpaces generally allows instructors to make their course content available for students to access and interact with, to enable communication between the instructors and students through forums, to post quizzes, create wikis for a class to edit, and to track student progress and performance. For these courses, CourseSpaces was used for work that the course instructor felt should not be publicly available. For example, student grades were hosted on CourseSpaces, as well as student responses to the course readings as the course instructor felt that their potential criticisms needed to be private.

%!TEX root = icse_seet16.tex
\section{Findings} % (fold)
\label{sec:Findings}

%We present the findings according to each of the research questions (RQ) posed for this study. For each question, we highlight the main themes that emerged from my analysis, providing relevant participant quotes from the interviews. Each participant quote will be identified depending on which course they were taking, as seen from Table \ref{table:interviews:students}. Before that, we provide some insights how GitHub was used in the courses in our case study.

\subsection{Student Perceptions on The Benefits of Using GitHub for Software Engineering Courses}
Previous work~\cite{zagalsky2015emergence} has shown that the use of GitHub introduces numerous and sometimes surprising benefits in the learning and teaching context, however, it is important to consider how these benefits are perceived by the students. Investigating the students' perspective, our study reveals emerging themes on the student perceptions on the benefits of using GitHub.

%In this section, we discuss the benefits that emerged from the students' perspectives from the three main uses of GitHub in their courses: for schedule and material dissemination, for discussions, and for hosting their project work.  %Many of these benefits stem from GitHub being a tool commonly used in industry, as well as from the advantages that Git offers for managing individual and group work.

\textbf{Benefit: Gaining Industry-Relevant Skills and Practices}\\
In modern software development landscape, it is essential for students to be familiar with best practices (e.g., peer review, cross-team collaboration) and commonly used tools (e.g., continues integration tools, distributed version control systems). Many of the interviewees [SE2, SE3, SE4, SE5, SE6, SE7, SE8, SE11, SE13, DS4] mentioned that using GitHub in the course provided a good introduction to the tool and to relevant practices. \textit{``I think when you go and work in software development too, you should get used to [having] lots of eyes being all over your work; that's just the way it's gonna be, so it's practice before real life.''} [SE8]

% Students came into the course with varying degrees of experience with GitHub, as shown on table \ref{table:interviews:students}. Five interviewees had minimal or no experience using it, while others were very knowledgeable about the tool, either through their own uses, through group projects for other classes, or through co-op jobs. Many, at least those who attended the University of Victoria for the majority of their undergraduate studies, had some experience with Subversion, a different version control tool that is taught in a second-year course.

% Many of the interviewees mentioned that using GitHub in class provided a good introduction to the tool for them, even with just the basic use of GitHub to manage course activities such as material dissemination and discussion: \textit{``I think it's pretty good. I mean one thing is that because I'm using it in class, it's made me learn the tool \ldots and that's where the big takeaway is: that I've been able to transfer those skills, I've done some other projects just on my own time using GitHub.''} [SE2]

% For the most part, students who supported this theme believed that the use of GitHub for their courses and projects helped them experience a style of collaboration that they will encounter often in their careers. In comparison to the use of more traditional LMSes, one student noted why using GitHub might be advantageous for them: \textit{``Well, I like how it's the bonus of more practice of something you're gonna use in industry, whereas none of us are gonna use CourseSpaces or Connex when we're out on a co-op or out on a job.''} [SE3]

% Importantly, however, putting their projects on GitHub provides practice for real-life scenarios. SE8 describes why it was beneficial to have their work publicly available for both classmates and outsiders to see: \textit{``I think when you go and work in software development too, you should get used to [having] lots of eyes being all over your work; that's just the way it's gonna be, so it's practice before real life.''} [SE8]

% Beyond the benefit of using GitHub in programming projects, which is what it was designed for, the basic use of GitHub to manage course activities such as material dissemination and discussion was also beneficial to students as an introduction to the tool, with some caveats. \textit{``It's a good introduction to GitHub as a platform; it might not be a good introduction to Git as a tool. Because there's a lot of wizardry that you can do with Git that you'd never learn just doing what we did here \ldots but definitely a good start to get people using Git.''} [SE11]

Despite previously using GitHub, the use of GitHub in the course introduced some students to specific features that they were not necessarily aware of, features they believe are important to learn. \textit{``This is the first time I've actually used the issues portion of GitHub.''} [SE13] Even with the most basic use of GitHub in a course (material dissemination), students were able to reap the benefits of being exposed to a prevalent industry tool.

% Out of all the benefits described by students, the benefit of getting an introduction to GitHub and its features was talked about the most, as [SE2, SE3, SE4, SE5, SE6, SE7, SE8, SE11, SE13, CS4] specifically mentioned this benefit. The importance of this benefit is further emphasized by these students asserting their intention to continue using GitHub or to use GitHub even more after the course ends. This benefit was shared by students irrespective of their prior experience with GitHub. \\

% \textbf{Benefit: GitHub as a Portfolio} \\
% Many students believed that using GitHub to host their course projects will be beneficial to them in the future. These students described that hosting their code from other courses or from personal projects on their GitHub accounts benefited them in various ways. For example, SE5 organized their code on GitHub for easy access when helping friends: \textit{``I know that when you're trying to help somebody out, you can always just say `Check out my GitHub', I know I've done that with a few of my buddies \ldots and I don't have to search through my files, it's just on GitHub, and you look on there. It's a good organization tool.''}
%
Another important aspect is mutual assessment~\cite{Singer2013}, and the ability of students to use GitHub as a portfolio. Interviewees [SE5, SE6, SE7, SE8, SE11, SE13, DS3, DS4] mentioned the importance of publicly presenting their work on GitHub. It is not uncommon for employers nowadays to refer to GitHub for hiring purposes\footnote{\url{http://www.cnet.com/news/forget-linkedin-companies-turn-to-github-to-find-tech-talent}}. In fact, some of the interviewed students already experienced potential employers inquiring them for their GitHub accounts: \textit{``I think all three companies that I applied to this semester wanted me to link to my GitHub. So I was really lucky that I had [a class] project on there. And I think when this [course's] project is done too, it'll also be really nice to have up there, after we clean it up.''} [SE6]

% CS2 also shared that interviewers inspected their code during an interview, highlighting the importance of having functional code in one's GitHub account: \textit{``These days I see that employers also want to see your GitHub page. While I was giving an interview for my coop, he did actually go into my GitHub profile and try to compile some of my code, so they do want you to have some online presence on GitHub.''}

%\textit{``Well I believe it's good for future employers. I remember I put directly on my resume saying you can check out the work I've done on GH. I included the link right on there and every person I handed my resume to were just like \'hey, fantastic!\' \ldots it's a good way to get your skillset out there.''} [SE5]

%The ability for students to use GitHub as a portfolio where they can show off their projects to potential employers was, for many, an important benefit of using the tool. There were students who were introduced to GitHub as part of their course, but knew the importance of having work on GitHub. This benefit motivated some students to continue putting their work on GitHub.
% The benefit of using GitHub as a portfolio was shared by [SE5, SE6, SE7, SE8, SE11, SE13, CS3, CS4]. \\


\textbf{Benefit: Supporting Student Contributions to Course Content} \\
One of the benefits that GitHub offers over traditional Learning Management Systems is the ability for students to suggest corrections and make the change themselves to the course material via Pull Requests (PRs) that instructors can easily reject, accept, or ask for additional resubmission.

Throughout the two courses in this case study, three PRs were submitted to make fixes to the materials or to add links to new materials. These pull requests were submitted in the first month of the courses and by only one student who was well-versed in GitHub (and who was registered in both courses). SE1 explains their reasoning: \textit{``I like being able to fix the mistakes that [the course instructor] might make, like with a bad link or something, by making a PR \ldots I really like being able to do that because it makes me feel a little more involved.''}

This style of contribution didn't continue, however, perhaps because these initial pull requests to fix material or add links to other content were not merged quickly enough. When they needed to be immediate, the instructor did not merge them for a day or two. Moreover, this method of contributing to the course is limited because of the type of files GitHub and its \emph{diff} feature supports, where they are optimally plain text files rather than files common to educational materials such as PDFs and Powerpoint presentations. However, students felt that contributing to the class material could have been a useful exercise had they taken advantage of it more, or had the feature been more advertised.

% Another student described how this hindered more participation of this kind: \textit{``Because we did not have the access. If we had the access, then I think people would have collaborated''} [CS2]

% Although SE6 did not contribute in this manner, they saw the potential advantages of using pull requests as follows: \textit{``I think everybody's had experience with mistakes in the course material. \ldots The alternative is just emailing the prof and asking them to change something \ldots this is always there, and they can always check it to see if there's something. This way someone can actually make the change, all they'd have to do is accept it.''} %SE6 discussed the convenience this feature offers to instructors, as changes are listed on a separate page in the GitHub repository and can be accepted with one click. Of the three PRs submitted to the courses, two other students participated by either trying to accept the PR (and failing), or adding a `+1' to the PR's comments, supporting its acceptance.

% However, this method of contributing to the course is limited because of the types of files GitHub supports.  \\ %Nevertheless, the following students agreed that being able to contribute to course materials via GitHub is a benefit: [SE1, SE3, SE5, SE6, SE10, SE13, CS2, CS4]. \\

\textbf{Benefit: Support for Students to Contribute to Each Other's Work} \\
In these courses, projects were open and visible to other students, which allowed more opportunities for student contribution, a workflow that GitHub encourages. This was demonstrated by a student's group getting feedback from and providing feedback to others: \textit{``For instance, one [issue] was our script wasn't taking in command line arguments if there were spaces in them properly. And then someone [gave us a suggestion that we used]. And then to be able to see what other people are having problems with and give suggestions.''} [SE3]

Making students host their course projects on GitHub and relying heavily on GitHub in the courses resulted in students looking at each other's work and making contributions in the way of advice or suggestions. While some lab assignments required students to look at other repositories, many reported that they would often peruse other projects outside of the requirements. As well, some students actually utilized code from other groups, so they helped fix issues in the code when necessary. \textit{``I believe that one other group decided for project 2 to use [our project 1] and they made a couple of pull requests I think.''} [SE10] Facilitating student contributions to each other's work is one such way that GitHub can encourage a participatory culture \cite{jenkins2009confronting} where students can create and remix content easily and believe their contributions matter.

Consequently, SE11 extended the feedback the students provided each other to the idea of peer reviewing, where students would judge the work of others and make comments on their work, explaining the benefits of such a system: \textit{``I thought [peer reviews] was the best way to learn actually \ldots It forced you to put yourself in a position where you have to defend what you did, which I think is good for quality because you have to actually care.''} The ability to open up student work to anyone simplifies the peer review process, particularly with the `social coding' emphasis of GitHub where others can make inline comments or corrections themselves.

% Helping other projects, either through discussion or through collaborating on the code, offered students new ways to participate that are unique to GitHub and similar types of systems. Students effectively collaborated with each other with the aim of producing better work, as was noted by [SE2, SE3, SE5, SE7, SE10, SE11, SE12, SE13]. \\

\textbf{Benefit: Keeping Each Other Accountable} \\
One benefit that stemmed from GitHub's transparency features was the ability to see a history of commits to a project. This was cited by some students who used GitHub to manage their group projects---they could easily see if and when their partners submitted work. Their repositories kept an account of when each change was made, which provided collaborators an easy way to track the work being done on the project. This helped the students to keep up with each other's work: \textit{``You can see exactly what the other person has contributed, and you can look it up again a month later \ldots then it's a good way to keep accountable. And it's good for yourself too, because you know they can see your work, so you wanna make sure that it's top notch and easily readable.''} [SE5]

% Moreover, the students knew exactly how much work each member of their group contributed to the project and this helped the student keep themselves and each other accountable: \textit{``We decided to switch to pull requests instead of just committing straight to master, because \ldots for a couple of reasons, first of all, if there's something majorly wrong with it, everyone can see it, right? And the second thing is, everyone sees it, so if people have to work on [the same code], in the future, which we all did, then they know exactly what just went in, so that next time they come to the code and pull it, they're not like `where did this all come from?'\,''} [SE9]

% This is a useful feature to have when working in group work as it allows for awareness between group members. By using GitHub for their group projects, students were able to take advantage of the collaborative features that GitHub offers to improve their processes or their product. This benefit was described by [SE5, SE9, SE11]. \\
%%%http://www.cs.usask.ca/faculty/gutwin/866/2010-T2/readings/p72-gutwin.pdf

\textbf{Benefit: Version Controlled Assignments} \\
% Using version control for their assignments and projects benefited the students in multiple ways. The ability to revert to previous states of the code was useful: \textit{``You're working on a project, and you make a change that breaks everything. Well you can just go back to a different commit, one that works. Boom, fixed, try again.''} [SE11]
%
Although the instructor for these courses did not use students' project repositories for marks, some students discussed the potential of GitHub where its use could allow instructors to give constructive feedback while they built their projects and assignments. One student believed that the ability to see the student's process could be important: \textit{``You'd see all the mistakes they made getting there, too, which is just as important to learning as the finished product.''} [DS3]
\\
% SE8 described a the potential of using GitHub for grading, where instructors could use the student's repositories as submissions as opposed to the traditional way of submitting through an LMS: sending the code only when finished. SE8 said that this way of submission would be \textit{``so much more useful \ldots You could see everybody's contributions, you could comment on them too \ldots Unless you're doing a live code demo with a TA or any instructor, you're not getting any real feedback [with the traditional submission system] \ldots You have no idea where you lost the marks or where you went wrong.''}

\textbf{Benefit: Bring in Outside Sources of Learning} \\
The final benefit that emerged from the interviews relates to the way in which work hosted on GitHub is often publicly available for others to contribute to. For example, SE1 was highly active in the community of a certain programming language, and for their first project, they were building something related to the language. They advertised their work to the community, and members from the community then tried to help with their project in multiple ways: \textit{``So here I have people involved in the discussion. These are just people in the community I've been talking to about how to do different things, and they've been giving me suggestions. And that's really cool because I actually have some community involvement in my course project.''} They also noted how this was helpful: \textit{``But for me, I find it really validating when someone else is like `that is really cool, have you considered doing this?' ''} [SE1] The ease in which students can bring in outside sources of learning and allow them to contribute addresses the `walled garden' issue of traditional LMSes \cite{mott2010envisioning}, where people who aren't course participants can be brought in and content can be shared across courses.

% SE1 was the only student interviewed who used the public nature of the course projects to solicit outside contributions. However, the exposure to GitHub gave students opportunities to discover work outside of the course and to use other repositories to aid their projects. When prompted, most interviewees mentioned that they sought out public repositories either to pull their code and use it, or to find inspiration for their own projects. One student recalled an experience where their group looked at an open-source library: \textit{``We just looked at how Gitstats, [an open-source library] did it, and then wrote our own thing into our project \ldots I think that more than anything is the biggest reason why Git should be used for education, because it takes, I think, until you start being forced to do it \ldots to actually go and look at other people's code, and I think looking at other people's code is the most important thing.''} [SE6]

% Speculatively, however, this likely would have happened regardless of whether or not GitHub was pushed by the instructor as students tended to seek out other code and libraries for their projects: \textit{``And in industry, the first thing you do is check Stack Overflow, look for someone else who has done the same thing and jack their code.''} [SE7] [SE1, SE2, SE3, SE4, SE6, SE7, SE10, SE12, SE13, CS2, CS5] mentioned looking at outside work and public repositories for their projects.

\subsection{RQ2: What are the challenges students face related to the use of GitHub in their courses?}
This section outlines the challenges the students described relating to GitHub use in courses. Some of these challenges were related to tool literacy, where more knowledge of the tool and more experience using it in an educational context could have mitigated the challenges. Yet they are worth mentioning as potential challenges that students might encounter. \\

% Noel: I renamed this quickly from 'Privacy is All-or-Nothing', I think it better describes the point. That said, not sure about the name!
\textbf{Challenge: The Perils of Public Projects} \\
While publicly sharing student projects on GitHub publicly provided several benefits, others acknowledged that it may not be appropriate for a class environment. SE4 describes this dilemma: \textit{``So [using GitHub for your work has] got benefits and drawbacks: benefits being that other people can access your data, drawbacks being that other people can access your data.''}

Most interviewees didn't mind that the class repository and their project work was public. However, many students could see the potential problems that might surface from their work being publicly available. Students mentioned two issues, 1) that their school work may not be of interest to the public and 2) that although they would ideally put 100\% effort into all their submissions, this is not always realistic due to the time constraints students face. For example, SE6 acknowledged that sometimes, students rush through their work, and therefore, they might not want that work to be publicly available: \textit{``You know it would actually be nice if they were separate or private somehow so I wouldn't have to go through everything and sanitize all the stuff I've submitted, because for as much as you'd want to think you're putting 100\% into it, you're not really. So private would be nicer for things like that.''}

%For example, CS3 noted that although it can be advantageous to host code publicly so that employers are able to see their projects, the employer may not always agree: \textit{``I think it comes back to what do you want to show your employers? When your employer looks at your work, will they understand that work I submitted in Git was when I didn't yet understand what I was doing, I was still learning? \ldots If I could make the assumption that an employer would understand that, I would have no problem with it being public. That said, I can't make that assumption. I have to assume that everything they look at they're judging in the harshest light possible. So I try to show only things that are of quality that I'm proud of. And that's unfortunately not a lot of the classwork until I'm done with it \ldots The final product I'm happy to show, but all those steps getting there, they're often filled with pitfalls and horrible programming and badly factored code.''} As such, courses mandating the use of Git and GitHub to publicly host student work could be problematic for students if employers are looking at work-in-progress in a negative light.

%not everything is 100% effort
% SE6 acknowledged that sometimes, students rush through their work, and therefore, they might not want that work to be publicly available: \textit{``You know it would actually be nice if they were separate or private somehow so I wouldn't have to go through everything and sanitize all the stuff I've submitted, because you know, for as much as you'd want to think you're putting 100\% into it, you're not really, you know, writing some great work of art or careful analysis, so private would be nicer. For things like that.''}

%not everything is of interest to public
% As well, SE1 felt that some of the work in the course repository wouldn't even be of interest to the public or to potential employers, and as such, they saw no need for the repository to be public: \textit{``I'd rather have [our comments] be private. But only because there's not a whole lot of participation, so I don't feel they're of interest to someone publicly.''}

% \emph{Discrepancy:} However, others saw no issue and even preferred all their work be in the public space. \textit{``Personally I don't have a problem with it being public. I would like to have a good online activity of myself on GitHub, so that's not really an issue. I'm not really concerned if someone is going to read my blog or not.''} [CS2]

One student acknowledged that there are workarounds to some of these privacy issues---where students do not have to attach their names to the work they contribute to the GitHub repositories used for the courses. \textit{``I think part of that would be \ldots you can decide that on your own, depending on if you use your main git account or just make a separate git account for your class.''} [SE3] Indeed, one student created a new GitHub account solely for their contributions to the class. Unfortunately, this student did not want to be interviewed, and group members were uncertain as to what the motivation behind creating a new user was; presumably, they were motivated by some of the issues discussed above.

% In summary, although students enjoyed the benefits that came with making their work publicly available, many students also acknowledged that these benefits are accompanied by a number of caveats. Importantly, some students described the lack of a middle ground as a limitation, where in the context of this course, students had to host their work publicly. This challenge may be mitigated by the instructor giving the students the option of creating a new GitHub user account for work done in their courses, as suggested by SE3 above. This challenge was shared by [SE1, SE4, SE5, SE6, SE7, SE10, SE13, CS3, CS6], while [SE3, SE5, CS2, CS5] disagreed on these issues. \\

\textbf{Challenge: Lack of Training on Git and GitHub} \\
% For example, SE9 believed that students who were less experienced with the tool could not take advantage of its benefits, such as the ability to make pull requests on the course materials. SE9 believed that if the instructor did not set a precedent for that behavior, it may not be used: \textit{``I think you just have to advertise it so that the students know [to] use this as a communication tool. And then layout or give some examples on how it could be used.''}

Another issue that many students described related to education and training is that there were varying degrees of experience with and knowledge of GitHub and its features, which presented difficulties with the use of GitHub in these courses. The main course instructor for these two cases was inexperienced with using GitHub, which made it difficult to educate the students on its features and caused some frustration for some of the interviewees. In fact, most students who were asked mentioned that the course could have benefited from more education on Git, GitHub, and what they can do with it. Students said that they could have hosted a lecture or a lab dedicated to learning the tool, perhaps at the beginning of the course or as an extra session. \textit{``I think it would've been good to do some demo \ldots cause I think [the instructor] talked too much about theory in class and there's no actual coding or no actual demoing.''} [DS1]

% SE2 acknowledged the potential difficulties in hosting such a session: \textit{``On the other hand, when someone teaches it to you, it often doesn't make sense until you actually do it yourself. Cause you'd actually have to go through the struggles of actually doing a commit and pressing all the buttons, so I don't really know how much could be done in that regard.''}

% Students also asserted that the University of Victoria needs to further emphasize teaching version control systems such as GitHub at the undergraduate level. As it stands, there is one required course that teaches version control systems and how to use them, utilizing Subversion and touching on Git. Some students, however, felt that one course was not enough, particularly when Subversion is not very popular anymore: \textit{``I think in [SENG265], we did SVN, which is a good introduction to the idea. But I don't think it's widely used anymore.''} [SE3]

% Three of the interviewees [SE1, SE6, and CS3] believed that students should get an account quickly after their first introductory Computer Science courses. \textit{``If I was teaching someone how to code, as soon as they start working on code that was bigger than 100 lines, I would teach them how to use version control.''} [SE1]

This is an issue of tool literacy and an instructor who was experienced in using GitHub might have been able to better educate their students on GitHub and the features they intended to use. Beyond the instructor, many students asserted that this issue could have been alleviated by a greater focus on version control, DVCSes and what students can do with these tools earlier in a curriculum. As it stands, students were not able to properly utilize some of the benefits of using GitHub due to inexperience and unfamiliarity. %This was discussed by [SE1, SE2, SE3, SE4, SE5, SE6, SE9, SE11, SE12, SE13, CS1, CS3, CS5]. \\

\textbf{Challenge: GitHub is Not Built for Education}
A few students described one drawback from using GitHub for education---GitHub is simply not built for it. Those that mentioned this particular drawback acknowledged that although there may be workarounds for many of the tasks needed, GitHub certainly struggles to meet some basic educational needs such as gradebooks and a formal assignment submission feature. This was one reason why the course instructor for this study decided to use a version of Moodle in conjunction with GitHub---to ensure privacy with matters such as grades or to make announcements, something that would be too cumbersome to do in GitHub.

This issue potentially hinders some of the benefits listed above. For example, it is more difficult to contribute to the course content because GitHub's \emph{diffs} feature does not support file types commonly used in education (such as PDFs and PowerPoint presentations). \textit{``I think one drawback of GitHub is that you cannot actually see the diff [of] commonly used files such as PPTs or PDFs, so you can't really use it for correcting professor's slides, or PDFs.''} [SE12] The pull request process then requires an extra step, which some students felt would be discouraging the use of this feature. \textit{``I think for readme files, it's a lot easier to edit, cause you can edit directly in GitHub. But for other files, you'll probably have to change and make a branch and then commit it and then send a PR, it might actually be more work.''} [SE13]

\textbf{Challenge: Notification Overload} \\
%Although few students brought up this issue, how GitHub handles notifications from the repository nevertheless emerged as a challenge.
Another challenge lies in how GitHub handles notifications from the course repository. The only way to get notifications from a repository is to `watch' the repository. `Watching' provides two different options: 1) to get a notification and an email only when the user is mentioned in issues or commits, and in discussions the user has commented on; or 2) to get a notification and an email when anything at all happens in the main branch (master), when someone comments on issues, commits, or pull requests, and when someone makes or accepts a pull request. %The `Watch' feature comes with some drawbacks, not the least of which was how a student's lack of familiarity with the feature prevented them from using it: \textit{``I didn't like that [repository] at all, because I didn't get notified when [the instructor] adds stuff to there, so I don't really know what's going on without remembering to check it on GitHub. ''} [SE9] This student did hear about the `watch' solution, but thought that it \textit{``would be a good solution, but it might be overkill. For like a spelling change.''} [SE9]

Unless students were `watching' the repository, they would not receive email notifications for any activities unless they were directly mentioned. However, if the students did `watch' the repository, they would receive an influx of notifications for every user comment on the discussions, which can become overwhelming. SE10 shared that they were engaged less in the activities of others because of the noise from notifications: \textit{``It sent me a million emails, both of [the tools] actually. I should have just turned that off, but I was worried about missing something. Because every time someone would post, you would get another email \ldots I actually did not read anyone else's feedback because it was just so many emails, to be totally honest.''} [SE10]

% As such, the `Watch' feature was problematic for courses like the ones studied, where every single comment would trigger a notification and an email, causing an overload of notifications. [SE7, SE9, SE10, SE11] shared this issue.

\subsection{RQ3: What are student recommendations for instructors wishing to use GitHub in a course?}

Given that the use of GitHub in these courses was relatively basic, many students, particularly those who were experienced with using GitHub for collaboration purposes, had ideas on how GitHub could be further utilized to be more beneficial for both themselves and for their instructors. Many students discussed recommendations such as which classes GitHub could best serve and the need to utilize additional GitHub features. This section outlines those responses, highlighting the suggestions students gave about the workflow for using GitHub in a course. \\

% \begin{table}[h]
%     \vspace{1pt}
%         \caption{Summary of student recommendations for instructors who wish to use GitHub for their courses.}\label{table:interviews:students:recommendations}
%     \vspace{1pt}
%     \begin{center}
%         \begin{tabular}{ | m{3cm} | m{12cm} | }
%             \hline
%             \emph{Recommendations} & \emph{Description} \\
%             \hline
%             Use GitHub in More Open-Ended Courses & Students recommended that GitHub should be used in courses were students are given more freedom for their projects and assignments rather than single-solution work. \\
%             \hline
%             Mandate the Use of GitHub's Unique Features & Students suggested that instructors use features such as pull requests and commit history extensively to take advantage of GitHub's benefits. \\
%             \hline
%             Define and Advertise a Workflow & Students believed the instructor should define a workflow for using GitHub and advertise the workflow to successfully use the tool in a course. \\
%             \hline
%         \end{tabular}
%     \end{center}
% \end{table}

\textbf{Recommendation: Use GitHub in More Open-Ended Courses} \\
As discussed earlier, students had concerns regarding the public nature of the work they host on GitHub. While most students interviewed did not mind their work and their comments being in the public space, there were concerns regarding how this way of working could apply to different types of courses, particularly courses in which students are afforded less freedom in the nature of their work. As such, some students suggested that for courses where the discussions are self-contained, the repository does not need to be public. This would avoid some challenges, such as students submitting incomplete or messy work, but would also conflict with some of the benefits extracted from these interviews, where students can build their online presence with public work and instructors can make their courses open for outsiders to contribute to.
%[SE1, SE5, SE6]

%As well, some did not even see sense in keeping a course repository open for the public because of issues highlighted earlier, such as that it just won't be interesting for outsiders and that their work might often be rushed and therefore unappealing to have on display.

%A suggestion for avoiding these issues came from one interviewee: \textit{``I think that as long as we have the option to make [our discussion comments] private, maybe after the course ends. So keep it intact while the course is ongoing and then we have the option [to change the privacy], everything will be okay.''} [SE12] Currently, GitHub does not support doing such tasks, unless the course instructor decides to privatize the course repository as a whole after a course concludes or an individual student deletes their comments and posts. %While this is a potential solution, these tools like GitHub would need work to meet such a solution. %this might be CSCS-worthy

However, students had opinions regarding what type of course would best suit GitHub. Interviewees suggested that a course similar to the two cases studied, where the work is very open-ended and could therefore exist in a public space, is where using a tool like GitHub would benefit students the most. When asked about their experiences with viewing others' projects in this course versus in other courses, SE7 said: \textit{``I would say this class is specifically different because we had so much flexibility over what we were doing. It's not like in our Operating Systems class, [where] we make a shell that does this, this, and this. Where this was way more open ended, everyone's doing something different, so even if you could see what everyone else is doing, no one could've helped us.''} [SE7] Other students acknowledged that the use of GitHub may not work in less-open ended courses because of plagiarism concerns.

% Students acknowledged that the open-ended nature of these courses was what enabled the successful use of GitHub, but that it would not work in less open-ended courses because of plagiarism concerns. Regarding the potential use of GitHub in their future courses, SE5 discussed: \textit{``Like I said, I like seeing other people's work and whatnot. Maybe not if everyone has the same assignment, because everyone's just gonna cheat off each other.''}

% These students related back to the privacy issue, where having completely public work might be a detriment to the work being done when there are concerns of plagiarism, such as when the assignments posted have a single solution. Some students, such as SE6, attempted to conceptualize a way to use private repositories, but ultimately felt it might be too cumbersome. As such, students believed that instructors would have to consider the nature of the work before deciding on the workflow they will use GitHub with, or indeed, whether they want to use GitHub at all. This consideration was suggested by [SE2, SE5, SE6, SE7, SE13].

It should be noted that there are ways to use GitHub privately within a course, where even student assignments are private. This type of workflow involves the instructor creating an organization and having each student create private repositories for their work to keep them private from each other. However, this workflow would then minimize many of the benefits listed in RQ1.

\textbf{Recommendation: Mandate the Use of GitHub's Collaborative Features} \\
The students who were more experienced with GitHub mentioned that GitHub's more collaborative features should have been further utilized to take advantage of the uniqueness of GitHub over traditional LMSes. One issue that some students discussed was that they saw little reason to use GitHub for courses if it was used only for material dissemination. For example, as mentioned in RQ1 above, only three pull requests were made throughout the semester.

DS3 was very outspoken on why using GitHub for this course was somewhat unnecessary: \textit{``I don't see any benefit that GitHub has offered that we wouldn't have had in CourseSpaces. All it appears to me is it's a place where it's a file repo \ldots and we already have that.''} They also noted that while there's potential, the unidirectional nature of the work being done meant that the potential benefits were not realized. \textit{``If there was a way to collaborate on the material, that would be useful \ldots But in this class, every one of our labs so far has been demo to the lab TA, so nothing's going back to GitHub \ldots Maybe if we were submitting things to it, maybe that would be helpful. I can see how it could be useful, it's just that in our usage it's not really adding anything to the experience.''} [DS3] %It should be noted that this student's project group did not use GitHub to collaborate, but they used Docker Hub\footnote{\url{https://hub.docker.com/account/signup/}} instead.

% SE7 echoed these sentiments: \textit{``I think that you can accomplish the same thing with a simple HTML website, honestly \ldots It's not using a lot of the features of Git, like looking at changes, commits, pull requests. The issues were kinda cool for the lab, and, again, you can accomplish that with any sort of forum, I would think \ldots We're not actually delivering code to the professor, so maybe it doesn't make a ton of sense [to be using GitHub].''}

% As such, many students believed that GitHub was not being used to its full potential in their courses. The underlying suggestion was to consider which features of GitHub the instructor would like to use, such as pull requests or grading via commits, and use those features thoroughly and consistently. As it stands, some of the benefits they described to using such a system were only possibilities. An example, which will be highlighted later in Section 5.8, was reported from a student in the CS course, where even the issues were not used for discussion during labs: \textit{``So basically we had to show it to our TA that we have done [the lab], and [the TA] used to mark it in a piece of paper. So putting [our responses in the issues] was not really necessary?''} [CS2]

Consequently, many students suggested that it would have been important to define a workflow for using this tool in a course in order to gain the benefits described earlier in the chapter. This workflow could include aforementioned activities such as utilizing pull requests or using just one tool instead of two. In the case of pull requests, for example, students advocated that the instructor should be advertising their use, thereby defining to the students that contributing to the material would be part of the course workflow: \textit{``I think [the idea is] good, but I think it would've needed to have been advertised more that [the instructor] was looking for input on things, and if [the instructor] said that, maybe more people would have [contributed] to maybe propose extensions for assignments or something.''} [SE7]

An important lesson to learn is that GitHub only equips instructors and students with the possibility to take advantage of the benefits on offer. It is then up to the instructor to realize those benefits by using the features involved. As such, the underlying suggestion was to consider which features of GitHub the instructor would like to use, such as pull requests or grading via commits, and use those features thoroughly and consistently. \\ %[SE3, SE5, SE6, SE7, SE11, CS2, CS3, CS4, CS6] touched on this suggestion. \\

% \textbf{Recommendation: Define and Advertise a Workflow} \\
% Students acknowledged that GitHub was not being used to its full potential and that there was confusion surrounding the use of two tools (GitHub and CourseSpaces). CourseSpaces was used to fill some of the gaps in education support offered by GitHub, such as private forums and a gradebook. However, students tended to be displeased with this decision. \textit{``One thing I really don't like is that we have both systems set up, and so sometimes the announcements are in GitHub, and some of the times, they're in CourseSpaces, and that can get kind of confusing, like did [the instructor] post an assignment here or there?''} [SE2]

% This was an almost unanimous issue between the students interviewed, with only a few stating that they did not mind either way. Most mentioned that they would have preferred the use of just one tool, even if everything was public in GitHub. As a result, many students suggested that it would have been important to define a workflow for using this tool in a course in order to gain the benefits described earlier in the chapter. This workflow could include aforementioned activities such as utilizing pull requests or using just one tool instead of two. In the case of pull requests, for example, students advocated that the instructor should be advertising their use, thereby defining to the students that contributing to the material would be part of the course workflow: \textit{``I think [the idea is] good, but I think it would've needed to have been advertised more that [the instructor] was looking for input on things, and if [the instructor] said that, maybe more people would have [contributed] to maybe propose extensions for assignments or something.''} [SE7]

% One student mentioned that although GitHub does not do everything needed in a course, defining a workflow will cover up many of those weaknesses: \textit{``Even if there are no enhancements on GitHub, but if you define a proper workflow for using it, then it can be quite successful, because even the present Learning Management Systems are not perfect right?''} [CS2]

% While most students did not have suggestions as to what workflow to use, they acknowledged the importance of defining it and teaching it to the students early on in the course. SE6 wanted to \textit{``enforce more actual Git and GitHub features in the way that we interact with the course material, and enforce GitHub use for actual projects. In a way that everybody had sort of a base level of understanding. So maybe at the beginning of the course \ldots there should definitely be a time when you learn Git.''}

% In summary, many of the students interviewed were frustrated by the lack of a clearly defined workflow, and believed that the course would have been improved greatly if a workflow had been created and advertised in the beginning. This recommendation emerged from interviews with [SE1, SE2, SE3, SE4, SE5, SE7, SE9, SE11, SE12, SE13, CS2, CS3, CS4, CS5].

\subsection{How GitHub was Used in the Courses}
Despite being relatively unfamiliar with GitHub and its features, the course instructor opted to utilize GitHub in the same way for both courses, using its features in three pivotal ways: material dissemination through the course repository, lab work through the `Issues' feature, and project hosting through various repositories. The advanced use cases other instructors described in \cite{zagalsky2015emergence}, such as utilizing pull requests and assignment submissions, were not used for these courses. The main course instructor was aware of some of these features but was not comfortable using GitHub beyond their knowledge.

The main use of GitHub was for material dissemination: the instructor hosted a public repository which all students could access to find the work they had to do for any given week. The instructor would update this repository weekly, adding lab assignments, links to readings, and the student homework for the week. All of the content was organized into a calendar table made with Markdown, and it was visible on the home page of the course repository as a `readme' file. Students could `fork' the repository to request changes to be made if they wished, though this possibility was not advertised to the students explicitly.

The other main use case was the use of the repository's `issues' page, where all labs (2-3 hour long sessions once a week in addition to the course lectures) were hosted. These labs would often involve researching a topic and reporting results, or giving other groups feedback on their projects. A dedicated issue would be created for each lab, similar to a forum post, and students would then make comments on these issues based on their lab work. Students were free to work in groups, and when commenting on an issue, would `@mention' their group members to indicate who the respondents were.

GitHub was also used for students to host their individual or group projects. Although students were not mandated to use GitHub for their course projects, most projects were hosted on GitHub in individual repositories. These repositories were public so others in the course could view the work and give feedback.

In addition to GitHub, the course instructor opted to use a version of the Moodle LMS\footnote{\url{https://moodle.org/}}. CourseSpaces generally allows instructors to make their course content available for students to access and interact with, to enable communication between the instructors and students through forums, to post quizzes, create wikis for a class to edit, and to track student progress and performance. For these courses, CourseSpaces was used for work that the course instructor felt should not be publicly available including student grades and student responses to the course readings.

\subsection{Validation Survey}
%In both courses, students indicated that their level of familiarity with GitHub had improved from when the course began. 30 students agreed that they would continue using GitHub for group work and for individual work after the course concluded. Given that 14 of these students were completely or somewhat unfamiliar with GitHub before the course began, students seemed to believe that using GitHub can be beneficial for them in some way.

A validation survey was sent to all students at the last laboratory session of each course in order to confirm or reject our findings. 33 students responded to a series of likert-scale style questions from each finding. 30 students agreed that they would continue using GitHub for group work and for individual work after the course concluded. Given that 14 of these students were completely or somewhat unfamiliar with GitHub before the course began, students seemed to believe that using GitHub can be beneficial for them outside of courses.

11 of the 15 SE respondents agreed with feeling more involved in the class from viewing and commenting on other projects compared to 8 of the 18 DS students who agreed. As well, while no students in SE felt that there was not enough collaboration in the courses to justify the use of GitHub, 7 of the DS students felt that way, which highlights a potential difference between how GitHub was used by the lab instructors or between the students of the two courses. %most students in SE (9) felt that there was enough collaboration or student contribution to justify using GitHub in the course, whereas only half of the participants in CS felt that GitHub use was justified.

Given the interview responses where privacy was a concern to many students, it was surprising that most students in both courses disagreed or were neutral with the suggestion that their school work should not be publicly available. 10 DS students disliked using the \emph{issues} as a discussion system on GitHub over forums with threaded discussions, while only 5 SE students disliked using GitHub `issues' for discussion. 10 students from each course disagreed that the classes needed a tutorial in the beginning of the semester, and both strongly agreed that Git, GitHub, and other DVCSes should play a bigger role in their curriculum.

% section Findings (end)
\section{Discussion}

The motivation behind this study was to uncover student perceptions on using GitHub as an educational tool by asking them to describe their thoughts and opinions during the experience. GitHub was used in three main ways: (a) as a place to disseminate material and host the class schedule, (b) as a place for students to submit their lab assignments and discuss these assignments, and (c) as a place where most students interviewed hosted their course projects, either collaboratively or alone.

\textbf{A Student-Oriented Learning Tool} \\
%What does GitHub provide? more opportunities for students to participate and contribute!
At a basic level, using GitHub for education can provide similar functions to those of traditional LMSes. As discussed in the last chapter, GitHub has the capabilities of providing many of the common activities found in Malikowski \textit{et al.}'s model of features found in LMSes \cite{malikowski2007model}. However, accomplishing tasks related to some of the finer-grain features of traditional LMSes, such as a formal assignment submission, requires workarounds. Even though GitHub can serve a similar purpose to formal educational tools, it was simply not built for education and is therefore lacking some educational features.

Where GitHub has the potential to excel, however, is in addressing some of the concerns regarding traditional LMSes outlined by various authors. Mott \cite{mott2010envisioning} discusses the `walled garden' approach of LMSes, lamenting that the content is limited to those officially enrolled in the course, and that the LMSes support administrative functions much more than actual teaching and learning activities. Garcia-Penalvo \cite{garcia2011opening} echoes these concerns, asserting that students need to be placed at the centre of the e-learning process. This could be addressed by giving students opportunities to participate in the course and connect with and learn from each other. GitHub can support these opportunities for students to become a part of each others' learning, creating a culture of participation \cite{jenkins2009confronting}. \\

\textbf{The Contributing Student} \\
GitHub provides opportunities for students to participate in their learning. Students are able to openly contribute to the course materials by making changes or additions directly to a course repository. Traditionally, students needed to talk to the instructor or send an email to make corrections or additions. GitHub provides a much more open and direct way for students to contribute to the course materials. This plays a key role in Collis and Moonen's concept of a `Contributing Student' \cite{collis2006contributing}, where GitHub provides students the ability to drive their coursework. Moreover, GitHub provides students opportunities to partake in many of the `Contributing Student Pedagogy' activities Hamer \textit{et al.} described \cite{hamer2011tools}, including peer reviews, discussion, content construction, solution sharing, and making links.

When student assignments and projects are public, GitHub can provide students the opportunity to contribute to other students' learning by easily providing direct feedback to each other's assignments or project work. A number of groups in one of the cases in this study used this ability by leaving feedback for other groups when they noted bugs or issues in the code, and students seemed to appreciate this ability to see others' work and provide feedback as they saw fit. Contributing to other students' work may provide benefits in developing soft skills such as communication and teamwork skills \cite{hamer2006some}. An instructor may also utilize GitHub to provide opportunities for students to peer review or grade each other's work. This could provide potential benefits such as more reflection for students while working, and the development of analysis and evaluation skills \cite{sondergaard2012collaborative}.

% However, it is important to note that like any technology, accessing these benefits requires the stakeholders to `buy in' and use the relevant features of the tool to support this pedagogy. It is possible, for example, that there were different levels of enthusiasm for the tool between the two courses because of the differences in how it was used in the lab sessions. The SE case required students to post often, which possibly encouraged them to look at others' responses, while the CS did not utilize the tool as much, requiring only a demo of the weekly assignments to the lab instructor instead.

\textbf{Transparency of Activities} \\
%accountability
In describing the benefits of using GitHub to support their group projects, some students described the transparency of activities as helpful for collaborating with each other. Few of the transparency features of GitHub were mentioned by the students---for example, the News Feed or the graphs were not discussed in the context of group projects. However, some students acknowledged the importance of seeing a history of work from other group members, describing the feature as a way to hold accountability and to keep up-to-date with the work. This is in line with the benefits related to GitHub use in industry \cite{dabbish2012social}.

%better grading from instructors
Moreover, some students described the potential for better grading methods as a benefit of the transparency of activities on GitHub, despite these courses not utilizing the tool for grading. Compared to the traditional way of assignment submission where an assignment is handed in as a complete product when it is due, GitHub offers instructors the opportunity to monitor assignments and projects, giving feedback while they are in progress. \\

\textbf{Beyond the Course} \\
%practice in tool
Supporting the findings from interviews with early adopters of GitHub in education \cite{zagalsky}, most of the students interviewed described being exposed to GitHub and its features as a benefit to using the tool in a course. As such, the exposure to GitHub its associated open, collaborative workflow may result in some transferable skills towards their careers. Moreover, the popularity of GitHub means that student GitHub accounts become part of their online presence \cite{treude2012programming}, which may serve an important role with potential employers who use GitHub for hiring purposes.

%outside help
With GitHub's popularity, many developers are putting their code on the platform, both publicly or privately. When a course is publicly visible, the `walled garden' that traditional LMSes tend to suffer from \cite{mott2010envisioning} can be overcome. Student projects, for example, could involve people from another community, or outsiders can contribute to the course materials in some manner. \\

% \textbf{Tool Literacy} \\
% %privacy issues
% An important note from some of the limitations that the students and the instructors described is the importance of understanding and being proficient with the tool. As an example, in discussing what considerations need to be made to design an effective workflow, students would discuss the difficulty of conducting courses with single-solution assignments rather than open-ended projects. This was due to the way in which GitHub repositories are required to be private or public, making it difficult to handle assignment submission.
%
% However, some experience with the tool or some investigation of GitHub's recommended practices for using their tool in education would have revealed the possibility of using private repositories for each assignment. An instructor could introduce new assignments or make clarifications in a student's private repository if they were simply added as a collaborator. As such, it is important to consider that some of the limitations described by the students and by the instructor may be from unfamiliarity with using the tool, especially in a context it originally was not meant to serve.

\subsection{Limitations}
One limitation in this study was that the semi-structured nature of the interviews meant that the interviewer would often go off-script to probe further, potentially resulting in leading questions. Moreover, the recruitment methods may have biased the population: by searching for instructors teaching appropriate courses, we first approached instructors we knew to invite them to participate in my study, which may have introduced a bias in comparison to finding a class that was already intending to use GitHub as a learning tool. As well, opportunistic recruitment may have resulted in a situation where the students willing to be interviewed were students who felt strongly about GitHub in either direction---those who may have had insights but had no strong opinions may have chosen not to participate.

%This resulted in a less than optimal use of GitHub (as described by many of the students interviewed) because the instructor had little prior experience with GitHub as a tool. As well, because of this inexperience, the researchers would give the instructor advice or resources on possibilities of how they can use GitHub to meet a goal---we didn't, however, directly give them step-by-step directions to avoid influencing the direction of the class as much as possible.

% In the data analysis, there was no inter-rater reliability because I was the sole coder. As such, biases may have been introduced in my selection of the themes. Having multiple raters analyze the data would have introduced more perspectives and interpretations, which would have reduced the potential biases in the analysis. Unfortunately, this study suffers from a single-rater limitation.

%As well, the opportunistic nature of recruitment may have resulted in possible biases in multiple ways. With only one instructor teaching two courses, this study is limited from having no other cases to compare with, particularly cases where the instructor might be experienced with using GitHub.

% \subsection{External Threats to Validity}
% External validity is concerned with the extent to which the findings from this work can be generalized and to what extent the findings are of interest to people outside the case \cite{runeson2012case}. As a case study, it cannot be assumed that these cases can be generalized to the use of GitHub in education. However, many of the findings are reflected in other studies that use similar tools for classes, such as Kelleher's study on Git and GitHub \cite{kelleher2014employing}, and Haaranen and Lehtinen's study on Git and GitLab \cite{haaranen2015teaching}. %Another external threat to validity, is that because of the instructor's lack of experience with using GitHub, these findings may not be of interest to those who are familiar and can visualize a specific workflow for using GitHub in the classroom.

% However, we used a number of approaches\cite{runeson2012case} to establish rigor and to minimize the threats to validity described above. One such approach included triangulating data from multiple sources---the students as well as the teaching team. In doing so, I was able to identify and highlight contradictions between different sources and report discrepant information.
%
% This study was also conducted over the span of the two courses which provided prolonged involvement with the population and allowed for a good understanding of the participants' perspectives. Moreover, we deployed a survey to students in both courses in order to validate the themes extracted from the interviews. This is a form of member checking and allows students to verify the analysis of the data. %Finally, this study also involved a peer with whom I discussed the study with regularly. This peer, who is experienced with qualitative research methods, was consulted throughout the study, including the study design and the collection and analysis of data, and would lower the risk of any biases affecting the study.

%In summary, this study has shown the effectiveness of using GitHub for educational purposes from the student perspective. This study describes the benefits of using GitHub for education, such as the possibilities for student contributions. However, these benefits are accompanied by limitations, such as the implications of having publicly available work on cheating and academic integrity. In the next chapter, I offer recommendations for instructors who want to attempt using GitHub in their courses in order to maximize the benefits of using the tool.

\subsection{Recommendations for Educators}
This section provides recommendations for educators who want to use GitHub to support their courses. These recommendations are based on the findings from the two phases of this work, as well as from the review of literature surrounding tools in computer science and software engineering education.

Before proceeding, we note that GitHub has their own set of recommendations for setting up an organization for a class\footnote{\url{https://education.github.com/guide}}. Their classroom guide is useful for those looking for a step-by-step process, where they recommend applying for an organization for a course and assigning a private repository for each assignment for each student. Likewise, it can also be helpful to use the available resources: use GitHub support, look for other instructor experiences for guidance, or discuss experiences in a blog or in spaces dedicated to the topic\footnote{\url{https://github.com/education/teachers/issues}}. Contributing to these resources can serve towards building a common knowledge base for instructors to share to and learn from. Moreover, GitHub recently released a tool that automates many of the tasks educators need to set up on GitHub \footnote{\url{https://classroom.github.com}}. \\

\textbf{Recommendation: Utilize GitHub's Features} \\
Computer science and software engineering students benefit from early exposure to Git and GitHub. By utilizing these (or similar) tools in their courses, educators provide students a way to familiarize themselves and practice with these tools, which can benefit their careers. Beyond exposure, hosting assignments, projects, and code on student accounts could be valuable when seeking employment, as companies continue to investigate the online presence prospective employees have (e.g., their GitHub accounts) for hiring purposes.

While simply using GitHub as a system for material dissemination can be helpful, using more of GitHub's features, such as pull requests and issues, provides even more benefits for the students. For example, allowing students to contribute to the course and to each other's work can help develop skills such as teamwork and communication \cite{hamer2006some}. For example, educators can use GitHub's transparency features to provide feedback to students in unique ways, such as tracing the history of student projects and assignments hosted on GitHub, detailing where students made mistakes and intervening when a student seems to be struggling. Moreover, in group projects, instructors can note how much work each student has contributed, and can use this transparency for assigning grades.

%As another example, exposure to GitHub's Issues feature, even for basic discussions, was helpful for one of the students interviewed during the second phase as the student learned how the feature works for use in future projects.
\\
%One important lesson noted from the case study was to communicate the workflow the instructor decides clearly and properly to the teaching team and to the students. When deciding to use a feature like pull requests on course material, for example, the instructor must advertise this workflow properly, perhaps even offering bonus points for added material. To communicate a workflow to students and introduce GitHub and its features to novices, instructors should consider creating a guide or hosting a tutorial session. \\

%notifications

\textbf{Recommendation: Use Free Private Repositories for Single Solution Assignments} \\
%type of course - better for open-ended
%assignment submission
Many students believed that GitHub worked best when a course has open-ended projects and assignments. This stems from plaigarism concerns that exist when students are putting their code up online where others can potentially see their solutions. Of course, students can host their code in private repositories controlled by the instructor; if the instructor creates a private repository for each student to submit their assignments and adds only the student as a collaborator, plaigarism would only be as much of a concern as it would be without using GitHub. Otherwise, an instructor could ask students to create a private repository for their assignments that only the instructors can view and contribute to.

% This style of repository management (where a private repository is dedicated to each student) could work for assignment submission as well. The instructor could ask the students to create a branch, or ask the students to fork off the main repositories and make the forks private, and then mandate that the student must make a pull request before a deadline. Thanks to GitHub's transparency features, an instructor can continuously observe the work in each student's repository and can provide further assistance to students based on the work history.

% However, the set up for this more private style of repository management requires some time and assistance from GitHub. An educator can create an organization for the course, which is granted an amount of private repositories depending on how much the instructor pays. While GitHub has stated that they would give teachers a free organization for their courses\footnote{\url{https://github.com/blog/1775-github-goes-to-school}}, an organization must be set up well before the course begins in order to get the private repositories in time.

However, single-solution assignments being hosted in private repositories limit one of the most important benefits of using a system like GitHub---the ability to view, comment on, and contribute to the work of other students. As such, although GitHub is usable and helpful in any type of course, courses with open-ended projects and courses with a culture of participation are where instructors and students will see the primary benefits of using GitHub as a learning tool. If an instructor chooses to pursue the open-ended style of work similar to the courses in this study, it is recommended that they list projects and assignments on the home page using the readme markdown file so students can easily access the other projects.
\\
% That said, GitHub continues to offer its benefits when used to submit single solution assignments. It involves some preparation to get free private repositories for students, but at the same time, it allows instructors to provide better feedback through versioning, and it maintains the benefits for students of learning Git and GitHub and hosting their work for future portfolio use (if allowed to publicize their work after the course concludes). \\

\textbf{Recommendation: Encourage Contributions from the Students} \\
%contributions from others (slides in html, comment on other projects issues)
Another way to utilize GitHub is to encourage contribution from the students in the ways that GitHub affords them. First, students can contribute to the course materials by making corrections, changes, and adding resources. Second, students can contribute to other students' work and projects (provided the work is open-ended). And third, students can contribute to projects outside the course by making changes and pull requests in open-source repositories. Encouraging this `Contributing Student Pedagogy' can help students develop skills such as critical analysis and collaboration \cite{falkner2012supporting}.

Moreover, all student contributions are available for the course instructor to see. As an example, an instructor can grade students based on their contributions, such as when they create an issue or a pull request on another project. However, one issue with student contributions that must be noted is that contributing to the course materials could present difficulties depending on the file types used, as binary files such as PDF documents and PowerPoint slides are not compatible with the GitHub web interface. Although GitHub has recently provided support for viewing PDF files on the platform\footnote{\url{https://github.com/blog/1974-pdf-viewing}}, these files remain unsupported by GitHub's `diff' feature, which means that changes to the file are difficult to discern and changes to the file by multiple people will always result in a `merge conflict'. For this reason, I recommend hosting class material and slides in either markdown or HTML, file types that GitHub supports and can be easily altered using its Web platform.

\section{Conclusion}

%Limitations

%Future Work

%Contributions
Another important consideration from this work relates to the future of tools for computer science and software engineering education---what's next? First, we consider the importance of participation, group work, and group learning for students in technical fields in order to develop non-technical `soft' skills such as communication and teamwork \cite{jazayeri2004education}. This work demonstrates how using GitHub can unlock activities where students can contribute to each other's learning, and as a result, I believe it can be beneficial to add support for GitHub's open, collaborative workflow to current and future tools focused on learning.

%GitHub for Education
% The fact that GitHub easily supports participatory activities has multiple implications. Literature has shown that LMSes have been adding `Web 2.0' features such as blogs and wikis to their feature set \cite{downes2005feature}---students are being offered more opportunities to participate by discussion or by contributing content in blogs or wikis. Where the GitHub Way excels in education, however, is in the opportunities for students to contribute to and change the materials, and to contribute to each other's learning by getting involved in and providing feedback to projects other than their own. This is potentially the next step for Learning Management Systems, where students are more easily able to make these contributions to the work of others. The concern, however, is that implementing features similar to GitHub in an LMS might seem forced and haphazardly planned, and tool builders would be better served building a tool that supports and encourages an open, collaborative workflow from the outset.

As such, one possible path is the `GitHub for Education' Greg Wilson discussed\footnote{\url{http://software-carpentry.org/blog/2011/12/fork-merge-and-share.html}}, where a tool like GitHub can be altered or built to be more focused towards education. The main weakness of GitHub when used in this context is in the lack of flexibility in its privacy and in the lack of administrative functions such as gradebooks and announcements. Meanwhile, there are open-source alternatives to GitHub such as GitLab\footnote{\url{https://about.gitlab.com}}, that could be further developed into a tool that fulfils more educational needs. As an example, it could be valuable to implement a form of announcements, a notification feature that students have more control over, and a way to make some discussions or issues within a repository private while others remain public. This is potentially an avenue for future work, where such a tool can be evaluated.

% As well, because of the exploratory nature of the work, we sought to obtain teacher and student perspectives regarding just the viability of GitHub as a tool for education. However, other studies have investigated using tools such as wikis \cite{minocha2007collaborative} and how they possibly affect or correlate with student performance. This is one natural extension of this work: running a field experiment to see whether or not using the tool simply engages the students more or if it can ultimately affect grades.

In summary, this work has shown the viability of using GitHub for education, and has demonstrated why the open, collaborative workflow associated with GitHub should be considered when deciding which tools to use to support a course. Based on the findings of this work, we included a set of recommendations for educators interested in using GitHub as a learning tool, and list the implications on tools that could provide the same benefits as GitHub while mitigating the limitations.

\balance

%ACKNOWLEDGMENTS are optional
%\section{Acknowledgments}

%
% The following two commands are all you need in the
% initial runs of your .tex file to
% produce the bibliography for the citations in your paper.
\bibliographystyle{abbrv}
\bibliography{icse_seet16}  % sigproc.bib is the name of the Bibliography in this case
% You must have a proper ".bib" file
%  and remember to run:
% latex bibtex latex latex
% to resolve all references
%
% ACM needs 'a single self-contained file'!
%

%\balancecolumns % GM June 2007
% That's all folks!
\end{document}
