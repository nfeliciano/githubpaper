%!TEX root = icse_seet16.tex
\section{Discussion}
The motivation behind this study was to uncover student perceptions on using GitHub as an educational tool during the experience. GitHub was used in three main ways: (a) as a place to disseminate material and host the class schedule, (b) as a place for students to submit their lab assignments and discuss these assignments, and (c) as a place where most students interviewed hosted their course projects, either collaboratively or alone.

% \textbf{A Student-Oriented Learning Tool} \\
%What does GitHub provide? more opportunities for students to participate and contribute!

%accomplishing tasks related to some of the finer-grain features of traditional LMSes, such as a formal assignment submission, requires workarounds. E

% Where GitHub has the potential to excel, however, is in addressing some of the concerns regarding traditional LMSes outlined by various authors, such as the `walled garden` approach of traditional LMSes \cite{mott2010envisioning}. This could be addressed by giving students opportunities to participate in the course and connect with and learn from each other. GitHub can support these opportunities for students to become a part of each others' learning, creating a culture of participation \cite{jenkins2009confronting}. \\

\textbf{GitHub Supports the Contributing Student} \\
At a basic level, using GitHub for education can provide similar functions to those of traditional LMSes. As discussed in the last chapter, GitHub has the capabilities of providing many of the common activities found in Malikowski \textit{et al.}'s model of features found in LMSes \cite{malikowski2007model}. In the courses highlighted in this study, GitHub supported the two of the primary uses of LMSes from this model: information transmission and class discussions. However, even though GitHub can serve a similar purpose to formal educational tools, it was simply not built for education and is therefore lacking some educational features.

GitHub, however, can excel by providing opportunities for students to participate in their learning to create a culture of participation \cite{jenkins2009confronting}. Students are able to openly contribute to the course materials by making changes or additions directly to a course repository. Although only a few sutdents participated in contributing to the course material, many felt that this activity would have seen more use had it been advertised more. This plays a key role in Collis and Moonen's concept of a `Contributing Student' \cite{collis2006contributing}, where GitHub provides students the ability to drive their coursework. Moreover, the collaborative features of GitHub provides students simplified opportunities to partake in many of the `Contributing Student Pedagogy' activities Hamer \textit{et al.} described \cite{hamer2011tools}, including peer reviews, discussion, content construction, solution sharing, and making links.

When student assignments and projects are public, GitHub can provide students the opportunity to contribute to other students' learning by easily providing direct feedback to each other's assignments or project work. A number of groups in one of the courses in this study used this ability by leaving feedback for other groups when they noted bugs or issues in the code, and students appreciated this ability to see others' work and provide feedback as they saw fit. Contributing to other students' work may provide benefits in developing soft skills such as communication and teamwork skills \cite{hamer2006some}. As well, repurposing or remixing code can be a strong assessment tool for students due to practicality and prevention of plaigiarism \cite{sant2015code}. An instructor may also utilize GitHub to provide opportunities for students to peer review or grade each other's work. This could provide potential benefits such as more reflection for students while working, and the development of analysis and evaluation skills \cite{sondergaard2012collaborative}.

% However, it is important to note that like any technology, accessing these benefits requires the stakeholders to `buy in' and use the relevant features of the tool to support this pedagogy. It is possible, for example, that there were different levels of enthusiasm for the tool between the two courses because of the differences in how it was used in the lab sessions. The SE case required students to post often, which possibly encouraged them to look at others' responses, while the CS did not utilize the tool as much, requiring only a demo of the weekly assignments to the lab instructor instead.

\textbf{Transparency of Activities} \\
%accountability
In describing the benefits of using GitHub to support their group projects, some students described the transparency of activities as helpful for collaborating with each other. Few of the transparency features of GitHub were mentioned by the students---for example, the News Feed or the graphs were not discussed in the context of group projects. However, some students acknowledged the importance of seeing a history of work from other group members, describing the feature as a way to hold accountability and to keep up-to-date with the work, gaining the awareness that can be important in collaborative learning \cite{janssen2013coordinated}. This is in line with the benefits related to GitHub use in industry \cite{dabbish2012social}.

%better grading from instructors
Moreover, some students described the potential for better grading methods as a benefit of the transparency of activities on GitHub, despite these courses not utilizing the tool for grading. Compared to the traditional way of assignment submission where an assignment is handed in as a complete product when it is due, GitHub offers instructors the opportunity to monitor assignments and projects, giving feedback while they are in progress, a useful exercise for both parties \cite{glassy2006using}.

\textbf{Beyond the Course} \\
%practice in tool
Supporting the findings from interviews with early adopters of GitHub in education \cite{zagalsky2015emergence}, most of the students interviewed described being exposed to GitHub and its features as a benefit to using the tool in a course. As such, the exposure to GitHub and its open, collaborative workflow may result in some transferable skills towards their careers. Moreover, the popularity of the tool means that students' GitHub accounts become part of their online presence \cite{marlow2013impression}, which may serve an important role with future collaborators or potential employers who use GitHub for hiring purposes.

%outside help
With GitHub's popularity, many developers are putting their code on the platform, both publicly or privately. When a course is publicly visible, the `walled garden' that traditional LMSes tend to suffer from \cite{mott2010envisioning} can be overcome. Student projects, for example, could involve people from another community, or outsiders can contribute to the course materials in some manner. \\

% \textbf{Tool Literacy} \\
% %privacy issues
% An important note from some of the limitations that the students and the instructors described is the importance of understanding and being proficient with the tool. As an example, in discussing what considerations need to be made to design an effective workflow, students would discuss the difficulty of conducting courses with single-solution assignments rather than open-ended projects. This was due to the way in which GitHub repositories are required to be private or public, making it difficult to handle assignment submission.
%
% However, some experience with the tool or some investigation of GitHub's recommended practices for using their tool in education would have revealed the possibility of using private repositories for each assignment. An instructor could introduce new assignments or make clarifications in a student's private repository if they were simply added as a collaborator. As such, it is important to consider that some of the limitations described by the students and by the instructor may be from unfamiliarity with using the tool, especially in a context it originally was not meant to serve.

%opportunistic
%group interviews
\subsection{Threats to Validity}
In this section, we discuss the threats to validity of our research.

\subsubsection{Internal Validity}
Internal validity is concerned with biases within a study that might affect our results \cite{creswell2013research}. In our research, the recruitment methods may have biased the population: by searching for instructors teaching appropriate courses, we first approached instructors we knew to invite them to participate in our study, which may have limited the study in comparison to finding a professor who was already intending to use GitHub as a learning tool and had experience in doing so. As well, opportunistic recruitment of the students may have resulted in a situation where the students willing to be interviewed were students who felt strongly about GitHub in either direction---those who may have had insights but had no strong opinions may have chosen not to participate. By interviewing many students however, we were able to discover any contradictions or discrepancies between them.

The data analysis also threatens the validity of the study as there was no inter-rater reliability because there was one coder. As such, biases may have been introduced in the selection of the themes. Having multiple raters analyze the data would have introduced more perspectives and interpretations, which would have reduced the potential biases in the analysis. The validation survey, however, mitigated some of the potential bias this limitation introduces, as the survey served as a form of member checking, allowing students to verify the analysis of the data.

\subsubsection{External Validity}
 External validity is concerned with the extent to which the findings from this work can be generalized and to what extent the findings are of interest to people outside the case \cite{runeson2012case}. As a case study, it cannot be assumed that these cases can be generalized to the use of GitHub in education. In particular, because of the instructor's lack of experience with using GitHub, these findings may vary from courses in which the instructor was able to offer more guidance with using the tool. However, many of the findings are reflected in other studies that use similar tools for classes, such as Kelleher's study on Git and GitHub \cite{kelleher2014employing}, and Haaranen and Lehtinen's study on Git and GitLab \cite{haaranen2015teaching}.

%However, we used a number of approaches\cite{runeson2012case} to establish rigor and to minimize the threats to validity described above. One such approach included triangulating data from multiple students. In doing so, we able to identify and highlight contradictions between students. Moreover, we deployed a survey to students in both courses in order to validate the themes extracted from the interviews. This is a form of member checking and allows students to verify the analysis of the data. %Finally, this study also involved a peer with whom I discussed the study with regularly. This peer, who is experienced with qualitative research methods, was consulted throughout the study, including the study design and the collection and analysis of data, and would lower the risk of any biases affecting the study.

%This resulted in a less than optimal use of GitHub (as described by many of the students interviewed) because the instructor had little prior experience with GitHub as a tool. As well, because of this inexperience, the researchers would give the instructor advice or resources on possibilities of how they can use GitHub to meet a goal---we didn't, however, directly give them step-by-step directions to avoid influencing the direction of the class as much as possible.

% In the data analysis, there was no inter-rater reliability because I was the sole coder. As such, biases may have been introduced in my selection of the themes. Having multiple raters analyze the data would have introduced more perspectives and interpretations, which would have reduced the potential biases in the analysis. Unfortunately, this study suffers from a single-rater limitation.

%As well, the opportunistic nature of recruitment may have resulted in possible biases in multiple ways. With only one instructor teaching two courses, this study is limited from having no other cases to compare with, particularly cases where the instructor might be experienced with using GitHub.

% However, we used a number of approaches\cite{runeson2012case} to establish rigor and to minimize the threats to validity described above. One such approach included triangulating data from multiple sources---the students as well as the teaching team. In doing so, I was able to identify and highlight contradictions between different sources and report discrepant information.
%
% This study was also conducted over the span of the two courses which provided prolonged involvement with the population and allowed for a good understanding of the participants' perspectives. Moreover, we deployed a survey to students in both courses in order to validate the themes extracted from the interviews. This is a form of member checking and allows students to verify the analysis of the data. %Finally, this study also involved a peer with whom I discussed the study with regularly. This peer, who is experienced with qualitative research methods, was consulted throughout the study, including the study design and the collection and analysis of data, and would lower the risk of any biases affecting the study.

%In summary, this study has shown the effectiveness of using GitHub for educational purposes from the student perspective. This study describes the benefits of using GitHub for education, such as the possibilities for student contributions. However, these benefits are accompanied by limitations, such as the implications of having publicly available work on cheating and academic integrity. In the next chapter, I offer recommendations for instructors who want to attempt using GitHub in their courses in order to maximize the benefits of using the tool.
