%!TEX root = icse_seet16.tex
\section{Discussion}
The motivation behind this study was to uncover student perceptions based on their experiences using GitHub as an educational tool. We noted that GitHub was used in three main ways: (a) as a place to disseminate material and host class schedules, (b) as a place for students to submit and discuss their lab assignments, and (c) as a place where most of our interviewees hosted their course projects, either collaboratively or alone.

% \textbf{A Student-Oriented Learning Tool} \\
%What does GitHub provide? more opportunities for students to participate and contribute!

%accomplishing tasks related to some of the finer-grain features of traditional LMSes, such as a formal assignment submission, requires workarounds. E

% Where GitHub has the potential to excel, however, is in addressing some of the concerns regarding traditional LMSes outlined by various authors, such as the `walled garden` approach of traditional LMSes \cite{mott2010envisioning}. This could be addressed by giving students opportunities to participate in the course and connect with and learn from each other. GitHub can support these opportunities for students to become a part of each others' learning, creating a culture of participation \cite{jenkins2009confronting}. \\

\textbf{GitHub Supports the Contributing Student} \\
At a basic level, GitHub can provide similar functions to those of traditional Learning Management Systems. It allows for many of the activities found in Malikowski \textit{et al.}'s model of features found in a traditional LMSes~\cite{malikowski2007model}. However, in the courses we investigated, GitHub was only used to support two of the primary uses from this model: information transmission and class discussions. And even though GitHub can serve a purpose similar to formal educational tools, it is important to note that it was not designed for education: GitHub is a framework, and therefore, it lacks features tailored specifically for education (e.g., grading features).
%\todo[inline]{CP: where are these lacking features discussed?}

GitHub does excel by creating a culture of participation \cite{jenkins2009confronting} and providing opportunities for students to participate in their learning. Students are able to openly contribute to course materials by making changes or additions directly to the course repository. This type of action plays a key role in Collis and Moonen's concept of a `Contributing Student' \cite{collis2006contributing} as GitHub provides students the ability to drive their coursework. Moreover, GitHub's collaborative features simplify many of the `Contributing Student Pedagogy' activities~\cite{hamer2011tools}, including peer reviews, discussions, content construction, solution sharing, and making links. Although only a few students contributed to the course materials in our study, many felt that this activity would have seen more use had it been advertised more.

When assignments and projects are public, GitHub provides people the opportunity to contribute to other students' learning with tools that allow them to easily provide direct feedback on assignments or project work. A number of groups in our study left feedback for other groups when they noted bugs or issues in the code, and students appreciated the ability to see others' work and provide feedback as they saw fit. Contributing to other students' work also helps refine soft skills such as communication and teamwork \cite{hamer2006some}. Additionally, re-purposing or remixing code requires students to show a deeper understanding of the material and the code involved~\cite{sant2015code}. An instructor may also utilize GitHub to provide opportunities for students to peer review or grade each other's work, which can help develop important analysis and evaluation skills \cite{sondergaard2012collaborative}.

% However, it is important to note that like any technology, accessing these benefits requires the stakeholders to `buy in' and use the relevant features of the tool to support this pedagogy. It is possible, for example, that there were different levels of enthusiasm for the tool between the two courses because of the differences in how it was used in the lab sessions. The SE case required students to post often, which possibly encouraged them to look at others' responses, while the CS did not utilize the tool as much, requiring only a demo of the weekly assignments to the lab instructor instead.

\textbf{Transparency of Activities} \\
%accountability
In describing the benefits of using GitHub to support their group projects, some students felt that the ability to see others' work encouraged collaboration. Students also acknowledged the importance of seeing the history of work from other group members, describing the feature as a way to hold people accountable and stay up to date with the work, gaining awareness that can be important in collaborative learning \cite{janssen2013coordinated}. This is in line with the benefits related to GitHub use in industry \cite{dabbish2012social}.
%\todo[inline]{CP: I'd list some examples.}

%better grading from instructors
Some students felt that GitHub's transparency could lead to better grading methods, despite these courses not utilizing the tool for grading. Compared to the traditional way of submitting assignments, where an assignment is handed in as a complete product when it is due, GitHub offers instructors the opportunity to monitor assignments and projects, giving feedback while they are in progress---a useful exercise for both parties \cite{glassy2006using}.

\textbf{Beyond the Course} \\
%practice in tool
Supporting our findings from interviews with early adopters of GitHub in education \cite{zagalsky2015emergence}, most of the students interviewed described being exposed to GitHub and its features during a course as a benefit---the exposure to GitHub's open and collaborative workflow may result in transferable skills for their careers. Moreover, the popularity of the tool means that students' GitHub accounts become part of their online presence \cite{marlow2013impression}, which may serve an important role with future collaborators or potential employers who use GitHub for hiring purposes.

%outside help
With GitHub's popularity, many developers are putting their code on the platform, both publicly and privately. When a course is publicly visible, the `walled garden' that traditional LMSes tend to suffer from \cite{mott2010envisioning} can be overcome. For example, student projects can involve people from another community, or outsiders can contribute to course materials in some manner. \\

% \textbf{Tool Literacy} \\
% %privacy issues
% An important note from some of the limitations that the students and the instructors described is the importance of understanding and being proficient with the tool. As an example, in discussing what considerations need to be made to design an effective workflow, students would discuss the difficulty of conducting courses with single-solution assignments rather than open-ended projects. This was due to the way in which GitHub repositories are required to be private or public, making it difficult to handle assignment submission.
%
% However, some experience with the tool or some investigation of GitHub's recommended practices for using their tool in education would have revealed the possibility of using private repositories for each assignment. An instructor could introduce new assignments or make clarifications in a student's private repository if they were simply added as a collaborator. As such, it is important to consider that some of the limitations described by the students and by the instructor may be from unfamiliarity with using the tool, especially in a context it originally was not meant to serve.
