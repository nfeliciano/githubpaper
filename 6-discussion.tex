\section{Discussion}

The motivation behind this study was to uncover student perceptions on using GitHub as an educational tool by gathering their opinions during the experience. GitHub was used in three main ways: (a) as a place to disseminate material and host the class schedule, (b) as a place for students to submit their lab assignments and discuss these assignments, and (c) as a place where most students interviewed hosted their course projects, either collaboratively or alone.

\textbf{A Student-Oriented Learning Tool} \\
%What does GitHub provide? more opportunities for students to participate and contribute!
At a basic level, using GitHub for education can provide similar functions to those of traditional LMSes. As discussed in the last chapter, GitHub has the capabilities of providing many of the common activities found in Malikowski \textit{et al.}'s model of features found in LMSes \cite{malikowski2007model}. However, even though GitHub can serve a similar purpose to formal educational tools, it was simply not built for education and is therefore lacking some educational features.
%accomplishing tasks related to some of the finer-grain features of traditional LMSes, such as a formal assignment submission, requires workarounds. E

% Where GitHub has the potential to excel, however, is in addressing some of the concerns regarding traditional LMSes outlined by various authors, such as the `walled garden` approach of traditional LMSes \cite{mott2010envisioning}. This could be addressed by giving students opportunities to participate in the course and connect with and learn from each other. GitHub can support these opportunities for students to become a part of each others' learning, creating a culture of participation \cite{jenkins2009confronting}. \\

% \textbf{The Contributing Student} \\
GitHub, however, can excel by providing opportunities for students to participate in their learning. Students are able to openly contribute to the course materials by making changes or additions directly to a course repository. Traditionally, students needed to talk to the instructor or send an email to make corrections or additions. GitHub provides a much more open and direct way for students to contribute to the course materials. This plays a key role in Collis and Moonen's concept of a `Contributing Student' \cite{collis2006contributing}, where GitHub provides students the ability to drive their coursework. Moreover, GitHub provides students opportunities to partake in many of the `Contributing Student Pedagogy' activities Hamer \textit{et al.} described \cite{hamer2011tools}, including peer reviews, discussion, content construction, solution sharing, and making links.

When student assignments and projects are public, GitHub can provide students the opportunity to contribute to other students' learning by easily providing direct feedback to each other's assignments or project work. A number of groups in one of the cases in this study used this ability by leaving feedback for other groups when they noted bugs or issues in the code, and students seemed to appreciate this ability to see others' work and provide feedback as they saw fit. Contributing to other students' work may provide benefits in developing soft skills such as communication and teamwork skills \cite{hamer2006some}. As well, repurposing or remixing code can be a strong assessment tool for students due to practicality and prevention of plaigiarism \cite{saint}. An instructor may also utilize GitHub to provide opportunities for students to peer review or grade each other's work. This could provide potential benefits such as more reflection for students while working, and the development of analysis and evaluation skills \cite{sondergaard2012collaborative}.

% However, it is important to note that like any technology, accessing these benefits requires the stakeholders to `buy in' and use the relevant features of the tool to support this pedagogy. It is possible, for example, that there were different levels of enthusiasm for the tool between the two courses because of the differences in how it was used in the lab sessions. The SE case required students to post often, which possibly encouraged them to look at others' responses, while the CS did not utilize the tool as much, requiring only a demo of the weekly assignments to the lab instructor instead.

% \textbf{Transparency of Activities} \\
%accountability
In describing the benefits of using GitHub to support their group projects, some students described the transparency of activities as helpful for collaborating with each other. Few of the transparency features of GitHub were mentioned by the students---for example, the News Feed or the graphs were not discussed in the context of group projects. However, some students acknowledged the importance of seeing a history of work from other group members, describing the feature as a way to hold accountability and to keep up-to-date with the work, gaining the awareness that can be important in collaborative learning \cite{janssen2013coordinated}. This is in line with the benefits related to GitHub use in industry \cite{dabbish2012social}.

%better grading from instructors
Moreover, some students described the potential for better grading methods as a benefit of the transparency of activities on GitHub, despite these courses not utilizing the tool for grading. Compared to the traditional way of assignment submission where an assignment is handed in as a complete product when it is due, GitHub offers instructors the opportunity to monitor assignments and projects, giving feedback while they are in progress, a useful exercise for both parties \cite{glassy2006using}.

% \textbf{Beyond the Course} \\
%practice in tool
Supporting the findings from interviews with early adopters of GitHub in education \cite{zagalsky}, most of the students interviewed described being exposed to GitHub and its features as a benefit to using the tool in a course. As such, the exposure to GitHub and its associated open, collaborative workflow may result in some transferable skills towards their careers. Moreover, the popularity of GitHub means that student GitHub accounts become part of their online presence \cite{MarlowDabbishHerbsleb}, which may serve an important role with future collaborators or potential employers who use GitHub for hiring purposes.

%outside help
With GitHub's popularity, many developers are putting their code on the platform, both publicly or privately. When a course is publicly visible, the `walled garden' that traditional LMSes tend to suffer from \cite{mott2010envisioning} can be overcome. Student projects, for example, could involve people from another community, or outsiders can contribute to the course materials in some manner. \\

% \textbf{Tool Literacy} \\
% %privacy issues
% An important note from some of the limitations that the students and the instructors described is the importance of understanding and being proficient with the tool. As an example, in discussing what considerations need to be made to design an effective workflow, students would discuss the difficulty of conducting courses with single-solution assignments rather than open-ended projects. This was due to the way in which GitHub repositories are required to be private or public, making it difficult to handle assignment submission.
%
% However, some experience with the tool or some investigation of GitHub's recommended practices for using their tool in education would have revealed the possibility of using private repositories for each assignment. An instructor could introduce new assignments or make clarifications in a student's private repository if they were simply added as a collaborator. As such, it is important to consider that some of the limitations described by the students and by the instructor may be from unfamiliarity with using the tool, especially in a context it originally was not meant to serve.

\subsection{Threats to Validity}
One limitation in this study was that the semi-structured nature of the interviews meant that the interviewer would often go off-script to probe further, potentially resulting in leading questions. Moreover, the recruitment methods may have biased the population: by searching for instructors teaching appropriate courses, we first approached instructors we knew to invite them to participate in my study, which may have introduced a bias in comparison to finding a class that was already intending to use GitHub as a learning tool. As well, opportunistic recruitment may have resulted in a situation where the students willing to be interviewed were students who felt strongly about GitHub in either direction---those who may have had insights but had no strong opinions may have chosen not to participate.

%This resulted in a less than optimal use of GitHub (as described by many of the students interviewed) because the instructor had little prior experience with GitHub as a tool. As well, because of this inexperience, the researchers would give the instructor advice or resources on possibilities of how they can use GitHub to meet a goal---we didn't, however, directly give them step-by-step directions to avoid influencing the direction of the class as much as possible.

% In the data analysis, there was no inter-rater reliability because I was the sole coder. As such, biases may have been introduced in my selection of the themes. Having multiple raters analyze the data would have introduced more perspectives and interpretations, which would have reduced the potential biases in the analysis. Unfortunately, this study suffers from a single-rater limitation.

%As well, the opportunistic nature of recruitment may have resulted in possible biases in multiple ways. With only one instructor teaching two courses, this study is limited from having no other cases to compare with, particularly cases where the instructor might be experienced with using GitHub.

% \subsection{External Threats to Validity}
% External validity is concerned with the extent to which the findings from this work can be generalized and to what extent the findings are of interest to people outside the case \cite{runeson2012case}. As a case study, it cannot be assumed that these cases can be generalized to the use of GitHub in education. However, many of the findings are reflected in other studies that use similar tools for classes, such as Kelleher's study on Git and GitHub \cite{kelleher2014employing}, and Haaranen and Lehtinen's study on Git and GitLab \cite{haaranen2015teaching}. %Another external threat to validity, is that because of the instructor's lack of experience with using GitHub, these findings may not be of interest to those who are familiar and can visualize a specific workflow for using GitHub in the classroom.

% However, we used a number of approaches\cite{runeson2012case} to establish rigor and to minimize the threats to validity described above. One such approach included triangulating data from multiple sources---the students as well as the teaching team. In doing so, I was able to identify and highlight contradictions between different sources and report discrepant information.
%
% This study was also conducted over the span of the two courses which provided prolonged involvement with the population and allowed for a good understanding of the participants' perspectives. Moreover, we deployed a survey to students in both courses in order to validate the themes extracted from the interviews. This is a form of member checking and allows students to verify the analysis of the data. %Finally, this study also involved a peer with whom I discussed the study with regularly. This peer, who is experienced with qualitative research methods, was consulted throughout the study, including the study design and the collection and analysis of data, and would lower the risk of any biases affecting the study.

%In summary, this study has shown the effectiveness of using GitHub for educational purposes from the student perspective. This study describes the benefits of using GitHub for education, such as the possibilities for student contributions. However, these benefits are accompanied by limitations, such as the implications of having publicly available work on cheating and academic integrity. In the next chapter, I offer recommendations for instructors who want to attempt using GitHub in their courses in order to maximize the benefits of using the tool.

\subsection{Recommendations for Educators}
This section provides recommendations for educators who want to use GitHub to support their courses. These recommendations are based on the findings from the two phases of this work, as well as from the review of literature surrounding tools in computer science and software engineering education.

Before proceeding, we note that GitHub has their own set of recommendations for setting up an organization for a class\footnote{\url{https://education.github.com/guide}}. Their classroom guide is useful for those looking for a step-by-step process, where they recommend applying for an organization for a course and assigning a private repository for each assignment for each student. Likewise, it can also be helpful to use the available resources: use GitHub support, look for other instructor experiences for guidance, or discuss experiences in a blog or in spaces dedicated to the topic\footnote{\url{https://github.com/education/teachers/issues}}. Contributing to these resources can serve towards building a common knowledge base for instructors to share to and learn from. Moreover, GitHub recently released a tool that automates many of the tasks educators need to set up on GitHub \footnote{\url{https://classroom.github.com}}. \\

% \textbf{Recommendation: Utilize GitHub's Features} \\
Computer science and software engineering students benefit from early exposure to Git and GitHub. By utilizing these (or similar) tools in their courses, educators provide students a way to familiarize themselves and practice with these tools, which can benefit their careers. Beyond exposure, hosting assignments, projects, and code on student accounts could be valuable when seeking employment, as companies continue to investigate the online presence prospective employees have (e.g., their GitHub accounts) for hiring purposes.

While simply using GitHub as a system for material dissemination can be helpful, using more of GitHub's features, such as pull requests and issues, provides even more benefits for the students. For example, allowing students to contribute to the course and to each other's work can help develop skills such as teamwork and communication \cite{hamer2006some}. For example, educators can use GitHub's transparency features to provide feedback to students in unique ways, such as tracing the history of student projects and assignments hosted on GitHub, detailing where students made mistakes and intervening when a student seems to be struggling. Moreover, in group projects, instructors can note how much work each student has contributed, and can use this transparency for assigning grades.

%As another example, exposure to GitHub's Issues feature, even for basic discussions, was helpful for one of the students interviewed during the second phase as the student learned how the feature works for use in future projects.
\\
%One important lesson noted from the case study was to communicate the workflow the instructor decides clearly and properly to the teaching team and to the students. When deciding to use a feature like pull requests on course material, for example, the instructor must advertise this workflow properly, perhaps even offering bonus points for added material. To communicate a workflow to students and introduce GitHub and its features to novices, instructors should consider creating a guide or hosting a tutorial session. \\

%notifications

% \textbf{Recommendation: Use Free Private Repositories for Single Solution Assignments} \\
%type of course - better for open-ended
%assignment submission
%something on Project BasedLearning here?
Many students believed that GitHub worked best when a course has open-ended projects and assignments. This stems from plaigarism concerns that exist when students are putting their code up online where others can potentially see their solutions. Of course, students can submit their assignments in private repositories that only the instructors can view and contribute to. However, single-solution assignments being hosted in private repositories limit one of the most important benefits of using a system like GitHub---the ability to view, comment on, and contribute to the work of other students. As such, although GitHub can be used in any type of course, its benefits are maximized in courses with open-ended projects and courses where student contributions and participations.

%; if the instructor creates a private repository for each student to submit their assignments and adds only the student as a collaborator, plaigarism would only be as much of a concern as it would be without using GitHub. Otherwise, an instructor could ask students to create a private repository for their assignments that only the instructors can view and contribute to.

% This style of repository management (where a private repository is dedicated to each student) could work for assignment submission as well. The instructor could ask the students to create a branch, or ask the students to fork off the main repositories and make the forks private, and then mandate that the student must make a pull request before a deadline. Thanks to GitHub's transparency features, an instructor can continuously observe the work in each student's repository and can provide further assistance to students based on the work history.

% However, the set up for this more private style of repository management requires some time and assistance from GitHub. An educator can create an organization for the course, which is granted an amount of private repositories depending on how much the instructor pays. While GitHub has stated that they would give teachers a free organization for their courses\footnote{\url{https://github.com/blog/1775-github-goes-to-school}}, an organization must be set up well before the course begins in order to get the private repositories in time.

 % As such, although GitHub is usable and helpful in any type of course, courses with open-ended projects and courses with a culture of participation are where instructors and students will see the primary benefits of using GitHub as a learning tool. If an instructor chooses to pursue the open-ended style of work similar to the courses in this study, it is recommended that they list projects and assignments on the home page using the readme markdown file so students can easily access the other projects.
\\
% That said, GitHub continues to offer its benefits when used to submit single solution assignments. It involves some preparation to get free private repositories for students, but at the same time, it allows instructors to provide better feedback through versioning, and it maintains the benefits for students of learning Git and GitHub and hosting their work for future portfolio use (if allowed to publicize their work after the course concludes). \\

% \textbf{Recommendation: Encourage Contributions from the Students} \\
%contributions from others (slides in html, comment on other projects issues)
Another recommendation is to encourage contribution from the students in the ways that GitHub affords them. First, students can contribute to the course materials by making corrections, changes, and adding resources. Second, students can contribute to other students' work and projects (provided the work is open-ended), bringing in an element of peer review that students may benefit from \cite{sondergaard2012collaborative}. And third, students can contribute to projects outside the course by making changes and pull requests in open-source repositories. Encouraging this `Contributing Student Pedagogy' can help students develop skills such as critical analysis and collaboration \cite{falkner2012supporting}.

% Moreover, all student contributions are available for the course instructor to see. As an example, an instructor can grade students based on their contributions, such as when they create an issue or a pull request on another project. However, one issue with student contributions that must be noted is that contributing to the course materials could present difficulties depending on the file types used, as binary files such as PDF documents and PowerPoint slides are not compatible with the GitHub web interface. Although GitHub has recently provided support for viewing PDF files on the platform\footnote{\url{https://github.com/blog/1974-pdf-viewing}}, these files remain unsupported by GitHub's `diff' feature, which means that changes to the file are difficult to discern and changes to the file by multiple people will always result in a `merge conflict'. For this reason, I recommend hosting class material and slides in either markdown or HTML, file types that GitHub supports and can be easily altered using its Web platform.
